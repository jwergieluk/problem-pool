



\paragraph{Stoppzeiten. } Seien $T,S$ Stoppzeiten bezüglich der Filtration
$\left( \cF_t \right)_{t \geq 0}$. 
\begin{enumerate}
    \item Ist $T \equiv t_0>0$ eine fixe Zeit, so ist $\cF_T=\cF_{t_0}$.
\end{enumerate}

\paragraph*{Lösung. }
\begin{enumerate}
    \item \begin{eqnarray}
            \cF_T &=& 
            \left\{ A\in\bF : A \cap \left\{ t_0 \leq t \right\}\in\cF_t \ \forall t\geq 0 \right\} \\
            &=& \left\{ A\in\bF : A\in\cF_t \ \forall t\geq t_0 \right\} = \cF_{t_0}.
        \end{eqnarray}
\end{enumerate}


\paragraph{Stoppzeiten und optionale Zeiten. } Eine optionale Zeit $T$ bezüglich
der Filtration $\bF=\left( \cF_t \right)_{t \geq 0}$ ist eine Stoppzeit, wenn 
$\bF$ rechtsstetig ist, d.h.\ $\cF_t=\cF_{t+}$ $\forall t$ gilt.

\paragraph*{Lösung.}  Betrachte die Darstellung
\begin{eqnarray}
    \left\{ T \leq t \right\} = \bigcap_{\varepsilon\in\bQ^{+}} \left\{ T > t+\varepsilon \right\}
    = \lim_{\varepsilon\to 0, \varepsilon\in\bQ} \left\{ T > t+\varepsilon \right\}.
\end{eqnarray}
Der Schnitt ist als Mengengrenzwert zu verstehen, weil die Mengen $\left\{ T >
t+\varepsilon \right\}$ ineinandergeschachtelt sind. $\left\{ T \leq t \right\}\in \bF_{t+}$
genau dann, wenn $\left\{ T \leq t \right\}\in \bF_{t+\varepsilon^{*}}$ $\forall \varepsilon^{*}>0$.
Es gilt aber für ein fixes $\varepsilon^{*}$
\begin{equation}
    \left\{ T \leq t \right\}\ = 
    \bigcap_{\varepsilon\in\bQ^{+}, \varepsilon\leq \varepsilon^{*}} \left\{ T > t+\varepsilon \right\} 
    \in\bF_{t+\varepsilon^{*}},
\end{equation}
denn es ein abzählbarer Schnitt der Mengen $\left\{ T> t+\varepsilon
\right\}\in\bF_{t+\varepsilon^{*}}$ ist. Behauptung gilt wegen $\cF_t=\cF_{t+}$ $\forall t$.




\paragraph{Ornstein-Uhlenbeck SDE.} Lösen Sie die \textsc{SDE}
\begin{equation}
    dX_t = \left( \alpha-\beta X_t \right)dt + \sigma dW_t, \quad X_0\in\R.
\end{equation}
Die Parameter $\alpha, \beta$ und $\sigma$ sind reelle Zahlen und $\sigma>0$. 
\begin{enumerate}
    \item Zeigen Sie mit Hilfe der Funktion $f(t,x)=x\exp(\beta t)$ und der It\^o-Formel, dass
        \begin{eqnarray}
            X_t = e^{-\beta t} X_0 + \frac{\alpha}{\beta}\left( 1-e^{-\beta t} \right)
                       + \sigma \int_{0}^{t} e^{\beta(s-t)} d W_s.
        \end{eqnarray}
    \item Leiten Sie aus der von Ihnen gefundenen Lösung die Identitäten
        \begin{eqnarray}
            \E X_t &=& e^{-\beta t} X_0 + \frac{\alpha}{\beta}\left( 1-e^{-\beta t} \right) \\
            \Var X_t &=& \frac{\sigma^2}{2\beta}\left( 1- e^{-2\beta t} \right)
        \end{eqnarray} 
        her.
    \item Zeigen Sie, dass $X_t$ normalverteilt ist. 
\end{enumerate}

\ifdefined\showsolutions
\paragraph*{Lösung. } Die Lösung ergibt sich durch die Anwendung der It\^o-Formel
auf die Funktion $f$. Da $\int_{}^{} g(s) d W_s$ für progressiv meßbares $g$ ein
Martingal ist, erhalten wir $\E X_t$ direkt aus der Lösung. Die Formel für die 
Varianz folgt mit Hilfe der It\^o-Isometrie.
\fi



