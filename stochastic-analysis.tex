
\section{Messbarkeit und Filtrationen}

\paragraph{Pfadeigenschaften und Filtration. }  Sei $X$ ein stochastischer
Prozess, $\cF^{X}$ die von $X$ erzeugte Filtration und $\cG$ eine Filtration,
die alle $\cF$-Nullmengen enthält und $\cF^{X}_t \subset \cG_t$ $\forall t$
erfüllt.  Sei $A \subset \Omega$ das Ereignis, dass $X$ stetig auf $[0, t_0)$
ist. 
\begin{enumerate}
    \item Falls alle Pfade von $X$ c\`adl\`ag sind, dann ist $A \in \cF^{X}_{t_0}$.
    \item Falls $X$ c\`adl\`ag ist, dann gilt $A \in \cG_{t_0}$, aber nicht
        notwendigerweise $A \in \cF^{X}_{t_0}$.
\end{enumerate}
\cite{Karatzas1991}.

\paragraph{Messbarkeitsbegriffe. } Sei $X$ ein stochastischer Prozess. Ist $X$
progressiv messbar, so folgt $X$ ist messbar und adaptiert.


\section{Stoppzeiten}

\paragraph{Stoppzeiten. } Seien $T,S$ Stoppzeiten bezüglich der Filtration
$\left( \cF_t \right)_{t \geq 0}$. 
\begin{enumerate}
    \item Ist $T \equiv t_0>0$ eine fixe Zeit, so ist $\cF_T=\cF_{t_0}$.
\end{enumerate}

\paragraph*{Lösung. }
\begin{enumerate}
    \item \begin{eqnarray}
            \cF_T &=& 
            \left\{ A\in\bF : A \cap \left\{ t_0 \leq t \right\}\in\cF_t \ \forall t\geq 0 \right\} \\
            &=& \left\{ A\in\bF : A\in\cF_t \ \forall t\geq t_0 \right\} = \cF_{t_0}.
        \end{eqnarray}
\end{enumerate}


\paragraph{Stoppzeiten und optionale Zeiten. } Eine optionale Zeit $T$ bezüglich
der Filtration $\bF=\left( \cF_t \right)_{t \geq 0}$ ist eine Stoppzeit, wenn 
$\bF$ rechtsstetig ist, d.h.\ $\cF_t=\cF_{t+}$ $\forall t$ gilt.

\paragraph*{Lösung.}  Betrachte die Darstellung
\begin{eqnarray}
    \left\{ T \leq t \right\} = \bigcap_{\varepsilon\in\bQ^{+}} \left\{ T > t+\varepsilon \right\}
    = \lim_{\varepsilon\to 0, \varepsilon\in\bQ} \left\{ T > t+\varepsilon \right\}.
\end{eqnarray}
Der Schnitt ist als Mengengrenzwert zu verstehen, weil die Mengen $\left\{ T >
t+\varepsilon \right\}$ ineinandergeschachtelt sind. $\left\{ T \leq t \right\}\in \bF_{t+}$
genau dann, wenn $\left\{ T \leq t \right\}\in \bF_{t+\varepsilon^{*}}$ $\forall \varepsilon^{*}>0$.
Es gilt aber für ein fixes $\varepsilon^{*}$
\begin{equation}
    \left\{ T \leq t \right\}\ = 
    \bigcap_{\varepsilon\in\bQ^{+}, \varepsilon\leq \varepsilon^{*}} \left\{ T > t+\varepsilon \right\} 
    \in\bF_{t+\varepsilon^{*}},
\end{equation}
denn es ein abzählbarer Schnitt der Mengen $\left\{ T> t+\varepsilon
\right\}\in\bF_{t+\varepsilon^{*}}$ ist. Behauptung gilt wegen $\cF_t=\cF_{t+}$ $\forall t$.


\section{Brownsche Bewegung}

\paragraph{Brownsche Bewegung. Invariante Transformationen und der isonormale Prozess.}


\paragraph{Brownsche Bewegung. Lokales Verhalten. }
Sei $B$ eine standard Brownsche Bewegung auf $(\Omega, \cF, P)$. Definiere die
Niveaumenge 
\begin{equation*}
    \cL_{\omega}^{\alpha} = \left\{ t\in\R_{\geq 0} : B_t(\omega)=\alpha \right\}
\end{equation*}
für ein $\omega\in\Omega$ und ein $\alpha\in\R$. 
Beweisen Sie folgende Aussagen:
\begin{enumerate}
    \item $\limsup_{t\to \infty} \frac{B_t}{\sqrt{t}} > 0$, $P$ fast sicher. 
    \item Die Niveaumenge $\cL_\omega^{0}$ hat an Null einen Häufungspunkt
        für fast alle $\omega\in \Omega$. 
    \item Brownsche Bewegung ist rekurrent, d.h.\ $L_\omega^{\alpha}$ ist für alle
        $\alpha\in\R$ $P$ fast sicher unbeschränkt. 
    \item Die Pfade von $B$ sind f.s.\ nirgends lokal Hölder stetig für alle 
        $\alpha>\frac{1}{2}$. Eine reellwertige Funktion $f$ ist lokal Hölder
        stetig der Ordnung $0<\alpha\leq 1$, wenn 
        \begin{equation*}
            \sup_{t,s} \left\{ \frac{ | f(t)-f(s) | }{ | t-s |^{\alpha}} : 
            |t|,|s| \leq L, t\neq s \right\} < \infty
        \end{equation*}
        für alle positiven $L$ gilt. 
\end{enumerate}
\paragraph*{Lösung.} 
\begin{enumerate}
    \item Wir setzen $A=\limsup_{t\to \infty} \frac{B_t}{\sqrt{t}}$. Angenommen
        es gilt $A\leq 0$, $P$-f.s. Aus Symmetriegründen muss also
        \begin{equation*}
            \lim_{t\to\infty} \frac{B_t}{\sqrt{t}} = 0 \ P-\text{f.s.}
        \end{equation*}
        gelten. Mit $\frac{B_t}{\sqrt{t}}\sim \cN(0,1)$ erhalten wir einen Widerspruch.
    \item Um das obige Resultat zu benutzen, berechnen wir
        \begin{equation*}
            \limsup_{t\to\infty} \frac{B_t}{\sqrt{t}} = 
            \limsup_{t\to 0} \sqrt{t} B_{1/t} = 
            \limsup_{t\to 0} \frac{t B_{1/t}}{\sqrt{t}}. 
        \end{equation*}
        Der Prozess $X_t = t B_{1/t}$ ist ebenfalls eine Brownsche Bewegung und
        es gilt
        \begin{align*}
            \limsup_{t\to 0} \frac{X_t}{\sqrt{t}} & > 0 \ P-\text{f.s.} \\
            \liminf_{t\to 0} \frac{X_t}{\sqrt{t}} & < \ P-\text{f.s.}.
        \end{align*}
        Fast jeder Pfad von $X$ muss in jeder Umgebung von $0$ unendlich oft
        das Vorzeichen wechseln, also hat die Menge der Nullstellen einen
        Häufungspunkt. 
    \item Nachdem die Nullstellen fast aller Pfade von $B$ einen Häufungspunkt bei
        $0$ haben und der Prozess $X_t = t B_{1/t}$ eine Brownsche Bewegung ist,
        ist die Menge der Nullstellen unbeschränkt. Somit erhalten wir 
        $\cL^{0}$ fast sicher unbeschränkt. \todo{Beweis für beliebige Niveaus fehlt noch. }
    \item Für den Beweis der Hölderstetigkeit betrachten wir
        \begin{equation*}
            \frac{| B_t - B_s|}{ |t-s|^{\frac{1}{2}+\varepsilon} } = 
            \frac{X}{|t-s|^{\varepsilon}}
        \end{equation*}
        wobei
        \begin{equation*}
            X = \frac{| B_t - B_s|}{ |t-s|^{\frac{1}{2}}} \sim \cN(0,1).
        \end{equation*}
        Nachdem aber $\varepsilon>0$, ist
        \begin{equation*}
            \sup_{|t|,|s| \leq L, t\neq s} \frac{|X|}{|t-s|^{\varepsilon}}
        \end{equation*}
        unbeschränkt. 
\end{enumerate}


\paragraph{Maximum processes of the Brownian Motion. } Show that the process $W
- W^*$ has independent increments. $W$ denotes Brownian Motion and $X^*_t=
\max_{0 \leq s \leq t} X_s$ is the maximum process of a \cadlag process $X$.



\section{L\'evy Prozesse}

\paragraph{L\'evy martingales.} Every Levy process which is a local martingale is
also a true martingale.




\section{Markov Prozesse}

\paragraph{Not Feller by discontinuity at zero}
Find a Markov process viotating the continuity part of 
the Feller property, namely
\begin{equation}
\lim_{t\to 0} P_t f(x) = f(x) \quad \forall f \in C_0 \ \forall x\in \R
\end{equation}
but perhaps preserving the $C_0$ property $P_t C_0 \subset C_0$.



\paragraph{Independence and transformations. } Let random variables
$X_1,\ldots,X_n$ and $Y_1,\ldots,Y_k$ satisfy $X_i \upmodels Y_j$ for all $i$
and $i$.
\begin{enumerate}
    \item It follows that $g(X_i)$ and $h(Y_j)$ are independent for any $i$ and
        $j$.
    \item A generalization of the above result is not true. Given any Borel
        functions $f$ and $g$ of suitable dimensionality $f(X_1,\ldots,X_n)$
        may be not independent of $g(Y_1,\ldots,Y_k)$.
\end{enumerate}



\section{Stochastische Integration}

\paragraph{Stieltjes-Integrale der Brownschen Bewegung.} 
Sei $B$ eine Brownsche Bewegung und eine Funktion $X:\Omega\to\R$ definiert durch
\begin{equation*}
    X(\omega) = \int_{0}^{1} B_s^{2} ds. 
\end{equation*}
\begin{enumerate}
    \item Zeigen Sie, dass $X$ eine Zufallsvariable ist.
    \item Berechnen Sie den ersten und zweiten Moment von $X$. 
\end{enumerate}



\paragraph{Stochastisches Integral einer deterministischen Funktion. } Sei
$f: \R\in\R$ eine deterministische meßbare Funktion. Zeigen Sie 
\begin{equation}
    \int_{0}^{t} f(s) d W_s \sim \cN(0, \int_{0}^{t} f^2(s) ds). 
\end{equation}








\section{Stochastische Differentialgleichungen. }

\paragraph{Ornstein-Uhlenbeck SDE.} Lösen Sie die \textsc{SDE}
\begin{equation}
    dX_t = \left( \alpha-\beta X_t \right)dt + \sigma dW_t, \quad X_0\in\R.
\end{equation}
Die Parameter $\alpha, \beta$ und $\sigma$ sind reelle Zahlen und $\sigma>0$. 
\begin{enumerate}
    \item Zeigen Sie mit Hilfe der Funktion $f(t,x)=x\exp(\beta t)$ und der It\^o-Formel, dass
        \begin{eqnarray}
            X_t = e^{-\beta t} X_0 + \frac{\alpha}{\beta}\left( 1-e^{-\beta t} \right)
                       + \sigma \int_{0}^{t} e^{\beta(s-t)} d W_s.
        \end{eqnarray}
    \item Leiten Sie aus der von Ihnen gefundenen Lösung die Identitäten
        \begin{eqnarray}
            \E X_t &=& e^{-\beta t} X_0 + \frac{\alpha}{\beta}\left( 1-e^{-\beta t} \right) \\
            \Var X_t &=& \frac{\sigma^2}{2\beta}\left( 1- e^{-2\beta t} \right)
        \end{eqnarray} 
        her.
    \item Zeigen Sie, dass $X_t$ normalverteilt ist. 
\end{enumerate}

\paragraph*{Lösung. } Die Lösung ergibt sich durch die Anwendung der It\^o-Formel
auf die Funktion $f$. Da $\int_{}^{} g(s) d W_s$ für progressiv meßbares $g$ ein
Martingal ist, erhalten wir $\E X_t$ direkt aus der Lösung. Die Formel für die 
Varianz folgt mit Hilfe der It\^o-Isometrie.



