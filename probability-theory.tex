
\section{Raw material}

\paragraph{Minimum von zwei exponentialverteilten Zufallsvariablen.}
\ldots ist wieder exponentialverteilt. Gilt das auch das Minimum von $n$
exponential verteilten Zufallsvariablen?


\section{Production-ready exercises}

\paragraph{Unabhängigkeit.} 
Sei $X$ eine Zufallvariable. $X$ ist von $X$ unabhängig, genau dann wenn
$P(X = c)=1$ für eine Konstante $c$ gilt.

\paragraph*{Lösung.} Jacod Protter 9.2. 


\paragraph{Eine Darstellung des Erwartungswertes. } Sei $X$ eine
positive Zufallsvariable mit $\E X < \infty$. 
\begin{enumerate}
    \item Ist $X \in \bN$ diskret, so gilt
        \begin{equation}
            E X = \sum_{n\in\bN}^{} P(X>n).
        \end{equation}
    \item Ist $X$ reellwertig, so gilt
        \begin{equation}
            E X = \int_{0}^{\infty} P(X>\lambda) d\lambda.
        \end{equation}
    \item Ist $X$ reellwertig und $E X^p<\infty$ für ein $p>0$, so gilt
        \begin{eqnarray}
            E X^p = \int_{0}^{\infty} p\lambda^{p-1} P(X>\lambda) d\lambda.
        \end{eqnarray}
\end{enumerate} 

\paragraph*{Lösung.} Das ist eine Anwendung des Satzes von Fubini.


\paragraph{Gleichverteilung und Exponentialverteilung. } Sei $X$ gleichverteilt
auf dem Intervall $[0,1]$, dann ist $Y = -\log X$ exponentialverteilt.

\paragraph*{Lösung. } $Y$ ist exponentialverteilt mit $\lambda=1$.


\paragraph{Beta-verteilung.} Sei $X$ eine Beta$(2,2)$-verteilte Zufallsvariable.
\begin{enumerate}
    \item Geben Sie die Dichte und Verteilungsfunktion von $X$ explicit an. 
    \item Geben Sie die Dichte und Verteilungsfuntion der Zufallsvariable
        $Y = \frac{1}{X} - 1$ an. 
    \item Berechnen Sie $\E Y$.  
\end{enumerate}



\paragraph{Erwartungstreue der Stichprobenvarianz. }  Seien $X_1,\ldots,X_n$ 
i.i.d. (independent indentically distributed = unabhängig identisch verteilt) mit 
Varianz $\sigma^2$. Die Stichprobenvarianz ist definiert durch
\begin{equation*}
    s^2(X) := \frac{1}{n-1} \sum_{i=1}^{n} (X_i - \bar X)^2.
\end{equation*}
Dann gilt $E(s^2(X)) = \sigma^2$, d.h.~die Stichprobenvarianz ist erwartungstreu.
% (A 1.3)


\paragraph{Geraden Momente der Normalverteilung. } Zeigen Sie, dass für eine
standardnormalverteilte Zufallsvariable $X$ und $n \in \N$
\begin{equation*}
    E\left( X^{2n} \right) = \frac{(2n)!}{2^n \cdot n!}
\end{equation*}
gilt. % A1.11


\paragraph{Faltungsformel. } Haben $X$ und $Y$ die Dichten $p_X$ und $p_Y$ und beide
sind unabhängig, so ist die Dichte von der Summe $Z=X+Y$ gegeben durch
\begin{equation*}
    p_Z(z) = \int_{\R}^{} p_X(x)p_Y(z-x) dx.
\end{equation*}
% (A 1.30)


\paragraph{Momentenerzeugende Funktion einer Gamma-Verteilung. }
Sei $X \sim \textrm{Gamma}(a, \lambda)$. Zeigen Sie, dass für $s<\lambda$
\begin{equation*}
    \Psi_X(s) = E \left( e^{sX} \right) = \frac{\lambda^a}{ (\lambda-s)^a}
\end{equation*}
gilt. Bestimmen Sie damit den Erwartungswert und die Varianz von $X$. %A 1.12



\paragraph{Eine Transformation der Gamma-Verteilung. } Seien $X \sim
\textrm{Gamma}(\alpha_1, \beta)$ und $Y \sim \textrm{Gamma}(\alpha_2, \beta)$
zwei unabhängige Zufallsvariablen.  Zeigen Sie, dass $\frac{X}{ X+Y}$ und $X+Y$
unabhängig sind. % (A 1.7)




\paragraph{Summen von normalverteilten Zufallsvariablen. }
Sind die Zufallsvariablen $X_1,\ldots,X_n$ unabhängig und normalverteilt
mit $X_i \sim \mathcal N (\nu_i, \sigma_i^2)$, so ist die Summe
wieder normalverteilt mit 
\begin{equation*}
    \sum_{i=1}^{n} X_i \sim 
    \mathcal N \left( \sum_{i=1}^{n} \nu_i, \sum_{i=1}^{n} \sigma_i^2 \right). 
\end{equation*}
%(A 1.31)

\paragraph{Dichte der multivariaten Normalverteilung. } Zeigen Sie, dass 
$X \sim \mathcal N_p\left( \mu, \Sigma \right) $ folgende Dichte hat, falls 
$\textrm{Rang}(\Sigma)=p$: 
\begin{equation*}
    p(x) = \frac{1}{ \textrm{det}(\Sigma)^{1/2} (2\pi)^{p/2}} \exp\left( - \frac{1}{2} (x  - \mu)^\top \Sigma^{-1} (x - \mu) \right).
\end{equation*}  %(A 1.37)


\paragraph{Rayleigh-Verteilung. }  Seien $X$ und $Y$ unabhängig und 
$\mathcal N (0, \sigma^2)$-verteilt. Dann ist
\begin{equation*}
    Z= \sqrt{X^2 + Y^2} 
\end{equation*}
Rayleight-verteilt, d.h. 
$Z$ hat die Dichte $\frac{z}{\sigma^2}e^{-\frac{z^2}{2\sigma^2}}1_{\R>0}(z)$.
Es gilt $E\left( Z \right) = \sigma \sqrt{\pi/2}$ und 
$\Var\left( Z \right) = \sigma^2 \left( 2 - \frac{2}{\pi} \right)$.


\paragraph{Dichte der $\chi^2$-Verteilung. }   Seien $X_1,\ldots,X_n$ unabhängige und
standardnormalverteilte Zufallsvariablen. Dann folgt $Y= \sum_{i=1}^{n} X_i^2$ einer
$\chi^2$-Verteilung mit $n$ Freiheitsgraden. Zeigen Sie, dass die Dichte von $Y$ durch
\begin{equation*}
    p(x) = \frac{1}{ 2^{\frac{n}{2}} \Gamma(\frac{n}{2})}e^{-\frac{x}{2}} x^{\frac{n-2}{2}} 1_{\R>0}(x)
\end{equation*}
gegeben ist. Verwenden Sie hierfür die Faltungsformel und die Beta-Funktion.



\paragraph{Verteilung der Stichprobenvarianz. }
Seien $X_1, \ldots, X_n$ i.i.d., normalverteilt und $\Var\left( X_1 \right)=\sigma^2$.
Für das zweite zentrierte empirische  Moment 
$\hat \sigma^2 \left( X \right)= \frac{1}{n} \sum_{i=1}^{n} \left( X_i - \bar X \right)^2 $
gilt, dass
\begin{equation*}
    \frac{n \hat \sigma^2 \left( X \right)}{\sigma^2 } = 
        \sum_{i=1}^{n} \left( \frac{X_i - \bar X}{ \sigma} \right)^2 \sim \chi^2_{n-1}.
\end{equation*}

\paragraph{Mittelwertvergleich bei Gamma-Verteilungen. }
Seien $X_1,\ldots,X_n$ i.i.d.\ und $\textrm{Gamma}(a, \lambda_1)$-verteilt sowie
$Y_1,\ldots,Y_n$ i.i.d.\ und $\textrm{Gamma}(a,\lambda_2)$-verteilt. Man nehme an,
dass die Vektoren $(X_1,\ldots,X_n)$ und $(Y_1,\ldots,Y_n)$ unabhängig sind. 
Bestimmen Sie die Verteilung von $\bar X/\bar Y$.




% vim: set spelllang=de; set spell
