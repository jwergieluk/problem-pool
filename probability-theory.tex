

\paragraph{Minimum von zwei exponentialverteilten Zufallsvariablen. } \ldots ist 
wieder exponentialverteilt. Gilt das auch das Minimum von $n$ exponential
verteilten Zufallsvariablen?


\paragraph{Eine Darstellung des Erwartungswertes. } Sei $X$ eine
positive Zufallsvariable mit $\E X < \infty$. 
\begin{enumerate}
    \item Ist $X \in \bN$ diskret, so gilt
        \begin{equation}
            E X = \sum_{n\in\bN}^{} P(X>n).
        \end{equation}
    \item Ist $X$ reellwertig, so gilt
        \begin{equation}
            E X = \int_{0}^{\infty} P(X>\lambda) d\lambda.
        \end{equation}
    \item Ist $X$ reellwertig und $E X^p<\infty$ für ein $p>0$, so gilt
        \begin{eqnarray}
            E X^p = \int_{0}^{\infty} p\lambda^{p-1} P(X>\lambda) d\lambda.
        \end{eqnarray}
\end{enumerate} 

\paragraph*{Lösung.} Das ist eine Anwendung des Satzes von Fubini.


\paragraph{Gleichverteilung und Exponentialverteilung. } Sei $X$ gleichverteilt
auf dem Intervall $[0,1]$, dann ist $Y = -\log X$ exponentialverteilt.

\paragraph*{Lösung. } $Y$ ist exponentialverteilt mit $\lambda=1$.



\paragraph{Beta-verteilung.} Sei $X$ eine Beta$(2,2)$-verteilte Zufallsvariable.
\begin{enumerate}
    \item Geben Sie die Dichte und Verteilungsfunktion von $X$ explicit an. 
    \item Geben Sie die Dichte und Verteilungsfuntion der Zufallsvariable
        $Y = \frac{1}{X} - 1$ an. 
    \item Berechnen Sie $\E Y$.  
\end{enumerate}






% vim: set spelllang=de; set spell
