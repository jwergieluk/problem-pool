\documentclass[11pt,oldfontcommands,oneside,onecolumn]{memoir}
\usepackage[utf8]{inputenc}\usepackage[T1]{fontenc}


\title{Food for thought}\author{Julian Wergieluk}\date{}
%%% A generic preamble. 

\usepackage[ngerman]{babel} 


\usepackage[final]{graphicx} % options = [final]  = all graphics displayed, regardless of draft option in class
                               % options = [pdftex] = necessary (?) if import PDF files
                               % no option : when importing ps- and eps-files (?)
% \graphicspath{{../images/}}  % tell LaTeX where to look for images
% \DeclareGraphicsExtensions{.pdf, .PDF, .jpg, .JPG, .jpeg, .JPEG, .png, .PNG, .bmp, .BMP, .eps, .ps}
\usepackage{float}                      % Improved interface for floating objects ; add [H] option

\usepackage[usenames,dvipsnames,svgnames,table]{xcolor}
%\usepackage{listings}\lstset{basicstyle=\ttfamily, breaklines=true}
%\makeindex

% memoir options
%\chapterstyle{companion}

% ams packages and symbols
\usepackage{amsmath}      % loads amstext, amsbsy, amsopn but not amssymb
                            % equation stuff (eqref, subequations, equation, align, gather, flalign, multline, alignat, split...)
\usepackage{amsfonts}     % may be redundant with amsmath
% \usepackage{amssymb}      % may be redundant with amsmath
\numberwithin{equation}{chapter}  % reset equation counters at start of each "section" and prefix numbers by section number
\numberwithin{figure}{chapter}    % reset figure   counters at start of each "section" and prefix numbers by section number

\usepackage{latexsym}
%\usepackage{MnSymbol}      % WARNING: This package changes common symbols such as \int

% writing helpers
\usepackage[color]{showkeys}
\usepackage[obeyDraft,backgroundcolor=white,linecolor=black,textsize=tiny,textwidth=2cm]{todonotes}
\setlength{\marginparwidth}{2cm}

% layout
\usepackage[final, babel]{microtype} % many good lay-out/justification effects, see:
                                     % texblog.net/latex-archive/layout/pdflatex-microtype/

%\usepackage{mathtools}%                  http://www.ctan.org/pkg/mathtools
%\usepackage[tableposition=top]{caption}% http://www.ctan.org/pkg/caption
%\usepackage{booktabs,dcolumn}%           http://www.ctan.org/pkg/dcolumn + http://www.ctan.org/pkg/booktabs

\usepackage{amsthm, thmtools}

\theoremstyle{plain}
\newtheorem{theorem}{Theorem}[chapter]
\newtheorem{acknowledgement}[theorem]{Acknowledgement}
\newtheorem{algorithm}[theorem]{Algorithm}
\newtheorem{axiom}[theorem]{Axiom}
\newtheorem{case}[theorem]{Case}
\newtheorem{claim}[theorem]{Claim}
\newtheorem{conclusion}[theorem]{Conclusion}
\newtheorem{conjecture}[theorem]{Conjecture}
\newtheorem{corollary}[theorem]{Corollary}
\newtheorem{criterion}[theorem]{Criterion}
\newtheorem{exercise}[theorem]{Exercise}
\newtheorem{lemma}[theorem]{Lemma}
\newtheorem{notation}[theorem]{Notation}
\newtheorem{problem}[theorem]{Problem}
\newtheorem{proposition}[theorem]{Proposition}
\newtheorem{solution}[theorem]{Solution}
\newtheorem{definition}[theorem]{Definition}
\newtheorem{condition}[theorem]{Condition}
\newtheorem{assumption}[theorem]{Assumption}

\theoremstyle{remark}
\newtheorem{remark}[theorem]{Remark}
\newtheorem{example}[theorem]{Example}
\newtheorem{summary}[theorem]{Summary}

\newtheorem{questions}[theorem]{Questions}

%\declaretheorem{style=remark,qed=$\blacksquare$,name=Example}{example}


%%% FONTS

%\usepackage{concmath}
\usepackage{fourier}
\usepackage{newcent}


% HYPERREF (last) then HYPCAP %%%%%%%%%%%%%%%%%%%%%%%%%%%%%%%%%%%%%%%%%%%%%%%%%

%
% See:
% http://tex.stackexchange.com/questions/1863/which-packages-should-be-loaded-after-hyperref-instead-of-before
%
\usepackage[
pdftex, 
final,                      % if you do    want to have clickable-colorful links
pdfstartview = FitV,
linktocpage  = false,       % ToC, LoF, LoT place hyperlink on page number, rather than entry text
breaklinks   = true,        % so long urls are correctly broken across lines
% pagebackref  = false,     % add page number in bibliography and link to position in document where cited
]{hyperref}
\hypersetup{colorlinks=true, linkcolor=black, bookmarksopenlevel={2}, bookmarksopen = true}

% \usepackage{cleveref} % enhance cross-referencing, allow full formatting, commands:
                        % \cref, \Cref, \crefrange, \cref{eq2,eq1,eq3,eq5,thm2,def3}
                        % supposedly better than \autoref as provided by hyperref

% \usepackage[all]{hypcap} % when link to float (using hyperref), link anchors to beginning (instead of below) float




\newcommand{\E}{\mathbf{E}}  % Expectation
\newcommand{\Prob}{P}    % Expectation
\newcommand{\sAlg}[1]{\mathfrak{#1}}
\newcommand{\ExpDist}{\mathbf{Ex}}
\newcommand{\NormDist}{\mathbf{N}}
\newcommand{\R}{\mathbb R}
\newcommand{\N}{\mathbb N}

% Logic
\newcommand{\impl}{\Rightarrow}
\DeclareMathOperator{\cov}{cov}
\DeclareMathOperator{\corr}{corr}
\DeclareMathOperator{\var}{var}\DeclareMathOperator{\Var}{Var}

\newcommand{\vb}{\;\middle\vert\;}

\newcommand{\bA}{\mathbb A}\newcommand{\cA}{\mathcal A}\newcommand{\fA}{\mathfrak A}\newcommand{\bfA}{\mathbf A}
\newcommand{\bB}{\mathbb B}\newcommand{\cB}{\mathcal B}\newcommand{\fB}{\mathfrak B}\newcommand{\bfB}{\mathbf B}
\newcommand{\bC}{\mathbb C}\newcommand{\cC}{\mathcal C}\newcommand{\fC}{\mathfrak C}\newcommand{\bfC}{\mathbf C}
\newcommand{\bD}{\mathbb D}\newcommand{\cD}{\mathcal D}\newcommand{\fD}{\mathfrak D}\newcommand{\bfD}{\mathbf D}
\newcommand{\bE}{\mathbb E}\newcommand{\cE}{\mathcal E}\newcommand{\fE}{\mathfrak E}\newcommand{\bfE}{\mathbf E}
\newcommand{\bF}{\mathbb F}\newcommand{\cF}{\mathcal F}\newcommand{\fF}{\mathfrak F}\newcommand{\bfF}{\mathbf F}
\newcommand{\bG}{\mathbb G}\newcommand{\cG}{\mathcal G}\newcommand{\fG}{\mathfrak G}\newcommand{\bfG}{\mathbf G}
\newcommand{\bH}{\mathbb H}\newcommand{\cH}{\mathcal H}\newcommand{\fH}{\mathfrak H}\newcommand{\bfH}{\mathbf H}
\newcommand{\bI}{\mathbb I}\newcommand{\cI}{\mathcal I}\newcommand{\fI}{\mathfrak I}\newcommand{\bfI}{\mathbf I}
\newcommand{\bJ}{\mathbb J}\newcommand{\cJ}{\mathcal J}\newcommand{\fJ}{\mathfrak J}\newcommand{\bfJ}{\mathbf J}
\newcommand{\bK}{\mathbb K}\newcommand{\cK}{\mathcal K}\newcommand{\fK}{\mathfrak K}\newcommand{\bfK}{\mathbf K}
\newcommand{\bL}{\mathbb L}\newcommand{\cL}{\mathcal L}\newcommand{\fL}{\mathfrak L}\newcommand{\bfL}{\mathbf L}
\newcommand{\bM}{\mathbb M}\newcommand{\cM}{\mathcal M}\newcommand{\fM}{\mathfrak M}\newcommand{\bfM}{\mathbf M}
\newcommand{\bN}{\mathbb N}\newcommand{\cN}{\mathcal N}\newcommand{\fN}{\mathfrak N}\newcommand{\bfN}{\mathbf N}
\newcommand{\bO}{\mathbb O}\newcommand{\cO}{\mathcal O}\newcommand{\fO}{\mathfrak O}\newcommand{\bfO}{\mathbf O}
\newcommand{\bP}{\mathbb P}\newcommand{\cP}{\mathcal P}\newcommand{\fP}{\mathfrak P}\newcommand{\bfP}{\mathbf P}
\newcommand{\bQ}{\mathbb Q}\newcommand{\cQ}{\mathcal Q}\newcommand{\fQ}{\mathfrak Q}\newcommand{\bfQ}{\mathbf Q}
\newcommand{\bR}{\mathbb R}\newcommand{\cR}{\mathcal R}\newcommand{\fR}{\mathfrak R}\newcommand{\bfR}{\mathbf R}
\newcommand{\bS}{\mathbb S}\newcommand{\cS}{\mathcal S}\newcommand{\fS}{\mathfrak S}\newcommand{\bfS}{\mathbf S}
\newcommand{\bT}{\mathbb T}\newcommand{\cT}{\mathcal T}\newcommand{\fT}{\mathfrak T}\newcommand{\bfT}{\mathbf T}
\newcommand{\bU}{\mathbb U}\newcommand{\cU}{\mathcal U}\newcommand{\fU}{\mathfrak U}\newcommand{\bfU}{\mathbf U}
\newcommand{\bW}{\mathbb W}\newcommand{\cW}{\mathcal W}\newcommand{\fW}{\mathfrak W}\newcommand{\bfW}{\mathbf W}
\newcommand{\bV}{\mathbb V}\newcommand{\cV}{\mathcal V}\newcommand{\fV}{\mathfrak V}\newcommand{\bfV}{\mathbf V}
\newcommand{\bX}{\mathbb X}\newcommand{\cX}{\mathcal X}\newcommand{\fX}{\mathfrak X}\newcommand{\bfX}{\mathbf X}
\newcommand{\bY}{\mathbb Y}\newcommand{\cY}{\mathcal Y}\newcommand{\fY}{\mathfrak Y}\newcommand{\bfY}{\mathbf Y}
\newcommand{\bZ}{\mathbb Z}\newcommand{\cZ}{\mathcal Z}\newcommand{\fZ}{\mathfrak Z}\newcommand{\bfZ}{\mathbf Z}



\setcounter{secnumdepth}{4}\renewcommand{\theparagraph}{\arabic{paragraph}.}
\renewcommand{\theenumi}{\Alph{enumi}.}
\renewcommand{\labelenumi}{\theenumi}

\usepackage{paralist}
%\setlength{\parindent}{0pt}

\newcommand{\showsolutions}



\begin{document}
\chapterstyle{bringhurst}
\frontmatter
\maketitle
\newpage\tableofcontents

\mainmatter




\chapter{Stochastic calculus}

\paragraph{Transformations of the Brownian Motion. } Let $B,W$ be independent Brownian motions.
Show that the processes
\begin{enumerate}
    \item $\left( -B_t \right)$,
    \item $\left( B_{T-t}-B_t \right)$,
    \item $\left( c B_{t/c^2} \right)$ with $c\in\R \setminus \left\{ 0 \right\}$,
    \item $X= \left( tB_{1/t} \right)$ with $X_0=0$, and
    \item $\left( cB_i + \sqrt{1-c^2}W_t \right)$ for $0<c<1$
\end{enumerate}
are also Brownian motions. Specify the relevant filtrations.


\paragraph{Maximum processes of the Brownian Motion. } 
Show that the process $W - W^*$ has independent increments. $W$ denotes
Brownian Motion and $X^*_t= \max_{0 \leq s \leq t} X_s$ is the maximum process of 
a \cadlag process $X$.


\paragraph{Transformations of matringales.} (a) If $M$ is a martingale and
$\varphi$ is a convex function such that each $\varphi(M_t)$ is integrable,
show that $\varphi(M)$ is a submartingale.  (b) If in (a) $M$ is a
submartingale, and $\varphi$ is also non-decreasing on the range of $M$, show
that again $\varphi(M)$ is a submartingale.


\paragraph{A basic submartingale transformation.}
Let $M$ be a submartingale and $\varphi$ a non-decreasing function
on the range of $M$. If $\varphi M_t$ is integrable for each $t$ then 
$\left( \varphi M_t \right)_t$ is again a submartingale.


\paragraph{Basic properties of local martingales. } A bounded local martingale is a 
true martingale. A local martingale with bounded jumps is a honest martingale.





\paragraph{Ornstein-Uhlenbeck process. } Consider a Ornstein-Uhlenbeck process
\begin{equation}
	dX_t = -\lambda \left( X_t -\theta \right)dt + \sigma dW_t, \quad X_0\in\R.
\end{equation}
Prove that
\begin{eqnarray}
	EX_t &=& \theta + e^{-\lambda t }\left( X_0 - \theta \right) \\
	\var X_t &=&  \frac{\sigma^2}{ 2 \lambda} \left( 1 - e^{-2\lambda t } \right).
\end{eqnarray}
Martin proved that using a basic properties of affine processes.



\paragraph{Feller processes. Not Feller by discontinuity at zero. }
Find a Markov process violating the continuity part of 
the Feller property, namely
\begin{equation}
\lim_{t\to 0} P_t f(x) = f(x) \quad \forall f \in C_0 \ \forall x\in \R
\end{equation}
but perhaps preserving the $C_0$ property $P_t C_0 \subset C_0$.




\paragraph{L\'evy martingales.  } Every L\'evy process which is a local
martingale is also a true martingale.












\chapter{Probability theory}

\paragraph{Independence and transformations. } Let random variables $X_1,\ldots,X_n$ and
$Y_1,\ldots,Y_k$ satisfy $X_i \upmodels Y_j$ for all $i$ and $j$.
\begin{enumerate}
    \item It follows that $g(X_i)$ and $h(Y_j)$ are independent for any $i$ and $j$.
    \item The following generalization of the above result is not true. Given
any Borel functions $f$ and $g$ of suitable dimensionality 
$f(X_1,\ldots,X_n)$ may be not independent of $g(Y_1,\ldots,Y_k)$.
\end{enumerate}

\paragraph{Solution. } $\sigma\left( g(X_i) \right) \subset \sigma(X)$ and therefore 
$A\in\sigma(g(X_i))$ and $B\in\sigma(h(Y_i))$ implies $P(A\cap B)=P(A)P(B)$.



\paragraph{Monotonic transformations of RVs. }   A monotonic transformation of a real
valued random variable with a continuous distribution function has a continuous
distribution function. 
%\url{http://math.stackexchange.com/questions/78101/monotonic-transformation-of-continuous-random-variable-are-continuous}


















\chapter{Statistics}


\section{Exponential families}

\paragraph{Non-central exponential distribution is not an exponential family. } % statistics
Show that non-central exponential distribution with the density
\begin{equation}
    p_\theta(x) = \frac{1}{\theta} \exp\left( 1- \frac{x}{\theta} \right)1_{\R_{>\theta}}(x)
\end{equation}
and $\theta\in\R$ is not an exponential family. 

\paragraph{Solution. } Assume 
\begin{equation}
    p_\theta(x)=h(x)c(\theta)\exp(w(\theta) t(x)).
\end{equation}
For any fixed $\theta$ and $x<\theta$ we have
$$p_\theta(x)=h(x)c(\theta)\exp(w(\theta) t(x))=0,$$ but on the other hand
$p_\theta(x)>0$ for all $x>\theta$. This implies $c(\theta)>0$ and $h(x)=0$ for
all $x<\theta$. Since $\theta$ was arbitrary, we get $h(x)\equiv 0$, a
contradiction.










\end{document}
