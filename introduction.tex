
Die vorliegende Sammlung enthält über $287$ Probleme und Aufgaben, die ich seit
2011 für meine Übungen an der TU Chemnitz verwendet habe. Die Aufgaben sind zum
Teil mit Lösungshinweisen oder mit vollständigen Lösungen versehen. 

\subsection*{Fehler und Korrekturen}

Bedenken Sie, dass einige Aufgaben und deren Lösungen unter Umständen
unvolständig, falsh oder \emph{schlecht \textbf{formatiert}} sind. 

Wenn Sie einen Fehler gefunden haben, erstellen Sie bitte eine ``Issue'' in dem
Fehlerverfolgunssystem des Projekts und verweisen Sie in dem Namen des Fehlers
auf den (innerhalb der Sammlung eindeutigen) Titel der Aufgabe. 

\subsection*{Quellen}

Bei der Erstellung dieser Aufgabensammlung habe ich unter anderen folgende
excellente Bücher verwendet:

\begin{itemize}
    \item \fullcite{czadoschmidt}
    \item \fullcite{JacodProtter2004}
    \item \fullcite{jakubowski2001}
    \item \fullcite{shao2003mathematical}
    \item \fullcite{tuc-wth-aufgaben}
\end{itemize}

\subsection*{Kompilierung der Aufgabensammlung} 

Um die Datei \texttt{problem-supermarkt.pdf} aus den \LaTeX{}-Quellen zu erstellen
wird eine aktuelle \TeX{}-Distribution benötigt, wie zum Beispiel \TeX{}Live 2012. 

Basierend auf den Aufgaben in dieser Sammlung können auch automatisiert 
Übungsblätter erstellt werden. Dies ist mit Hilfe eines separaten Werkzeugs 
\href{https://github.com/jwergieluk/problem-extractor}{problem-extractor}
möglich. 

\subsection*{Mitwirkende}

Ich möchte mich bei folgenden Personen für das Beitragen der Aufgaben, Lösungen und
Verbesserungsvorschläge herzlich bedanken: 

Thorsten Schmidt, Bernd Hofmann, Holger Langenau, Michael Pippig, Toni Volkmer,
Susanne Lindner.

\subsection*{Weiterverwendung der Probleme aus dieser Sammlung}

Problemsupermarkt ist lizenziert unter einer
\href{http://creativecommons.org/licenses/by-sa/4.0/}{Creative Commons
Namensnennung - Weitergabe unter gleichen Bedingungen 4.0 International
Lizenz}.

\begin{flushright}
Julian Wergieluk, Chemnitz, 2014.
\end{flushright}


