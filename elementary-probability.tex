

\section{Elementary probability}



\paragraph{Ereignisse. Mengenoperationen.}
Gegeben seien jeweils die Ergebnismenge $\Omega$ sowie zwei Teilmengen
$A$ und $B$:
\begin{enumerate}
\item $\Omega=\{1,2,...,20\}$, $A=\{4,5,6,7,9,11\}$, $B=\{3,5,9,20\}$
\item $\Omega=[-1,3]$, $A=[0,1)$, $B=(\frac{1}{2},2]$
\item $\Omega=\mathbb R$, $A=\{x\in \mathbb R: |x-1|<3\}$, $B=[0,\infty)$.
\end{enumerate}
Bilden Sie die Mengen $\overline{A}$, $\overline{B}$, $A\cap B$, $A\cup B$,
$\overline{ A\cup B}$, $\overline{A}\cap\overline{B}$, $B\cap\overline{A}$,
$(\overline{A}\cup\overline{B})\cap\overline{B}$,
$B\cup(\overline{B\cap\overline{A}})$.


\paragraph{Ereignisse. Kraftwerk.}
Die Arbeit eines Kraftwerkes wird durch drei unabhängig voneinander arbeitende
Kontrollsysteme überwacht, die jedoch auch einer gewissen Störanfälligkeit
unterliegen. Es bezeichne $S_i$ das Ereignis, dass das $i$-te System
störungsfrei arbeitet $(i=1,2,3)$.
\begin{enumerate}
    \item Finden Sie einen geeigneten Wahrscheinlichkeitsraum, der diese
        Zufallssituation beschreibt. Geben Sie die Ergebnismenge $\Omega$
        explicit an. Ist die Ergebnismenge eindeutig bestimmt?
    \item Drücken Sie folgende Ereignisse mit Hilfe der Ereignisse $S_1$, $S_2$
        und $S_3$ aus:
        \begin{itemize}
            \item[$A$:] Alle drei Systeme arbeiten störungsfrei.
            \item[$B$:] Kein System arbeitet störungsfrei.
            \item[$C$:] Mindestens ein System arbeitet störungsfrei.
            \item[$D$:] Genau ein System arbeitet störungsfrei.
            \item[$E$:] Höchstens zwei Systeme sind gestört.
        \end{itemize}\label{ereignisse-kraftwerk-1}
    \item Welche der unter \ref{ereignisse-kraftwerk-1} genannten Ereignisse
        sind Elementarereignisse?
    \item Aus wie vielen Elementen bestehen die Ereignisse $D$ und $C$?
\end{enumerate}

\paragraph*{Lösung.} Die Ergebnismenge $\Omega$ kann als $\Omega= \left\{ (ijk)
: i,j,k\in \left\{ 0,1 \right\} \right\}$ gewählt werden. Diese Wahl ist aber nicht 
eindeutig. Die Ereignisse $A$ und $B$ sind in diesem Fall Elementarereignisse. 


\paragraph{Ereignisse. Aktienmarkt.}
Beim Monatsvergleich zweier Technologieaktien wird für jede Aktie
festgestellt, ob es zu einem Gewinn von mindestens 3\% kam, ob sich ein Verlust
um mehr als 3\% ergab oder ob sich die jeweilige Aktie innerhalb der 6\%-Spanne
bewegte.
\begin{enumerate}
    \item Finden Sie einen geeigneten Wahrscheinlichkeitsraum, der diese
        Zufallssituation beschreibt. Geben Sie die Ergebnismenge $\Omega$
        explicit an.
    \item Stellen Sie folgende Ereignisse mit Hilfe der Elementarereignisse
        dar:
        \begin{itemize}
            \item[$A$:] Beide Aktien erzielten einen Kursgewinn von mindestens $3\%$.
            \item[$B$:] Die Kurse der beiden Aktien lagen innerhalb der
                $6\%$-Spanne. %Keine der beiden Aktien veränderte sich signifikant.
            \item[$C$:] Der Kurs von höchstens einer der beiden Aktien
                verschlechterte sich um mehr als $3\%$. 
            \item[$D$:] Der Kurs von mindestens einer der beiden Aktien
                verschlechterte sich um mehr als $3\%$. 
        \end{itemize}
    \item Welche Bedeutung haben die Ereignisse\\
        $E_1=A\cup C$, $E_2=A\cup D$,
        $E_3=A\cap C$, $E_4=A\cap\overline{C}$, $E_5= \overline{A\cap D}$ ?
\end{enumerate}



\paragraph{Ereignisse. Fertigungsstraße.}
Eine Fertigungsstraße bestehe  aus einer Maschine vom Typ I, vier Maschinen vom
Typ II und zwei Maschinen vom Typ III. Wir bezeichnen mit $A$, $B_k$ bzw.\
$C_j$ ($k=1,2,3,4$; $j=1,2$) die Ereignisse, dass die Maschine vom Typ I
bzw.~die $k$-te Maschine vom Typ~II bzw.~die $j$-te Maschine vom Typ III intakt
ist. Die Fertigungsstraße sei arbeitsfähig, wenn mindestens eine Maschine von
jedem Maschinentyp intakt ist. Dieses Ereignis werde mit $D$ bezeichnet.

Beschreiben Sie die Ereignisse $D$ und $\overline{D}$ mit Hilfe der Ereignisse
$A$, $B_k$, $C_j$.


\paragraph{Ereignisse. Kosten.}
Drei Betriebsteile werden auf Einhaltung eines bestimmten
Kostenfaktors überprüft. Das Ereignis $A$ liege vor, wenn mindestens ein
Betriebsteil nicht den geforderten Kostenfaktor einhält, das Ereignis $B$ liege
vor, wenn alle drei Betriebsteile den geforderten Kostenfaktor einhalten.

Was bedeuten dann die Ereignisse $A\cup B$ und $A\cap B$ ?


\paragraph{Ereignisse. Würfel und Münze.}
Ein Experiment bestehe aus dem Werfen eines fairen Würfels und einer fairen Münze.
\begin{enumerate}
    \item Geben Sie eine geeignete Ergebnismenge $\Omega$ an.
    \item Zeigt die Münze Wappen, so wird die doppelte Augenzahl des Würfels
        notiert, bei Zahl nur die einfache. Wie groß ist die
        Wahrscheinlichkeit, dass eine gerade Zahl notiert wird?
\end{enumerate}


\paragraph{Ereignisse. Zerlegung des Würfels.}
Ein Würfel, dessen Seitenflächen gleichartig gefärbt sind, werde in 1000
kleine Würfel einheitlicher Größe zerlegt.

Wie groß ist die Wahrscheinlichkeit dafür, dass ein zufällig ausgewählter
Würfel auf mindestens einer Seite gefärbt ist?


\paragraph{Ereignisse. Elementare Wahrscheinlichkeiten.}
Für die Ereignisse $A$ und $B$ seien folgende Wahrscheinlichkeiten bekannt:
$P(A)=0.25$, $P(B)=0.45$, $P(A\cup B)=0.5$. Berechnen Sie die
Wahrscheinlichkeiten:
\begin{enumerate}
    \item $P(A\cap\overline{B})$,
    \item $P(\overline{A}\cap\overline{B})$ und
    \item $P\left((A\cap\overline{B})\cup(\overline{A}\cap B)\right)$.
\end{enumerate}



\section{Wahrscheinlichkeitsmaße.}



\paragraph{Wahrscheinlichkeitsmaße. Monotonie.} Sei ein Wahrscheinlichkeitsraum
$(\Omega, \fA, P)$ und die Ereignisse $A,B \in \fA$ gegeben. Zeigen Sie 
\begin{equation*}
    A \subset B  \impl P(A) \leq P(B). 
\end{equation*}


\paragraph{Wahrscheinlichkeitsmaße. Subaditivität.} Sei ein
Wahrscheinlichkeitsraum $(\Omega, \fA, P)$ und die Ereignisse $A_1, A_2, \cdots
\in \fA$ gegeben. 
\begin{enumerate}
    \item Zeigen Sie
        \begin{equation*}
            P \left(  \cup_{i=1}^n A_i \right) \leq \sum_{i=1}^{n} P\left( A_i \right)
        \end{equation*}
        für alle $n\in \bN$. 

    \item Zeigen Sie 
        \begin{equation*}
            P \left(  \cup_{i=1}^\infty A_i \right) \leq \sum_{i=1}^{\infty} P\left( A_i \right).
        \end{equation*}
\end{enumerate}


\paragraph{Wahrscheinlichkeitsmaße. Bonferroni Formel.} Sei ein
Wahrscheinlichkeitsraum $\left( \Omega, \fA, P \right)$ und die Ereignisse
$A,B,C\in \fA$ sowie $A_1, A_2, \cdots, A_n \in \fA$ gegeben.
Zeigen Sie folgende Aussagen:
\begin{enumerate}
    \item \begin{equation*}
            P \left( A \cup B \cup C \right) = 
            P(A)+P(B)+P(C) - P(A\cap B) - P(B\cap C) + P(A\cap B \cap C). 
        \end{equation*}

    \item
        \begin{align*}
            P\left( \cup_{i=1}^{n} A_i \right) =&
            \sum_{i=1}^{n} P(A_i) - \sum_{i<j} P(A_i \cap A_j) + \\
            & \sum_{i<j<k} P(A_i \cap A_j \cap A_k) - \cdots + 
            (-1)^{n+1} P\left( A_1 \cap \cdots \cap A_n \right). 
        \end{align*}
\end{enumerate}




\section{Kombinatorik.}


\paragraph{Kombinatorik. Passwörter.} Sei 
$\cA = \left\{ a, b, \cdots, z, 0, \cdots, 9 \right\}$ die Menge der Zeichen
gegeben. 
\begin{enumerate}
    \item Wie viele voneinander verschiedene Passwörter der Länge 8 können
        aus $\cA$ gebildet werden?
    \item Betrachten wir nun die Passwörter aus obiger Menge, die an der
        letzen Stelle eine Ziffer und sonst aus lauter Buchstaben bestehen. 
        Wie viele solche Passwörter gibt es?
    \item Vergleichen Sie die Größenordnungen der Mächtigkeiten der obigen
        Passwortmengen. 
\end{enumerate}


\paragraph{Kombinatorik. PINs und runs.} In einer Folge $\left( a_1, a_2, a_3,
\cdots \right)$ wird eine Teilfolge $(a, a, \cdots, a)$, die aus $n$-facher
Wiederholung eines Elements $a$ gebildet wird, als ein $n$-run bezeichnet.
Z.b.\ hat die Folge $(0, 0, 1, 1, 1, 0)$ acht runs, nämlich einen $2$-run
$(0,0)$ und einen $3$-run $(1,1,1)$, sowie sechs $1$-runs.

Betrachten wir nun die Menge der 5-stelligen PINs, die aus den Ziffern $\left\{
0,1, \cdots, 9 \right\}$ gebildet werden können. 
\begin{enumerate}
    \item Wie viele solche PINs gibt es?
    \item Wie viele PINs gibt es, die keine $2$-runs, $3$-runs, $4$-runs, sowie keine 
        $5$-runs enthalten.
    \item Vergleichen Sie die Größenordnungen der Mächtigkeiten der obigen
        PIN-Mengen. 
\end{enumerate}


\paragraph{Kombinatorik. Zeichenketten.}
\begin{enumerate}
\item Wie viele voneinander verschiedene dreistellige (ganze, positive)
Zahlen können mit Hilfe der Ziffern $1,2,3,4,5$ gebildet werden?
\item Welches Ergebnis ergibt sich bei Aufgabe (a), wenn jede Ziffer nur
höchstens einmal in der zu bildenden Zahl vorkommen darf?
\end{enumerate}


\paragraph{Kombinatorik. Tippscheine.}
Wie viel verschiedene Tippscheine gibt es
\begin{enumerate}
\item beim Lotto (6 aus 49),
\item bei der Glücksspirale (7 stellige Zahl),
\item bei der 11er Wette (Fußballtoto)?
\end{enumerate}


\paragraph{Kombinatorik. Positionen.}
Aus einer Gruppe von drei Männern und vier Frauen sind drei Positionen in verschiedenen Kommissionen zu besetzen. Wie groß ist die Wahrscheinlichkeit für die Ereignisse, dass mindestens eine der drei Positionen mit einer Frau besetzt wird bzw. dass höchstens eine der drei Positionen mit einer Frau besetzt wird,
\begin{enumerate}
\item falls jede Person nur eine Position erhalten kann?
\item falls jede Person mehrere Positionen erhalten kann?
\end{enumerate}


\paragraph{Kombinatorik. Zwei Würfel.}
Wie groß ist die Wahrscheinlichkeit dafür, beim Werfen von zwei Würfeln eine
Augensumme zu erzielen, die größer oder gleich 10 ist?


\paragraph{Kombinatorik. Wanderwege.}
In der Umgebung eines Urlaubsortes sollen 15 Wanderwege durch je zwei farbige,
parallele Striche gekennzeichnet werden. Wie viel verschiedene Farben werden
mindestens benötigt, wenn gleichfarbige Paare auftreten dürfen.\\
Geben Sie zwei verschiedene Lösungen an, und diskutieren Sie diese
Ergebnisse.


\paragraph{Kombinatorik. Roulette.}
Ein Spieler setzt an einem Abend beim Roulette nur auf die Farbe rot. Wie viele
Spiele müssen mindestens absolviert werden, damit die Wahrscheinlichkeit für
mindestens einen Gewinn größer ist als 0.9\,?


\paragraph{Kombinatorik. Würfel. 4 vs 24.}
Was ist wahrscheinlicher:
\begin{enumerate}
    \item Beim Werfen von vier Würfeln auf wenigstens einem eine Sechs zu
        erzielen, oder
    \item bei 24 Würfen von zwei Würfeln wenigstens einmal zwei Sechsen zu
        erhalten?
\end{enumerate}


\paragraph{Kombinatorik. Gütekontrolleur.}
Ein Gütekontrolleur entnimmt einem aus $N$ Teilen bestehenden Prüflos
nacheinander ohne Zurücklegen $n$ Teile.
\begin{enumerate}
\item Wie groß ist die Wahrscheinlichkeit dafür, dass sich unter diesen $n$
Teilen genau $m$ Ausschussteile befinden, falls das Prüflos $M$ Ausschussteile
enthält?
\item Berechnen Sie für $N=100$, $M=4$ und $n=10$ die Wahrscheinlichkeit
dafür, dass sich unter den zehn ausgewählten Teilen mindestens ein
Ausschussteil
befindet!
\end{enumerate}


\paragraph{Kombinatorik. Prüfungsfragen.}
Ein Student kann von 20 Prüfungsfragen 18 Fragen beantworten. Die Note 1 wird
genau dann erteilt, wenn von 4 zufällig gewählten Fragen alle beantwortet
werden können. \\
Berechnen Sie die Wahrscheinlichkeit dafür, dass der Student nicht die Note 1
erhält.


\paragraph{Kombinatorik. Single choice test.} Bei einem single choice test werden
den Prüfenden $n$ Fragen gestellt, wobei jeweils genau eine richtige Antwort aus
$m$ Möglichkeiten gewählt werden soll. Der Test wird als bestanden angesehen, wenn mindestens 
Halfte der Fragen richtig beantwortet wurden.

Wir testen die Prüfmethode indem wir zufällig jeweils eine Antwort bei jeder Frage
ankreuzen. 
\begin{enumerate}
    \item Finden Sie einen geeigneten Wahrscheinlichkeitsraum, der diese
        Zufallssituation beschreibt.
    \item Wie hoch ist die Wahrscheinlichkeit, dass die zufällige Antwortwahl zum 
        Bestehen des Test führt?
    \item Berechnen Sie die obige Wahrscheinlichkeit explicit für $n=25$ und $m=4$.
\end{enumerate}





