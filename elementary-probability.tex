

\section{Elementary probability}



\paragraph{Ereignisse. Mengenoperationen.}
Gegeben seien jeweils die Ergebnismenge $\Omega$ sowie zwei Teilmengen
$A$ und $B$:
\begin{enumerate}
\item $\Omega=\{1,2,...,20\}$, $A=\{4,5,6,7,9,11\}$, $B=\{3,5,9,20\}$
\item $\Omega=[-1,3]$, $A=[0,1)$, $B=(\frac{1}{2},2]$
\item $\Omega=\mathbb R$, $A=\{x\in \mathbb R: |x-1|<3\}$, $B=[0,\infty)$.
\end{enumerate}
Bilden Sie die Mengen $\overline{A}$, $\overline{B}$, $A\cap B$, $A\cup B$,
$\overline{ A\cup B}$, $\overline{A}\cap\overline{B}$, $B\cap\overline{A}$,
$(\overline{A}\cup\overline{B})\cap\overline{B}$,
$B\cup(\overline{B\cap\overline{A}})$.


\paragraph{Ereignisse. Kraftwerk.}
Die Arbeit eines Kraftwerkes wird durch drei unabhängig voneinander arbeitende
Kontrollsysteme überwacht, die jedoch auch einer gewissen Störanfälligkeit
unterliegen. Es bezeichne $S_i$ das Ereignis, dass das $i$-te System
störungsfrei arbeitet $(i=1,2,3)$.
\begin{enumerate}
    \item Finden Sie einen geeigneten Wahrscheinlichkeitsraum, der diese
        Zufallssituation beschreibt. Geben Sie die Ergebnismenge $\Omega$
        explicit an. Ist die Ergebnismenge eindeutig bestimmt?
    \item Drücken Sie folgende Ereignisse mit Hilfe der Ereignisse $S_1$, $S_2$
        und $S_3$ aus:
        \begin{itemize}
            \item[$A$:] Alle drei Systeme arbeiten störungsfrei.
            \item[$B$:] Kein System arbeitet störungsfrei.
            \item[$C$:] Mindestens ein System arbeitet störungsfrei.
            \item[$D$:] Genau ein System arbeitet störungsfrei.
            \item[$E$:] Höchstens zwei Systeme sind gestört.
        \end{itemize}\label{ereignisse-kraftwerk-1}
    \item Welche der unter \ref{ereignisse-kraftwerk-1} genannten Ereignisse
        sind Elementarereignisse?
    \item Aus wie vielen Elementen bestehen die Ereignisse $D$ und $C$?
\end{enumerate}

\paragraph*{Lösung.} Die Ergebnismenge $\Omega$ kann als $\Omega= \left\{ (ijk)
: i,j,k\in \left\{ 0,1 \right\} \right\}$ gewählt werden. Diese Wahl ist aber nicht 
eindeutig. Die Ereignisse $A$ und $B$ sind in diesem Fall Elementarereignisse. 


\paragraph{Ereignisse. Aktienmarkt.}
Beim Monatsvergleich zweier Technologieaktien wird für jede Aktie
festgestellt, ob es zu einem Gewinn von mindestens 3\% kam, ob sich ein Verlust
um mehr als 3\% ergab oder ob sich die jeweilige Aktie innerhalb der 6\%-Spanne
bewegte.
\begin{enumerate}
    \item Finden Sie einen geeigneten Wahrscheinlichkeitsraum, der diese
        Zufallssituation beschreibt. Geben Sie die Ergebnismenge $\Omega$
        explicit an.
    \item Stellen Sie folgende Ereignisse mit Hilfe der Elementarereignisse
        dar:
        \begin{itemize}
            \item[$A$:] Beide Aktien erzielten einen Kursgewinn von mindestens $3\%$.
            \item[$B$:] Die Kurse der beiden Aktien lagen innerhalb der
                $6\%$-Spanne. %Keine der beiden Aktien veränderte sich signifikant.
            \item[$C$:] Der Kurs von höchstens einer der beiden Aktien
                verschlechterte sich um mehr als $3\%$. 
            \item[$D$:] Der Kurs von mindestens einer der beiden Aktien
                verschlechterte sich um mehr als $3\%$. 
        \end{itemize}
    \item Welche Bedeutung haben die Ereignisse\\
        $E_1=A\cup C$, $E_2=A\cup D$,
        $E_3=A\cap C$, $E_4=A\cap\overline{C}$, $E_5= \overline{A\cap D}$ ?
\end{enumerate}



\paragraph{Ereignisse. Fertigungsstraße.}
Eine Fertigungsstraße bestehe  aus einer Maschine vom Typ I, vier Maschinen vom
Typ II und zwei Maschinen vom Typ III. Wir bezeichnen mit $A$, $B_k$ bzw.\
$C_j$ ($k=1,2,3,4$; $j=1,2$) die Ereignisse, dass die Maschine vom Typ I
bzw.~die $k$-te Maschine vom Typ~II bzw.~die $j$-te Maschine vom Typ III intakt
ist. Die Fertigungsstraße sei arbeitsfähig, wenn mindestens eine Maschine von
jedem Maschinentyp intakt ist. Dieses Ereignis werde mit $D$ bezeichnet.

Beschreiben Sie die Ereignisse $D$ und $\overline{D}$ mit Hilfe der Ereignisse
$A$, $B_k$, $C_j$.


\paragraph{Ereignisse. Kosten.}
Drei Betriebsteile werden auf Einhaltung eines bestimmten
Kostenfaktors überprüft. Das Ereignis $A$ liege vor, wenn mindestens ein
Betriebsteil nicht den geforderten Kostenfaktor einhält, das Ereignis $B$ liege
vor, wenn alle drei Betriebsteile den geforderten Kostenfaktor einhalten.

Was bedeuten dann die Ereignisse $A\cup B$ und $A\cap B$ ?


\paragraph{Ereignisse. Würfel und Münze.}
Ein Experiment bestehe aus dem Werfen eines fairen Würfels und einer fairen Münze.
\begin{enumerate}
    \item Geben Sie eine geeignete Ergebnismenge $\Omega$ an.
    \item Zeigt die Münze Wappen, so wird die doppelte Augenzahl des Würfels
        notiert, bei Zahl nur die einfache. Wie groß ist die
        Wahrscheinlichkeit, dass eine gerade Zahl notiert wird?
\end{enumerate}


\paragraph{Ereignisse. Zerlegung des Würfels.}
Ein Würfel, dessen Seitenflächen gleichartig gefärbt sind, werde in 1000
kleine Würfel einheitlicher Größe zerlegt.

Wie groß ist die Wahrscheinlichkeit dafür, dass ein zufällig ausgewählter
Würfel auf mindestens einer Seite gefärbt ist?


\paragraph{Ereignisse. Elementare Wahrscheinlichkeiten.}
Für die Ereignisse $A$ und $B$ seien folgende Wahrscheinlichkeiten bekannt:
$P(A)=0.25$, $P(B)=0.45$, $P(A\cup B)=0.5$. Berechnen Sie die
Wahrscheinlichkeiten:
\begin{enumerate}
    \item $P(A\cap\overline{B})$,
    \item $P(\overline{A}\cap\overline{B})$ und
    \item $P\left((A\cap\overline{B})\cup(\overline{A}\cap B)\right)$.
\end{enumerate}

\paragraph*{Lösung.}
\begin{enumerate}
    \item \begin{align*}
            P(A \cup B) &= P( (A \cap \bar B) \cup B ) = 
            P( A \cap \bar B  ) + P(B) \\
            P(A \cup \bar B) &= P(A \cup B) - P(B) = 0.05.
        \end{align*}
    \item \begin{align*}
            P(\bar A \cap \bar B) &= P(\overline{A \cup B}) = 0.5.
        \end{align*}
    \item Die symmetrische Differenz $A \Delta B = (A \setminus B) \cup (B \setminus A)$
        ist eine disjunkte Vereinigung, und daher
        \begin{align*}
            P(A \Delta B) &= P(A \cap \bar B) + P(\bar A \cap B).
        \end{align*}
        Die vorherigen Überlegungen liefern
        \begin{align*}
            P( A \cap \bar B) &= P(A \setminus B) = P(A \cup B) - P(B) = 0.05 \\
            P( \bar A \cap B) &= P(B \setminus A) = P(A \cup B) - P(A) = 0.25.
        \end{align*}
        Insgesamt gilt also $P(A \Delta B) = 0.3$.
\end{enumerate}


\section{Wahrscheinlichkeitsmaße.}

\paragraph{Wahrscheinlichkeitsmaße. Monotonie.} Seien ein Wahrscheinlichkeitsraum \linebreak
$(\Omega, \fA, P)$ und die Ereignisse $A,B \in \fA$ gegeben. Zeigen Sie
\begin{equation*}
A \subseteq B \quad  \impl \quad P(A) \leq P(B).
\end{equation*}

\paragraph*{Lösung.} $P(B) = P(A \cup (B \setminus A))= P(A) + P(B\setminus A) \geq P(A)$. 


\paragraph{Wahrscheinlichkeitsmaße. Subaditivität.} Seien ein
Wahrscheinlichkeitsraum $(\Omega, \fA, P)$ und die Ereignisse $A_1, A_2, \cdots
\in \fA$ gegeben. 
\begin{enumerate}
    \item Zeigen Sie, dass für alle  $n\in \bN$ 
        \begin{equation*}
            P \left(  \bigcup \limits_{i=1}^n A_i \right) \leq \sum_{i=1}^{n} P\left( A_i \right)
        \end{equation*}
        gilt. 

    \item Zeigen Sie, dass sogar gilt 
        \begin{equation*}
            P \left(  \bigcup \limits_{i=1}^\infty A_i \right) \leq \sum_{i=1}^{\infty} P\left( A_i \right).
        \end{equation*}
\end{enumerate}

\paragraph*{Lösung.}
\begin{enumerate}
    \item Betrachte folgende Darstellung der Vereinigung als Summe disjunkter
        Elemente von $\fA$.
        \begin{align*}
            \bigcup_{i=1}^n A_i &= A_1 \cup (\bar A_1 \cap A_2) \cup 
            (\bar A_1 \cap \bar A_2 \cap A_3) \cup \cdots \cup
            (\bar A_1 \cap \bar A_2 \cap \cdots \cap \bar A_{n-1} \cap A_n).
        \end{align*}

    \item Genau der gleiche Argument funktioniert für $n=\infty$. Das ist 
        die Konsequenz der $\sigma$-Additivität der Wahrscheinlichkeitsfunktion $P$. 
\end{enumerate}


\paragraph{Wahrscheinlichkeitsmaße. Zerlegung der Vereinigung.} Seien ein
Wahrscheinlichkeitsraum $\left( \Omega, \fA, P \right)$ und die Ereignisse
$A,B,C\in \fA$ sowie $A_1, A_2, \cdots, A_n \in \fA$ gegeben.
Zeigen Sie folgende Aussagen:
\begin{enumerate}
    \item \begin{equation*}
            P \left( A \cup B \cup C \right) = 
            P(A)+P(B)+P(C) - P(A\cap B) - P(B\cap C) + P(A\cap B \cap C). 
        \end{equation*}

    \item
        \begin{align*}
            P\left( \bigcup \limits_{i=1}^{n} A_i \right) =&
            \sum_{i=1}^{n} P(A_i) - \sum_{i<j} P(A_i \cap A_j) + \\
            & \sum_{i<j<k} P(A_i \cap A_j \cap A_k) - \cdots + 
            (-1)^{n+1} P\left( A_1 \cap \cdots \cap A_n \right). 
        \end{align*}
\end{enumerate}

\paragraph*{Lösung.} 
\begin{enumerate}
    \item Hier genügt eine Überlegung mit Hilfe der Venn-Diagrame. 

    \item Die Menge $A_1 \cup \cdots \cup A_n$ ist Vereinigung der Mengen der
        Form $A_{i_1} \cap \cdots \cap A_{i_k} \cap \bar A_{i_{k+1}} \cap
        \cdots \cap \bar A_{i_{n}}$, $k\geq 1$, wobei $i_1,\cdots, i_n$ eine
        Umnumerierung von $1, \cdots, n$ ist. Nun kommt solche Menge in der
        ersten Summe als Teilmenge von $A_i$ genau $\binom{k}{1}
        $ vor. In der zweiten Summe wird die Menge $\binom{k}{2}$ mal abgezogen. 
        Nachdem $\sum_{i=0}^{k} (-1)^{i} \binom{k}{i} = 0$, erhalten wir 
        $\sum_{i=1}^{k} (-1)^{i+1} \binom{k}{i}=1$.
\end{enumerate}




\section{Kombinatorik.}


\paragraph{Kombinatorik. Passwörter.} Wir betrachten die Menge
$\cA = \left\{ a, b, \cdots, z, 0, \cdots, 9 \right\}$ als gegebenen Zeichensatz.
\begin{enumerate}
    \item Wie viele voneinander verschiedene Passwörter der Länge 8 können
        aus $\cA$ gebildet werden?
    \item Betrachten wir nun die Passwörter aus dem obigen Zeichensatz, die an der
        letzen Stelle eine Ziffer aufweisen und sonst aus lauter Buchstaben bestehen.
        Wie viele solche Passwörter gibt es?
    \item Vergleichen Sie die Größenordnungen der Mächtigkeiten der obigen
        Passwortmengen. 
\end{enumerate}

\paragraph*{Lösung.} 
\begin{enumerate}
    \item Die Menge $\cA$ enthält $36$ Elemente. Nachdem in einem Passwort die
        Reihenfolge der Zeichen wichtig ist und die Zeichen mehrmals vorkommen
        können, ist die Anzahl der Passwörter gegeben durch die Anzahl der
        Variationen mit Zurücklegen, also durch
        \begin{equation}
            36^{8} = 2.821.109.907.456 \approx 10^{12.45}.
        \end{equation}
    \item Es gibt $26^7$ Passwörter mit Länge $7$, die nur aus Buchstaben bestehen. 
        Wenn wir an jedem solchen Passwort eine Ziffer am Ende hinzufügen, 
        erhalten wir $10\cdot 26^7$ Möglichkeiten. 
        \begin{equation}
            10\cdot 26^7 = 80.318.101.760 \approx 10^{10.9}.
        \end{equation}
    \item Die Mächtigkeiten der obigen Mengen sind also fast zwei Größenordnungen 
        auseinander.
\end{enumerate}



\paragraph{Kombinatorik. PINs und runs.} In einer Folge $\left( a_1, a_2, a_3,
\cdots \right)$ wird eine Teilfolge $(a, a, \cdots, a)$, die aus $n$-facher
Wiederholung eines Elements $a$ gebildet wird, als ein $n$-run bezeichnet.
Zum Beispiel hat die Folge $(0, 0, 1, 1, 1, 0)$ acht runs, nämlich einen $2$-run
$(0,0)$ und einen $3$-run $(1,1,1)$, sowie sechs $1$-runs.

Betrachten wir nun die Menge der 4-stelligen PINs, die aus den Ziffern $\left\{
0,1, \cdots, 9 \right\}$ gebildet werden können. 
\begin{enumerate}
    \item Wie viele solche PINs gibt es?
    \item Wie viele PINs gibt es, die keine $2$-runs, $3$-runs, sowie keine
        $4$-runs enthalten?
    \item Vergleichen Sie die Größenordnungen der Mächtigkeiten der obigen
        PIN-Mengen. 
\end{enumerate}
\textbf{Zusatz:} Führen Sie die obige Berechnung für die $5$-stelligen PINs durch.

\paragraph*{Lösung.}
\begin{enumerate}
    \item Es gibt $10^4$ solche PINs.
    \item Um die Anzahl der PINs ohne runs zu erhalten, zählen wir alle
        möglichen PINs mit runs auf. Erste Spalte gibt die Form des PINs,
        zweite die Anzahl solcher PINs und dritte die Anzahl der symmetrischen
        Fälle.
\begin{lstlisting}
----    10          1
---*    10 9        2
--==    10 9        1
--**    10 9 8      2
*--*    10 9 9      1
\end{lstlisting}
        Das sind insgesamt $2530$ Fälle. Es gibt also $7470$ PINs ohne runs.
    \item Wenn wir PINs mit runs ausschliessen, reduziert sich unser pool
        möglicher PINs um ein Viertel. 
\end{enumerate}


\paragraph{Kombinatorik. Zwei Würfel.}
Wie groß ist die Wahrscheinlichkeit dafür, beim Werfen von zwei Würfeln eine
Augensumme zu erzielen, die größer oder gleich 10 ist?

\paragraph*{Lösung} Wir lösen die Aufgabe einmal unter der Annahme, dass die Würfel
unterschieden werden können und einmal ohne diese Annahme. 
\begin{enumerate}
    \item Wenn wir zwei wohlunterscheidbare Würfel werfen, ist die
        $36$-elementige Ergebnismenge gegeben durch $\Omega = \left\{ (i,j) :
        i,j \in \left\{ 1, \cdots, 6 \right\} \right\}$. Die günstigen Fälle
        sind 
        \begin{equation}
            (4,6), (5,5), (5,6), (6,4), (6,5), (6,5).
        \end{equation}
        Nachdem es sich hier um ein Laplace-Modell handelt, ist die
        Wahrscheinlichkeit, dass die Summe größer als $10$ ist, gleich
        $\frac{6}{36}= \frac{1}{6}$.

    \item In Falle der nichtunterscheidbaren Würfel ist das kein
        Laplace-Experiment mehr. Deswegen muss geeignetes
        Wahrscheinlichkeitsmaß $P$ betrachtet werden. 
\end{enumerate}


\paragraph{Kombinatorik. Würfel. 4 vs.~24.}
Was ist wahrscheinlicher:
\begin{enumerate}
    \item Beim Werfen von vier Würfeln auf wenigstens einem eine Sechs zu
        erzielen, oder
    \item bei 24 Würfen von zwei Würfeln wenigstens einmal zwei Sechsen zu
        erhalten?
\end{enumerate}

\paragraph*{Lösung.}
\begin{enumerate}
    \item Das Betrachen des komplementären Ereignisses liefert
        \begin{equation}
            1 - \frac{5^4}{6^4} = 1.
            \label{}
        \end{equation}
    \item Ein äquivalentes Model ist, $24$ mal einen $36$-Würfel werfen. Die
        Wahrscheinlichkeit wenigstens ein mal die Zahl $36$ zu erhalten,
        berechnen wir indem wir das komplementäre Ereignis betrachten:
        \begin{equation} 
            1 - \frac{35^{24}}{36^{24}}.  
        \end{equation}
    \item Was ist größer? $\frac{5^4}{6^4} = 0.4822530864$ und
        $\frac{35^{24}}{36^{24}} = 0.5085961239$.
\end{enumerate}




\paragraph{Kombinatorik. Single choice test.} Bei einem single choice test
werden $n$ Fragen gestellt, wobei jeweils genau eine richtige Antwort aus $m$
Möglichkeiten gewählt werden soll. Der Test wird als bestanden angesehen, wenn
mindestens Hälfte der Fragen richtig beantwortet wurden.

Wir testen die Prüfmethode indem wir zufällig jeweils eine Antwort bei jeder Frage
ankreuzen. 
\begin{enumerate}
    \item Finden Sie einen geeigneten Wahrscheinlichkeitsraum, der diese
        Zufallssituation \linebreak beschreibt.
    \item Wie hoch ist die Wahrscheinlichkeit, dass die zufällige Antwortwahl zum 
        Bestehen des Tests führt?
    \item Berechnen Sie die obige Wahrscheinlichkeit explizit für $n=25$ und $m=4$.
\end{enumerate}

\paragraph*{Lösung.} Laplace-Modell. Anzahl der Möglichen: $| \Omega| = m^n$. 
Anzahl der Günstigen:
\begin{equation*}
    \binom{n}{ \lceil\frac{n}{2}\rceil }(m-1)^{n - \lceil\frac{n}{2}\rceil } +
    \binom{n}{   \lceil\frac{n}{2}\rceil +1 }(m-1)^{n -   \lceil\frac{n}{2}\rceil-1   } + \cdots +
    \binom{n}{n} (m-1)^{n-n}. 
\end{equation*}
Für den Fall $n=25$ und $m=13$, ist $\lceil \frac{n}{2} \rceil = 13$ und die Anzahl 
der günstigen Fälle gleich 
\begin{equation}
    3794787166756 \approx 10^{12.5}.
\end{equation}
Nachdem Anzahl der möglichen Fälle ist 
\begin{equation}
    m^{n} = 4^{25} = 1125899906842624 \approx 10^{15.05},
\end{equation}
ist die Wahrscheinlichkeit für eine zufällig bestandene Prüfung ungefähr gleich
$0.0033$.


