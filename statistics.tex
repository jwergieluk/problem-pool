
\section{Identifizierbarkeit}

\paragraph{Identifizierbarkeit im linearen Modell. } 
Man nehme an, dass folgendes Modell gegeben sei: 
\begin{equation*}
    Y_i = \sum_{j=1}^{p} z_{ij} \beta_j + \varepsilon_i, \quad i=1,\ldots,n. 
\end{equation*}
Hierbei seien $z_{11},\ldots,z_{np}$ bekante Konstanten und $\varepsilon_1,\ldots,\varepsilon_n$
i.i.d.\ mit $\varepsilon_1 \sim \mathcal N\left( 0,1 \right)$.
\begin{enumerate}
    \item Zeigen Sie, dass $\left( \beta_1,\ldots,\beta_p \right)$ genau dann identifizierbar ist, 
        falls $\mathbf z_1,\ldots,\mathbf z_p$ linear unabhängig sind, wobei 
            $\mathbf z_j = \left( z_{1j},\ldots,z_{nj} \right)^\top$.
    \item Begründen Sie, warum $\left( \beta_1, \ldots,\beta_p \right)$ nicht identifizierbar ist,
        falls $n<p$.
\end{enumerate}


\paragraph{Qualitätskontrolle.} 
Es sei eine \textsc{lkw}-Ladung mit $N$ Fernsehgeräten gegeben, wovon $N
\theta$ defekt sind. Es werden $N$ Fernseher (ohne Zurücklegen) überprüft.  Man
definiere
\begin{equation*}
    X_i = \begin{cases}
        1 & \text{$i$-ter überprüfter Fernseher ist defekt} \\
        0 & \text{sonst}.
    \end{cases}
\end{equation*}
\begin{enumerate}
    \item Zeigen Sie ohne Verwendung des Faktorisierungstheorems, dass $\sum_{i=1}^{n} X_i$ 
        suffizient für $\theta$ ist. 
    \item Zeigen Sie mit Hilfe des Faktorisierungstheorems, dass $\sum_{i=1}^{n} X_i$ suffizient
        für $\theta$ ist.
\end{enumerate}

\section{Suffizienz}

\paragraph{Suffizienz bei inversen Gamma-Verteilung. }     Eine i.i.d.-Stichprobe $X_1, \ldots, X_n$
sei invers Gamma-verteilt mit der Dichte 
\begin{equation*}
    p_{\alpha, \beta}\left( x \right) = 
    \frac{\beta^\alpha}{ \Gamma\left( \alpha \right)} x^{-(\alpha+1)}\exp\left( -\frac{\beta}{x} \right)1_{\R>0}(x)
\end{equation*}
Finden Sie eine zweidimensionale suffiziente Statistik für $\alpha$ und $\beta$.


\paragraph{Suffizienz. Beispiele. } Seien $X_1,\ldots,X_n$ i.i.d.~mit jeweils folgender
Dichte. Finden Sie in allen drei Fällen eine reellwertige suffiziente Statistik für $\theta$: 
\begin{enumerate}
	\item $p_\theta (x) = \frac{1}{2\theta} \exp( \frac{-|x-\mu|}{\theta})$, wobei $\theta >0$ und $\mu$ bekannt sei.
	\item $p_\theta (x) = \frac{1}{2\theta} 1_{\left( -\theta ,\theta \right)}(x) $, wobei $\theta >0$.
	\item $p_\theta (x) = \frac{\beta^\alpha}{\Gamma(\alpha)} x^{- (\alpha+1)} \exp\left(  -\frac{\beta}{x} \right) 1_{\R_{>0}}(x)$, wobei $\theta = (\alpha, \beta)$ und $\alpha, \beta > 0$.
\end{enumerate}


\paragraph{Suffizienz. Nichtzentrale Exponentialverteilung. } Seien $X_1,\ldots,X_n$ i.i.d.,
\begin{equation*}
	p_\theta (x) = \frac{1}{\sigma} \exp \left(-\frac{x-\mu}{\sigma} \right) 1_{\R_{\geq\mu}}(x)
\end{equation*}
die Dichte von $X_1$ sowie $\theta = (\mu, \sigma)^\top$ und $\Theta = \R
\times \R_{>0}$. Zeigen Sie, dass $\min(X_1,\ldots,X_n)$ eine suffiziente
Statistik für $\mu$ ist, falls $\sigma$ bekannt ist. Finden Sie eine
eindimensionale, suffiziente Statistik für $\sigma$, falls $\mu$ bekannt ist
und eine zweidimensinale suffiziente Statistik für $\theta$ an.

\section{Exponentielle Familien}

\paragraph{Laplace-Verteilung.  } Für jedes $\theta\in\R$ sei eine Dichte
\begin{equation*} p_\theta(x) = \frac{1}{2} \exp\left( -| x-\theta| \right),
\quad x\in\R \end{equation*} gegeben. Sei $P_\theta$ das zur Dichte
$p_\theta$ gehörige Wahrscheinlichkeitsmaß.  Zeigen Sie, dass $\left\{
P_\theta : \theta\in\Theta \right\}$ keine exponentielle Familie ist.



\paragraph{Verteilung der natürlichen suffizienten Statistik. }
Sei $X$ eine Zufallsvariable mit den Werten in $\left( \R^n, \mathcal B(\R^n)\right)$ und der Dichte
\begin{equation*}
	p(x,\theta) = \exp\left( c(\theta)\cdot T(x) + d(\theta) + S(x) \right) 1_{A}(x),
\end{equation*}
die einer einparametrigen exponentiellen Familie $\left\{ P_\theta :
\theta\in\Theta \right\}$ mit stetigem $T$ angehört. Zeigen Sie, dass die Dichte der natürlichen
suffizienten Statistik $T(X)$ die Form
\begin{equation*}
	q(t, \theta) = \exp\left( c(\theta)\cdot t + d(\theta) + S^*(t) \right) 1_{A^*}(t)
\end{equation*}
mit $A^* = \left\{ T(x) : x\in A \right\}$ hat.


\paragraph{Cauchy-Verteilung ist keine exponentielle Familie. }  Sei eine
Familie von Cauchy-Verteilungen $\left\{ P_\theta : \theta\in\R \right\}$ mit 
den zugehörigen Dichten
\begin{equation}
    p_\theta(x) = \frac{1}{\pi\left[ 1 + \left( x - \theta \right)^2 \right]}
\end{equation}
gegeben. Zeigen Sie, dass die Familie $\left\{ P_\theta : \theta\in\R \right\}$ keine 
exponentielle Familie ist.

\paragraph*{Lösungsskizze.} Nehmen wir an Cauchy-Vereilung ist eine expoentielle Familie. Dann 
gibt es eine Darstellung 
\begin{equation}
    p_\theta(x) = \frac{1}{\pi\left[ 1 + \left( x - \theta \right)^2 \right]} =
    \exp\left( c(\theta) T(x) + d(\theta) \right)h(x).
\end{equation}
TODO: Finish it! 




\section{Momentenmethode}

\paragraph{Momentenschätzer. Beispiele.} Bestimmen Sie mittels der Momentenmethode
einen Momentenschätzer für $\theta$ bei den folgenden Verteilungen: 
\begin{enumerate}
	\item Die Gleichverteilung mit Dichte
		\begin{equation*}
			p_\theta(x) = \frac{1}{ 2\theta } 1_{(-\theta,\theta)}(x), \quad \theta>0
		\end{equation*}
		Der Schätzer ist $\hat \theta = \left( \bar X \right)^{-1}$.
	\item Die Gamma-Verteilung mit der Dichte 
		\begin{equation*}
			p_\theta(x) = \frac{\beta^\alpha}{ \Gamma(\beta)} x^{\alpha-1} e^{-\beta x}1_{\R>0}(x)
		\end{equation*}
\end{enumerate}

\paragraph{Momentenschätzer für Beta-Verteilung. } Die Zufallsvariablen
$X_1,\ldots,X_n$ seien i.i.d.\ Beta-Verteilt, d.h.\ $X_1 \sim \textrm{Beta}(a,b)$.
Bestimmen Sie einen Momentenschätzer für $\theta=(a,b)^\top$.

\paragraph{Momentenschätzer für Weibull-Verteilung. } Seien 
$X_1,\ldots,X_n$ i.i.d.\ mit Dichte 
\begin{equation*}
	p(x) = \sqrt{ \frac{2 \theta^3}{\pi} } x^2 \exp\left( - \frac{\theta}{2}x^2 \right) 1_{\R>0}(x)
\end{equation*}
wobei der Parameter $\theta>0$ unbekannt ist. Berechnen Sie den Momentenschätzer für $\theta$
basierend auf dem zweiten Moment. 

\paragraph{Momentenschätzung vs Suffizienz. }   Betrachten 
Sie die Verteilungsfamilie von zweiseitigen Exponentialverteilungen
gegeben durch die Dichte
\begin{equation*}
	p_\theta(x) = \frac{1}{2} \exp\left( -| x- \theta | \right), \quad \theta\in\R
\end{equation*}
Zeigen Sie mit dem ersten Moment, dass $\bar X$ ein Momentenschätzer für 
$\theta$ ist. Weisen Sie nach, dass dieser nicht suffizient für $\theta$ ist.

\paragraph{AR1. }  Die Zufallsvariablen $Z_1,\ldots, Z_n$ seien i.i.d. 
mit $Z_1 \sim \mathcal N(0,\sigma^2)$. Die Zeitreihe $\left( X_i \right)_{1\leq i\leq n}$
heißt \emph{autoregressiv der Ordnung 1} oder \textsc{ar(1)}, falls mit $X_0:= \mu$
und für $1\leq i \leq n$ 
\begin{equation*}
	X_i = \mu + \beta(X_{i-1} - \mu) + Z_i.
\end{equation*}
\begin{enumerate}
	\item Verwenden Sie $\E X_i$, um einen Momentenschätzer für $\mu$ zu finden. 
	\item Nun seien $\mu= \mu_0$ und $\beta=\beta_0$ fix und bekannt und weiterhin
		\begin{equation*}
			U_i = \frac{X_i - \mu_0}{ \sqrt{ \sum_{j=0}^{i-1} \beta^{2j}_0 }}.
		\end{equation*}
		Verwenden Sie $\E (U_i^2)$, um einen Momentenschätzer für $\sigma^2$ zu finden.
\end{enumerate}


\section{Kleinste-Quadrate-Schätzer}

\paragraph{Gewichtete einfache lineare Regression. }
Finden Sie eine Formel für den Kleinste-Quadrate -Schätzer $\hat \theta^w$
im Modell
\begin{equation*}
	Y_i = \theta_1 + \theta_2 x_i + \varepsilon_i
\end{equation*}
wobei $\varepsilon_1,\ldots, \varepsilon_n$ unabhängig seien mit $\varepsilon_i \sim \mathcal N(0, \sigma^2w_i)$.



\section{Maximum-likelihood Schätzer}

\paragraph{MLE. Mischung von Gleichverteilungen.  }
Seien $X_1,\ldots,X_n$ i.i.d.\ mit Dichte $p_\theta$ 
und $\theta\in \left[ 0,1 \right]$. Zeigen Sie, dass der Maximum-Likelihood-Schätzer
für
\begin{equation*}
	p_\theta (x) = \theta 1_{\left( -1,0 \right)}(x) + \left( 1-\theta \right)1_{\left( 0,1 \right)}(x)
\end{equation*}
gerade $\hat \theta = \frac{1}{n} \sum_{i=1}^{n} 1_{(-1,0)}\left( X_i \right)$ ist.

\paragraph{MLE. Normalverteilung. }
Seien $X_1,\ldots,X_n$ i.i.d.\ und $X_i \sim \mathcal N\left( \mu, \sigma^2 \right)$ mit bekanntem $\mu$. 
Zeigen Sie, dass der Maximum-Likelihood-Schätzer von $\sigma$ gerade
\begin{equation*}
	\hat \sigma^2 (x) = \frac{1}{n} \sum_{i=1}^{n} (X_i - \mu)^2  
\end{equation*}
ist.

\paragraph{MLE. Exponentialverteilung. } Seien $X_1,\ldots,X_n$ exponentialverteilt
zum Parameter $\theta$. Zeigen Sie, dass $\hat \theta = (\bar X)^{-1}$ sowohl der 
Maximum-Likelihood-Schätzer als auch ein Momentenschätzer ist.

\paragraph{Maximum-Likelihood-Methode und Suffizienz.   }
Sei $\left\{ p_\theta : \theta\in\Theta \right\}$ ein reguläres statistisches
Modell und $T\left( X \right)$ eine suffiziente Statistik für $\theta$. 
Weisen Sie nach, dass ein Maximum-Likelihood-Schätzer für $\theta$ 
eine Funktion von $T\left( X \right)$ ist.


\paragraph{MLE. Zweidimensionale Exponentialverteilung. } Betrachtet werden 
i.i.d.\ Zufallsvariablen $(Y_1, Z_1), \ldots,(Y_n,Z_n)$. Weiterhin seien
$Y_1$ und $Z_1$ unabhängig und exponentialverteilt mit Parametern
$\lambda > 0$ bzw.\ $\mu > 0$. Bestimmen Sie den Maximum-Likelihood-Schätzer
für $(\lambda,\mu)$.

\paragraph*{Lösung.}    Nachdem $Y_i$ und $Z_i$ unabhängig sind, genügt es den 
MLE für einen der beiden Zufallsvektoren zu bestimmen. Wir berechnen also den MLE für eine
eindimensionale Exponentialverteilung mit Parameter $\theta$.
Die Likelihood-Funktion ist gegeben durch
\begin{eqnarray}
	L(x, \theta) &=&  \prod_{i=1}^n \theta \exp\left(  -\theta x_i \right) \\
	&=&  \exp \left(  -\theta \sum_{}^{} x_i + \log \theta^n \right)
\end{eqnarray}
Die Log-Likelihood-Funktion ist dann gegeben durch
\begin{eqnarray}
	l(\theta, x) &=& n \log \theta - \theta \sum_{i}^{} x_i \\
	\frac{\partial}{\partial \theta} l(\theta,x) &=&  \frac{n}{\theta } - \sum_{i}^{} x_i \\
	\frac{\partial^2}{\partial \theta^2} l(\theta,x) &=& - \frac{n}{\theta^2} < 0
\end{eqnarray}
Wir setzen $\frac{\partial}{\partial \theta} l(\theta,x) = 0$ und erhalten den MLE
\begin{equation}
	\hat \theta = \frac{1}{\frac{1}{n} \sum_{i}^{} x_i} = \frac{1}{\bar x}.
\end{equation}
Somit ist der MLE für $\left( \lambda, \mu \right) = \left( 1/\bar y, 1/\bar z \right)^\top$.


\paragraph{MLE. Weibull-Verteilung. } Seien $X_1,\ldots,X_n$ i.i.d.
mit der Dichte 
\begin{equation}
	p_\theta(x)=
	\sqrt{\frac{2\theta^3}{\pi}} x^2 e^{ - \frac{\theta}{2}x^2} 1_{\R>0}(x)
\end{equation}
wobei der Parameter $\theta>0$ unbekannt ist. Finden Sie
den Maximum-Likelihood-Schätzer für $ \theta$ und klären Sie, ob dieser
eindeutig ist.

\paragraph*{Lösung. } Wir berechnen den \textsc{mle} für die oben angegebene Familie von Weibull-Verteilungen
indem wir die Log-Likelihood-Funktion maximieren. 

Die Likelihood-Funktion der Familie ist gegeben durch
\begin{eqnarray}
	L(\theta, x) &=& \left( \frac{2\theta^3}{\pi} \right)^\frac{n}{2} 
	\prod_i x_i^2 \exp\left( - \frac{\theta}{2}x_i^2 \right).
\end{eqnarray}
Für die Log-Likelihood-Funktion gilt
\begin{eqnarray}
	l(\theta,x) &=& \frac{n}{2}\left( \log 2 + 3 \log \theta - \log \pi \right) + 2\sum_{i}^{} \log x_i - \frac{\theta}{2}\sum_{i}^{} x_i^2 \\
	\frac{\partial}{\partial \theta} l(\theta,x) &=& 
	\frac{3n}{2 \theta}  - \frac{1}{2}\sum_{i}^{} x_i^2 \\
	\frac{\partial^2}{\partial \theta^2} l(\theta,x) &=& - \frac{3n}{2} \frac{1}{\theta^2} < 0 \quad \forall \theta\in\R_{>0}.
\end{eqnarray}
Somit ergibt sich die Formel für \textsc{mle} 
\begin{eqnarray}
	\frac{\partial}{\partial \theta} l(\theta,x) &=& \frac{3n}{2 \theta}  - \frac{1}{2}\sum_{i}^{} x_i^2 = 0 \\
	\hat\theta &=&  \frac{3}{\frac{1}{n} \sum_{i}^{} x_i^2}.
\end{eqnarray}
Wegen $ \frac{\partial^2}{\partial \theta^2} l(\theta,x) <0$ ist der \textsc{mle} $\hat\theta$ auch eindeutig.





\paragraph{MLE. Mischung der Verteilungen. }
Seien $p_1$ und $p_2$ zwei Dichten. Für jedes $\theta\in\left[ 0,1 \right]$ ist dann die Mischung
der beiden Verteilungen durch die Dichte
\begin{equation}
	p_\theta(x) = \theta p_1(x) + (1-\theta)p_2(x)
\end{equation}
gegeben. Betrachten Sie das parametrische Modell 
$\left\{ p_\theta : \theta \in \left[ 0,1 \right] \right\}$
und bestimmen Sie eine notwendige und hinreichende Bedingung dafür, dass die Likelihood-Gleichung
eine Lösung besitzt. Weisen Sie nach, dass diese Lösung, falls sie existiert, 
der eindeutige Maximum-Likelihood-Schätzer für $\theta$ ist. Was ist der Maximum-Likelihood-Schätzer,
wenn die Likelihood-Gleichung keine Lösung besitzt?

\paragraph*{Lösung. } 
Betrachten wir zunächst die Log-Likelihood-Funktion und ihre Ableitungen
\begin{eqnarray}
	L(\theta, x) &=&  \prod_i \left( \theta p_1(x_i) + \left( 1-\theta \right)p_2(x_i) \right) \\
	l(\theta, x) &=& \sum_{i}^{} \log \left( \theta p_1(x_i) + \left( 1-\theta \right) p_2(x_i) \right) \\
	\frac{\partial}{\partial \theta} l(\theta,x) &=&  \sum_{i}^{} \frac{p_1(x_i)-p_2(x_i)}{\theta p_1(x_i) +\left( 1-\theta \right)p_2(x_i)} \\
	\frac{\partial^2}{\partial \theta^2} l(\theta,x) &=& \sum_{i}^{} \frac{ -\left\{ p_1(x_i)-p_2(x_i) \right\}^2 }{ \left\{ \theta p_1(x_i) +\left( 1-\theta \right)p_2(x_i) \right\}^2 }
\end{eqnarray}

Betrachten wir die Frage der Existenz des MLE. Im Falle $p_1(x_i)=p_2(x_i)$ für
einen Vektor $\left( x_1,\cdots,x_n \right)$, hängt die Likelihood-Funktion
$L\left( \theta ,x \right)$ nicht von $\theta$ ab und somit kann
$\hat\theta(x)$ beliebig gesetzt werden. $\hat\theta$ ist in diesem Fall nicht
eindeutig.

Wenn es einen Vektor $\left( x_1,\cdots,x_n \right)$ gibt, der uns erlaubt $p_1$ und $p_2$ zu unterscheiden,
können wir die Bedingungen für die Existenz und Eindeutigkeit des MLE wie folgt konstruieren. 
Die Gleichung
\begin{eqnarray}
	\frac{\partial}{\partial \theta} l(\theta,x) &=& 0
\end{eqnarray}
hat genau dann genau  eine Lösung wenn $ \frac{\partial}{\partial \theta} l(\theta,x)|_{\theta=0}\geq 0$ und
$ \frac{\partial}{\partial \theta} l(\theta,x)|_{\theta=1}\leq 0$ gilt, da die zweite Ableitung
der Log-Likelihood-Funktion stickt negativ ist und somit ist $ \frac{\partial}{\partial \theta} l(\theta,x)$ 
strikt monoton fallend in $\theta$.
Im Falle
\begin{equation}
	 \frac{\partial}{\partial \theta} l(\theta,x) > 0 \quad \forall \theta\in\left[ 0,1 \right]
\end{equation}
bzw.
\begin{equation}
	 \frac{\partial}{\partial \theta} l(\theta,x) < 0 \quad \forall \theta\in\left[ 0,1 \right]
\end{equation}
hat die Maximum-Likelihood-Gleichung keine Lösung und die
die Maxima werden jeweils am Rand angenommen, d.h.\ $\hat\theta=1$ bzw. $\hat\theta=0$ und 
der \textsc{mle} $\hat\theta(x)$ kann für $x$ eindeutig festgelegt werden.





\paragraph{Verschobene Gleichverteilung. } Seien $X_1,\ldots,X_n$ i.i.d.\
mit $X_1\sim U(\theta,\theta+1)$. Der Parameter $\theta$ sei  unbekannt 
und $X_{(1)}=\min\left\{ X_1,\ldots,X_n \right\}$ die kleinste Ordnungsgröße 
der Daten und $\bar X= \frac{1}{n}\sum_{i=1}^{n}X_i$. Betrachten Sie die
beiden Schätzer 
\begin{eqnarray}
	T_1(X) &=& \bar X - \frac{1}{2} \\
	T_2(X) &=&  X_{(1)} - \frac{1}{n+1}.
\end{eqnarray}
Zeigen Sie, dass beiden Schätzer erwartungtreu sind. Berechnen Sie 
die Varianz der beiden Schätzer.

\paragraph*{Lösung.} Wir berechnen jeweils Erwartungswert und Varianz der beiden Schätzer. 
Zunächst ist 
\begin{eqnarray}
	E X_i &=&  \int_{\R}^{} x p_\theta(x) dx = \int_{\theta}^{\theta+1} x dx = \theta + \frac{1}{2}. \\
	E T_1(X) &=&  \frac{1}{2} \sum_{i}^{} X_i - \frac{1}{2} = \frac{1}{n} \sum_{i}^{} \left( \theta + \frac{1}{2} \right) -\frac{1}{2} = \theta.
\end{eqnarray}
Das heißt $T_1$ ist erwartungstreu. Die Varianz von $T_1$ berechnen wir folgendermassen
\begin{eqnarray}
	\Var T_1 &=& \Var \left[ \frac{1}{n} \sum_{i}^{} x_i - \frac{1}{2} \right] \\
	&=& E\left[ \left(  \frac{1}{n} \sum_{i}^{} X_i - \frac{1}{2} \right)^2  \right] - \left( E T_1 \right)^2 \\
	&=& E \left[  \left( \frac{1}{n} \sum_{i}^{} X_i \right)^2  - \frac{1}{n}\sum_{i}^{} X_i +\frac{1}{4} \right]-\theta^2 \\
	&=& E \left[ \left( \frac{1}{n} \sum_{i}^{} X_i \right)^2 \right] - \left( \theta + \frac{1}{2} \right) + \frac{1}{4}-\theta^2.
\end{eqnarray}
Nun ist 
\begin{eqnarray}
	E \left[ \left( \frac{1}{n} \sum_{i}^{} X_i \right)^2 \right] &=& \frac{1}{n^2} \sum_{ij}^{} E X_i X_j \\
	E X_i X_j &=& \int_{[\theta,\theta+1]^2}^{} xy dx dy = \left( \int_{\theta}^{\theta+1} t dt \right)^2 = \left( \theta+\frac{1}{2} \right)^2 \\
	E X_i^2 &=& \int_{\theta}^{\theta+1} x^2 dx = \frac{1}{3}\left( \left( \theta+1 \right)^3 - \theta^3 \right )\\
	&=& \theta^2 + \theta + \frac{1}{3}.
\end{eqnarray}
Also gilt
\begin{eqnarray}
	\Var T_1 &=& \left( 1-\frac{1}{n} \right)\left( \theta+\frac{1}{2} \right)^2+\frac{1}{n}\left( \theta^2+\theta+\frac{1}{3} \right)     - \left( \theta+\frac{1}{2} \right)^2 \\
	&=& \frac{1}{12 n} = \mathcal O \left( n^{-1} \right).
\end{eqnarray}

Nun kommen wir zu $T_2 = X_{(1)}-\frac{1}{n+1}$. Die Berechnung des Erwartungswertes führt auf
\begin{eqnarray}
	E T_2(X) &=&  -\frac{1}{n+1} + E X_{(1)} \\
	E X_{(1)} &=& \int_{[\theta,\theta+1]^n}^{} \min\left\{ x_1,\cdots,x_n \right\} dx_1\cdots d x_n \\
	&=& \int_{[0,1]^n}^{} \min\left\{ x_1+\theta,\cdots,x_n+\theta \right\} dx_1\cdots d x_n \\
	&=& \theta + \int_{[0,1]^n}^{} \min\left\{ x_1,\cdots,x_n \right\} dx_1\cdots d x_n \\
	&=& \theta +\sum_{\sigma\in S_n}^{} \int_{0\leq x_{\sigma(1)}\leq \ldots \leq x_{\sigma(n)}\leq 1} x_{\sigma(1)} dx_1 \cdots dx_n 
	\label{}
\end{eqnarray}
wobei die Summe über alle $n!$ Permutationen der $n$ Elemente geführt wird.
Es gilt
\begin{equation}
	\int_{0\leq x_1\leq \ldots\leq x_n\leq 1} x_1 dx_1 \cdots d x_n = 
	\int_{0}^{1}\cdots \int_{0}^{x_2} x_1 dx_1 \cdots dx_n = 
	\frac{1}{\left( n+1 \right)!}
\end{equation}
und
\begin{eqnarray}
	\sum_{\sigma\in S_n}^{} \int_{0\leq x_{\sigma(1)}\leq \ldots \leq x_{\sigma(n)}\leq 1} x_{\sigma(1)} dx_1 \cdots dx_n 
	=\frac{1}{n+1}.
\end{eqnarray}
Wir erhalten also
\begin{eqnarray}
	E X_{(1)} &=& \frac{1}{n+1} + \theta
\end{eqnarray}
und 
\begin{eqnarray}
	E T_2 = -\frac{1}{n+1} + E X_{(1)} = \theta.
\end{eqnarray}
Für die Berechnung der Varianz können wir folgendermassen vorgehen. 
\begin{eqnarray}
	\Var T_2 &=& E \left( T_2^2 \right) - \left( E T_2 \right)^2 \\
	E T_2^2 &=& E \left[ \left( X_{(1) - \frac{1}{n+1}} \right)^2 \right] = \\
	&=& E \left[ X_{(1)}^2 - \frac{2}{n+1}X_{(1)}+ \left( \frac{1}{n+1} \right)^2 \right] \\
	E\left[  X_{(1)}^2 \right] &=&  E \left[ \min\left\{ X_1,\cdots,X_n \right\}^2 \right] \\
	&=& \int_{[\theta,\theta+1]^n}^{} \min\left\{ x_1,\cdots,x_n \right\}^2 d \lambda^n \\
	&=& \int_{[0,1]^n}^{} \min\left\{ x_1+ \theta,\cdots,x_n+ \theta \right\}^2 d \lambda^n \\
	&=& \int_{[0,1]^n}^{} \theta^2 +2\theta\min\left\{ x_1,\cdots,x_n \right\}+ \min\left\{ x_1,\cdots,x_n \right\}^2 d\lambda^n\\
\end{eqnarray}
Mit der Hilfsrechnung
\begin{eqnarray}
	\int_{[0,1]^n}^{} \min\left\{ x_1,\cdots,x_n \right\}^2 d\lambda^n 
	&=& \sum_{\sigma\in S_n}^{} \int_{0\leq x_{\sigma(1)}\leq \ldots \leq x_{\sigma(n)}\leq 1} x_{\sigma(1)}^2 d\lambda^n \\
	&=& n! \frac{1}{\left( n+2 \right)\cdots 3} = \frac{2}{\left( n+1 \right)\left( n+2 \right)}
\end{eqnarray}
erhalten wir
\begin{eqnarray}
	E\left[ X_{(1)}^2 \right] &=& \theta^2 +  \frac{2\theta}{n+1} + \frac{2}{\left( n+1 \right)\left( n+2 \right)}
\end{eqnarray}
und
\begin{eqnarray}
	\Var T_2 &=& E \left[ X_{(1)}^2 - \frac{2}{n+1}X_{(1)}+ \left( \frac{1}{n+1} \right)^2 \right] -\theta^2 \\
	&=& \theta^2 + \frac{2\theta}{n+1} + \frac{2}{\left( n+1 \right)\left( n+2 \right)} 
	- \frac{2}{n+1}\left( \theta + \frac{1}{n+1} \right) + \left( \frac{1}{n+1} \right)^2 -\theta^2 \\
	&=& \frac{n}{\left( n+1 \right)^2 \left( n+2 \right)} = \mathcal O (n^{-2}).
\end{eqnarray}
Die Varianz von $T_1$ ist also um eine Größenordung höher als die von $T_2$.




\section{UMVUE}

\paragraph{UMVUE. Bernoulli-Verteilung. } Seien $X_1,\ldots,X_n$
i.i.d.\ und $X_1$ Bernoulli$(\theta)$-verteilt. Zeigen Sie, dass der \textsc{mls}
$\bar X$ ein \textsc{umvue}-Schätzer ist.

\paragraph*{Lösung. } Betrachten wir den Schätzer $\bar X$. Aus bereits gelösten
Übungsblättern wissen wir, dass $\bar X$ is unverzerrt für $\theta$ ist.
Wir zeigen, dass die Statistik $\sum_{i}^{} X$ suffizient und vollständig ist und 
die Behauptung folgt mit dem Satz von Lehmann-Scheff\'e. 

Dazu sei $g$ eine messbare reellwertige Funktion mit
\begin{equation}
	E_\theta \left[ g\left( \sum_{i}^{} X_i \right) \right] = 0 
\end{equation}
für alle $\theta\in \left[ 0,1 \right]$. Das ist äquivalent zu
\begin{equation}
	\sum_{k=0}^{n} g\left( k \right) P\left( \sum_{i}^{}X_i = k \right) = 0 \quad \forall\theta.
\end{equation}
Nachdem aber 
$P\left( \sum_{i}^{} X_i = k \right)= \binom{n}{k}\theta^k \left( 1-\theta \right)^{n-k}$
impliziert, dass die Funktion 
\begin{equation}
	f_n(\theta ) = \sum_{k=0}^{n} g\left( k \right) \binom{n}{k}\theta^k \left( 1-\theta \right)^{n-k}
\end{equation}
ein Polynom vom Grad höchstens $n$ in $\theta$ ist, kann es auch höchstens 
$n$ verschiedene Nullstellen
$\theta\in \left[ 0,1 \right]$ haben. Als Normalform dieses Polynoms erhalten wir
\begin{eqnarray}
	f_n(\theta) 
	&=& \sum_{k=0}^{n} g\left( k \right) \binom{n}{k}\theta^k \left( 1-\theta \right)^{n-k} \\ 
	&=& \sum_{k=0}^{n} g\left( k \right) \binom{n}{k}\theta^k \sum_{i=0}^{n-k}\binom{n-k}{i}(-1)^{n-k-i}\theta^{n-k-i} \\
\end{eqnarray}
und mit $\binom{n-k}{i}=\binom{n-k}{n-k-i}$ schließlich
\begin{eqnarray}
	f_n(\theta) 
	&=& \sum_{k=0}^{n} g\left( k \right) \binom{n}{k}\theta^k \sum_{i=0}^{n-k}\binom{n-k}{n-k-i}(-1)^{n-k-i}\theta^{n-k-i} \\
	&=& \sum_{k=0}^{n} g\left( k \right) \binom{n}{k}\theta^k \sum_{i=0}^{n-k}\binom{n-k}{i}(-1)^i\theta^i \\
	&=& \sum_{k=0}^{n} \sum_{i=0}^{n-k}\binom{n-k}{i} \binom{n}{k} (-1)^i  g\left( k \right)  \theta^{i+k} \\
	&=& \sum_{p=0}^{n} \left[ \sum_{k=0}^{p} \binom{n-k}{p-k}\binom{n}{k}(-1)^{p-k} g(k) \right] \theta^p.
\end{eqnarray}
Durchs Betrachten der Koeffizienten dieses Polynoms in aufsteigender Reihenfolge erhalten wir
induktiv $g(0)=0$, $g(1)=0,\ldots$ und nach $n$ Schritten $g(n)=0$.
Damit ist $g \equiv 0$ und die Statistik $\sum_{i}^{} X_i$
vollständig. Aus den bereits gelösten Übungsblättern wissen wir, dass $\sum_{i}^{} X_i$
suffizient für $\theta$ ist. Mit
\begin{equation}
	E\left[ \bar X | \sum_{i}^{} X_i \right] = \bar X
\end{equation}
und dem Satz von Lehmann-Scheff\'e folgt die Behauptung.





\paragraph{Vollständigkeit und UMVUE. }  Seien $X_1,\ldots,X_n$ i.i.d.,
wobei $X_1$ eine diskrete Zufallsvariable mit Wahrscheinlichkeitsfunktion
\begin{equation}
	p_\theta(x)=P_\theta(X_1=x)= 
	\left( \frac{\theta}{2} \right)^{|x|} (1-\theta)^{1-|x|},\  
	x\in\left\{ -1,0,1 \right\},
\end{equation}
und unbekanntem Parameter $\theta\in (0,1)$ sei. Untersuchen Sie die beiden
Schätzer $T_1(X)=X_1$ und $T_2(X)=|X_1|$ auf Vollständigkeit. Bestimmen Sie
einen \textsc{umvue}-Schätzer für $\theta$.

\paragraph*{Lösung. } 
Wir untersuchen zuerst die Vollständigkeit der beiden Schätzer $T_1$ und $T_2$. 
Die Dichte von $X_1,\ldots,X_n$ ist
\begin{eqnarray}
	p_\theta(x_1, \cdots, x_n) &=& \prod_i \left( \frac{\theta}{2} \right)^{|x_i|} \left( 1-\theta \right)^{1-|x_i|} \\
	&=& \left( \frac{\theta}{2} \right)^{\sum_{i}^{} |x_i|} \left( 1-\theta \right)^{n-\sum_{i}^{} |x_i|}.
\end{eqnarray}
Nehmen wir an, dass für alle $\theta$
\begin{equation}
	E_\theta g\left( T_1\left( X \right) \right) = 0
\end{equation}
gilt. Das ist äquivalent zu 
\begin{eqnarray}
	E_\theta g\left( X_1 \right) &=&  \sum_{x_i\in\left\{ -1,0,1 \right\}}^{} g(x_1) p_\theta \left( x_1,\cdots,x_n \right) \\
	&=& \sum_{x_i\in\left\{ -1,0,1 \right\}}^{} g(x_1) \left( \frac{\theta}{2} \right)^{\sum_{i}^{} |x_i|} \left( 1-\theta \right)^{n-\sum_{i}^{} |x_i|} \\
	&=& \sum_{x_1\in\left\{ -1,0,1 \right\}}^{} g(x_1) \left[  \sum_{x_{2,\ldots,n}\in\left\{ -1,0,1 \right\}}^{}\left( \frac{\theta}{2} \right)^{\sum_{i}^{} |x_i|}\left( 1-\theta \right)^{n-\sum_{i}^{} |x_i|}   \right].
\end{eqnarray}
Sei nun
\begin{eqnarray}
	A(x_1, \theta) &=& \sum_{x_{2,\ldots,n}\in\left\{ -1,0,1 \right\}}^{}\left( \frac{\theta}{2} \right)^{\sum_{i}^{} |x_i|}\left( 1-\theta \right)^{n-\sum_{i}^{} |x_i|}.
\end{eqnarray}
Es ist also 
\begin{equation}
	E_\theta g\left( T_1\left( X \right) \right) = \sum_{x_i\in\left\{ -1,0,1 \right\}}^{} g(x_1) A\left( x_1, \theta \right).
\end{equation}
Nachdem aber $A\left( x_1,\theta \right) = A(-x_1,\theta)$ wählen wir $g\left( x \right)=x$ und erhalten
\begin{equation}
	E_\theta g\left( T_1\left( X \right) \right) = 0 \quad \forall \theta\in\Theta
\end{equation}
obwohl $g\nequiv 0$. $T_1(X)=X_1$ ist demzufolge nicht vollständig. 

Für $T_2\left( X \right)$ führen wir gleiche Rechnung durch und erhalten
\begin{eqnarray}
	E_\theta(g\left( T_2\left( X \right) \right)) &=& \sum_{x_i\in\left\{ -1,0,1 \right\}}^{} g(|x_1|) A\left( x_1, \theta \right) \\
	&=& 2 g(1) A(1,\theta) + g(0) A(0,\theta) = 0 \quad \forall\theta\in (0,1).
\end{eqnarray}
$A(1,\theta)$ und $A(0,\theta)$ sind Polynome vom Grad $n$. $A(0,\theta)$ hat aber im Gegensatz zu $A(1,\theta)$ 
einen nicht verschwindenden konstanten Koeffizienten mit Wert $1$. Dieser Koeffizient kommt vom Summanden $(x_2,\ldots,x_n)=0$
in der Darstellung
\begin{equation}
	A(0, \theta) = \sum_{x_{2,\ldots,n}\in\left\{ -1,0,1 \right\}}^{}\left( \frac{\theta}{2} \right)^{\sum_{i}^{} |x_i|}\left( 1-\theta \right)^{n-\sum_{i}^{} |x_i|}.
\end{equation}
Daher ist $g\equiv 0$ und die Statistik $T_2(X)$ vollständig. 

Für die Berechnung eines \textsc{umvue}-Schätzers sind die obigen Berechnungen nur indirekt 
hilfreich, da sowohl $T_1(X)$ als auch $T_2(X)$ nicht suffizient für $\theta$ sind. 

Für das Auffinden eines \textsc{umvue}-Schätzers benötigen wir einiger Beobachtungen. 
Zunächst ist die Zufallsvariable $|X_i|$ Bernoulli verteilt und daher gilt $E_\theta |X_i|=\theta$.
$T_3(X)= \frac{1}{n} \sum_{i=1}^{n} |X_i|$ ist also ein unverzerrter Schätzer für $\theta$ und
binomialverteilt. 
$T_3$ ist auch suffizient für $\theta$, denn es gilt
\begin{eqnarray}
	P( X=\left( x_1,\ldots,x_n \right)| T_3=y) &=& \frac{ 1_{ \left\{ \sum_{i}^{} |x_i|=y \right\} } \left( \frac{\theta}{2} \right)^{\sum_{i}^{} |x_i|}  \left( 1-\theta \right)^{n-\sum_{i}^{} |x_i| }   }{ \binom{n}{k} \theta^{\sum_{i}^{} |x_i|   } (1-\theta)^{n-\sum_{i}^{} |x_i|}        } \\
	&=& 1_{ \left\{ \sum_{i}^{} |x_i|=y \right\} } \binom{n}{k}^{-1} \left( \frac{1}{2} \right)^{\sum_{i}^{} |x_i|}.
\end{eqnarray}
Daher ist $T_3$ ein \textsc{umvue}-Schätzer für $\theta$.





\paragraph{Stichprobenvarianz als UMVUE bei Normalverteilung.} 
Seien $X_1,\ldots,X_n$ i.i.d.\ mit $X_i \sim \mathcal N(\mu,\sigma^2)$ und
$\mu\in\R$ sowie $\sigma>0$.  Dann ist die Stichprobenvarianz 
\begin{equation}
	s^2(X) = \frac{1}{n-1} \sum_{i=1}^{n} (X_i - \bar X)^2
\end{equation}
ein \textsc{umvue}-Schätzer für $\sigma$, falls $\mu$ unbekannt ist. 
Ist $\mu$ hingegen bekannt, so ist $s^2(X)$ kein \textsc{umvue}-Schätzer
von $\sigma$. 

\paragraph*{Lösung.} 
Die erste Aussage wurde bereits in der Vorlesung behandelt. 

Sei also $\mu=\mu_0$ bekannt und fix.  Wir wollen die Normalverteilung mit
bekanntem Erwartungswert als exponentielle Familie schreiben und mit Hilfe
eines Satzes aus der Vorlesung eine vollständige und suffiziente Statistik
gewinnen. Es gilt
\begin{eqnarray}
	p_\sigma(x_i) &=& \prod_i \frac{1}{\sqrt{2 \pi \sigma^2}} 
	\exp\left( -\frac{1}{2} \left( \frac{x_i - \mu_0}{\sigma}\right)^2 \right) \\
	&=& \exp\left( -\frac{1}{2 \sigma^2} \sum_{i}^{} \left( x_i - \mu_0 \right)^2 +
		\log\left( 2 \pi \sigma^2 \right)^{-\frac{n}{2}}  \right).
\end{eqnarray}
Damit erhalten wir $c(\sigma^2) = -\frac{1}{2 \sigma^2}$ und $T(X) = \sum_{i}^{} \left( X_i - \mu_0 \right)^2$.
Da $c\left( \Theta \right) = c(\R_{>0})= \R_{<0}$ ein offenes Rechteck enthält, ist $T\left( X \right)$
eine vollständige und suffiziente Statistik für $\sigma^2$.

Da die Stichprobenvarianz
\begin{equation}
	\hat\sigma^2(X) = \frac{1}{n} \sum_{i=1}^{n} (X_i - \mu_0)
\end{equation}
bei festgehaltenem $\mu_0$ ein unverzerrter Schätzer für $\sigma^2$ ist, können 
wir den Satz von Lehmann-Scheff\'e anwenden und berechnen 
\begin{equation}
	T^*\left( X \right) = E\left[ \sum_{i=1}^{n} \frac{1}{n} \left( X_i-\mu_0 \right)^2 | \sum_{i=1}^{n} \left( X_i-\mu_0 \right)^2 \right] = \sum_{i=1}^{n} \frac{1}{n} \left( X_i-\mu_0 \right)^2 = \hat\sigma^2(X).
\end{equation}
Die Stichprobenvarianz $\hat\sigma^2(X)$ ist somit der eindeutige \textsc{umvue}-Schätzer
für $\sigma^2$. Daher kann $s^2(X)$ in diesem Kontext kein
\textsc{umvue}-Schätzer sein.







\paragraph{UMVUE. Normalverteilung. } Sei $X_1,\ldots,X_n$ eine i.i.d.\
Stichprobe mit $X_1\sim\mathcal N(\mu,1)$. Finden Sie den 
\textsc{umvue}-Schätzer für
\begin{equation}
	P_\mu (X_1>0).
\end{equation}
\emph{Hinweis:} Betrachten Sie die gemeinsame Verteilung von $(X_1,\bar X)$.

\paragraph*{Lösung. }   Der Satz von Lehmann-Sch\'effe ist das wichtigste Tool zur Gewinnung
der \textsc{umvue}-Schätzern das uns derzeit zur Verfügung. Wir versuchen ihn auch diesmal anzuwenden. 

Dazu brauchen wir einen unverzerrten Schätzer für $P\left( X_1 >0 \right)$, also einen
Schätzer $S(X)$ mit der Eigenschaft
\begin{equation}
	E_\mu \left[ S(X) \right] = P_\mu(X_1 > 0).
\end{equation}
Dies ist aber genau die Eigenschaft der Zufallsvariable $1_{ \left\{ X_1 > 0 \right\}  }$, denn
\begin{equation}
	E_\mu \left[   1_{ \left\{ X_1 > 0 \right\}  }  \right] = P_\mu( X_1 > 0).
\end{equation}
Als eine suffiziente Statistik verwenden wir $\bar X$. Nun mit Hilfe der Rao-Blackwillisierung
erhalten wir einen \textsc{UMVU} Schätzer
\begin{equation}
	E_\mu \left[ 1_{ \left\{ X_1 > 0 \right\}  } | \bar X \right]
	= P_\mu \left( X_1 > 0 | \bar X \right) 
	= P_\mu \left( X_1 - \bar X > -\bar X | \bar X \right).
\end{equation}
Da $X_1$ und $\bar X$ normalverteilt sind mit $E X_1 = E \bar X = \mu$, 
$\Var \left( X_1 \right)=1$ und $\Var\left( \bar X \right) = \frac{1}{n}$ sowie
$\cov \left( X_1- \bar X, \bar X \right) = \cov(X_1, \bar X) - \cov(\bar X, \bar X) = 0$ gilt, 
sind die Zufallsvariablen $X_1 - \bar X$ und $\bar X$ unkorreliert und damit unabhängig. 
So ist
\begin{equation}
	P_\mu \left( X_1 - \bar X > -\bar X | \bar X = \bar x \right) = P_\mu \left( Z > -\bar x  \right)
\end{equation}
für $Z=X_1 - \bar X$. Nachdem $EZ=0$ und 
\begin{eqnarray}
	\Var Z &=& \Var \left( X_1- \bar X \right) \\
	&=& \Var \left( \frac{n-1}{n} X_1 - \frac{1}{n} \sum_{i\geq 2}^{} X_i  \right) \\
	&=& \left( \frac{n-1}{n} \right)^2 + \frac{1}{n^2} \sum_{i\geq 2}^{} \Var \left( X_i  \right) \\
	&=& \frac{n-1}{ n}
\end{eqnarray}
is $\frac{Z}{\sqrt{\frac{n-1}{2}}}$ standardnormalverteilt. So gilt
\begin{eqnarray}
	P_\mu \left( Z > -\bar X \right) &=& 
	P_\mu \left(  \frac{Z}{ \sqrt{\frac{n-1}{n}} } > \frac{-\bar X}{ \sqrt{\frac{n-1}{n}}   } \right) \\
	&=& 1- \Phi\left( \frac{-\bar X}{ \sqrt{\frac{n-1}{n}}}    \right) 
\end{eqnarray}
und ein \textsc{umvue}-Schätzer für $P\left( X_1 >0 \right)$ ist bestimmt.




\paragraph{Cram\'er-Rao Regularität und exponentielle Familien. }
Für eine einparametrige exponentielle Familie mit
\begin{equation}
	p(x, \theta) = \exp \left( c(\theta) T(x) + d(\theta) +S(x) \right)1_A(x),
\end{equation}
sodass $c$ differenzierbar mit $\frac{\partial}{\partial \theta} c(\theta)\neq 0$ 
für alle $\theta\in \Theta$ und $\Theta$ offen in $\R$ ist, sind die Bedingungen (\textsc{cr})
erfüllt.

\emph{Hinweis:} Sie können für die Vertauschbarkeit von $\frac{\partial}{\partial \theta}$ und $\int_{}^{}$
den Satz von der monotonen Konvergenz verwenden. 

\paragraph*{Lösung. }
Für eine Statistik $T(X)$ mit $E_\theta (|T(X)|) < \infty$ zeigen wir, dass für alle $\theta\in\Theta$
\begin{equation}
	\frac{\partial}{\partial \theta} \int_{\R}^{} T(x) p(x,\theta) dx = 
	\int_{\R}^{} \frac{\partial}{\partial \theta} p(x,\theta) T(x) dx
	\label{main-interchange}
\end{equation}
gilt.

Zunächst können wir die exponentielle Darstellung von $p(x,\theta)$
reparametrisieren und die Dichte als natürliche exponentielle Familie 
\begin{equation}
	p(x,\theta) = \exp \left( \theta T(x) + d_0(\theta) + S(x) \right)1_A(x)
\end{equation}
schreiben.

Wir vereinfachen nun die Darstellung von $p(x,\theta)$ indem wir das eindimensionale
Lebesguemass durch ein absolut stetiges Mass auf $\R$ ersetzen. So ist 
\begin{eqnarray}
	\int_{\R}^{} T(x) p(x,\theta) dx &=& 
	\int_{\R}^{} T(x) \exp \left( \theta T(x) + d_0(\theta) + S(x) \right)1_A(x) dx \\
	&=& \int_{\R}^{} T(x) \exp \left( \theta T(X) + d_0(\theta) \right) d\mu 
\end{eqnarray}

Wir zeigen nun ein Hilfsresultat. Es gilt
\begin{equation}
	\frac{\partial}{\partial \theta} \int_{\R}^{} e^{\theta T(x)} d\mu 
	= \int_{\R} \frac{\partial}{\partial \theta} e^{\theta T(x)} d\mu.
	\label{simple-interchange}
\end{equation}
falls die entsprechenden Integrale existieren und endlich sind.
Die Ableitung von $\int e^{\theta T(x)}$ nach $\theta$ ist der Grenzwert des Quotienten
\begin{equation}
	\int \frac{e^{\theta_1 T(x)} - e^{\theta_2 T(x)}}{\theta_1 - \theta_2}.
\end{equation}
Den Integranden schätzen wir mit $|e^y - e^x|\leq e^x |y-x|$ wie folgt
\begin{eqnarray}
	\left| \frac{e^{\theta_1 T(x)} - e^{\theta_2 T(x)}}{\theta_1 - \theta_2} \right| &\leq& 
	\frac{e^{\theta_2 T(x)} | (\theta_2 -\theta_1) T(x) | }{ |\theta_1 - \theta_2| }
\end{eqnarray}
ab. Die letzte Funktion ist integrierbar was zusammen mit dem Satz von der majorisierten Konvergenz
die Gleichung (\ref{simple-interchange}) rechtfertigt. Beachten Sie, dass wir bei dieser Überlegung
die Voraussetzungen $\Theta$ ist offen und $\frac{\partial}{\partial \theta}c(\theta)\neq 0$
benutzt haben.

Da $p(x,\theta)$ eine Dichte ist und somit integrierbar, können wir aus
\begin{equation}
	\int_{\R}^{} e^{\theta T(x) + d_0(\theta)}d \mu = 1
\end{equation}
die Darstellung
\begin{equation}
	e^{-d_0(\theta)} = \int_{\R}^{} e^{\theta T(x)} d \mu
\end{equation}
ableiten und stellen mit Hilfe von (\ref{simple-interchange}) die Differenzierbarkeit von $d_0(\theta)$ fest.


$E_\theta \left( | T(X) | \right) < \infty$ impliziert, dass $T(X)$ als Differenz zweier
positiver integrierbarer Funktionen $T(X) = T^+(X) - T^-(X)$ geschrieben werden kann.
Das führt uns zu
\begin{eqnarray}
	\int_{\R}^{} T(x) e^{ \theta T(X) + d_0(\theta) } d\mu &=& 
	\int_{\R}^{} T(x)^+ e^{ \theta T(X) + d_0(\theta) } d\mu - \\
	&& \int_{\R}^{} T(x)^- e^{ \theta T(X) + d_0(\theta) } d\mu  \\
	&=&  e^{d_0(\theta)} \int_{\R}^{} e^{ \theta T(X) } d\mu^+ - e^{d_0(\theta)} \int_{\R}^{} e^{ \theta T(X) } d\mu^-
\end{eqnarray} 
indem wir $T^+$ und $T^-$ in den neuen Massen $\mu^+$ und $\mu^-$ verschwinden lassen.
Nun können wir das Resultat (\ref{simple-interchange}) und die Differenzierbarkeit von $d_0(\theta)$
anwenden und erhalten (\ref{main-interchange}).

Mit Hilfe der obigen Überlegungen können wir auch die Existenz der Ableitung 
$\frac{\partial}{\partial \theta} \log p(x,\theta)$ sicherstellen.
Die Übrigen Bedingungen aus (\textsc{cr}) sind in unseren Voraussetzungen enthalten.






\paragraph{UMVUE. Exponentialverteilung.   }
Seien $X_1,\ldots,X_m$ i.i.d.\ und $X_i\sim \textrm{Exp}(\theta)$ exponentialverteilt
mit der Dichte $\theta e^{-\theta x}1_{\R_{>0}}(x)$ und $\theta>0$.
Finden Sie einen \textsc{umvue}-Schätzer für $q(\theta) = \frac{1}{\theta^2}$. Zeigen Sie, dass
dieser die untere Schranke der Informationsgleichung nicht annimmt.

\paragraph*{Lösung.} Wir stellen die Dichte der Exponentialverteilung als exponentielle
Familie 
\begin{equation}
    p_\theta(x) = \exp \left( -\theta x + \log \theta \right) 1_{\R_{>0}}(x)
\end{equation}
dar. Damit ist $T(X)=\sum_{i}^{} X_i$ eine vollständige und suffiziente Statistik.

Laut Satz von Lehmann-Sch\'effe muss der eindeutige \textsc{umvue}-Schätzer eine Funktion $h(T(X))$
der vollständigen Statistik $T(X)$ sein. Nachdem der Schätzer $h(T(X))$ erwartungstreu
ist, muss
\begin{equation}
    E_\theta h(T(X)) = \frac{1}{\theta^2}
\end{equation}
gelten. Mit Hilfe von $\sum_{i}^{} X_i \sim \textrm{Gamma}(n,\theta)$, können wir eine 
Gleichung für $h$ konstruieren.
\begin{equation}
    E_\theta h(T(X)) =  \int_{\R} h(x) \frac{\theta^n}{\Gamma(n)} x^{n-1}e^{-\theta x} 1_{\R_{>0}}(x) dx =\frac{1}{\theta^2} 
\end{equation}
\begin{equation}
    \int_{0}^{\infty} h(x) \frac{\theta^{n+2}}{\Gamma(n)}  x^{n-1} e^{-\theta x} dx = 1
\end{equation}
Die Gleichheit gilt für $h(x)= \frac{x^2}{n(n+1)}$. Damit ist 
\begin{equation}
    S(X) = h\left( \sum_{i}^{} X_i \right) = \frac{\left( \sum_{i}^{} X_i \right)^2}{n(n+1)}
\end{equation}
der eindeutige \textsc{umvue}-Schätzer für $\frac{1}{\theta^2}$.

Wir berechnen nun die Schranke der Informationsungleichung -- die \textsc{(cr)} Bedingungen
sind erfüllt, da $\textrm{Exp}(\theta)$ eine exponentielle Familie mit $c'(\theta)\neq 0$ und 
offenem Parameterraum ist. Es gilt $\Psi(\theta)=\frac{1}{\theta^2}$ und 
$\left( \Psi'(\theta) \right)^2= \frac{4}{\theta^6}$, sowie
\begin{eqnarray}
    I_1(\theta) &=& E_\theta \left( \frac{\partial}{\partial \theta} \log p(X, \theta) \right)^2 \\
    &=& E_\theta \left( \frac{\partial}{\partial \theta} \log \theta -\theta x \right)^2 = \frac{1}{\theta^2}.
\end{eqnarray}
Damit bekommen wir die Schranke für die Varianz
\begin{equation}
    \frac{\left( \Psi'(\theta) \right)^2}{n I_1(\theta) } = \frac{4}{n \theta^4}.
\end{equation}
Mit dieser Schranke können wir die Varianz des Schätzers $S(X)$ in Beziehung setzen.
\begin{eqnarray}
    E_\theta S(X)^2 &=& \int_{0}^{\infty} \left( \frac{t^2}{n(n+1)} \right)^2 \frac{\theta^n}{\Gamma(n)} t^{n-1} e^{-\theta t} dt \\
    &=& \frac{(n+2)(n+3)}{n(n+1)\theta^4} \int_{0}^{\infty} \frac{\theta^4}{\Gamma(n+4)}t^{n+4-1} e^{-\theta t} dt \\
    &=&  \frac{(n+2)(n+3)}{n(n+1)\theta^4} \\
    \Var_\theta S(X) &=& \frac{(n+2)(n+3)}{n(n+1)\theta^4} - \frac{1}{\theta^4}
\end{eqnarray}
Nun zeigen wir
\begin{equation}
    \Var_\theta S(X) > \frac{\left( \Psi'(\theta) \right)^2}{n I_1(\theta) }
\end{equation}
\begin{equation}
    \frac{(n+2)(n+3)}{n(n+1)\theta^4} - \frac{1}{\theta^4} > \frac{4}{n \theta^4}
\end{equation}
\begin{equation}
    \frac{(n+2)(n+3)}{n(n+1)\theta^4} > \frac{n+4}{\theta^4 n}
\end{equation}
\begin{equation}
    (n+2)(n+3) > (n+4)(n+1).
\end{equation}







\paragraph{UMVUE. Gamma-Verteilung. } Eine Stichprobe $X_1, \ldots, X_n$ sei i.i.d.\
und Gamma-verteilt mit bekanntem Parameter $a>0$ und unbekanntem Parameter 
$\lambda>0$, d.h.\ $X_1$ hat die Dichte
\begin{equation}
    p_\lambda (x) = \frac{\lambda^a}{\Gamma(a)} x^{a-1}e^{-\lambda x} 1_{\R_{>0}}(x).
\end{equation}
Finden Sie mit Hilfe der Informationsungleichung einen \textsc{umvue}-Schätzer für
$q(\lambda) = \frac{1}{\lambda}$.

\paragraph*{Lösung.  } Die Lösungsmethode ist hier einen unverzerrten Schätzer mit der
kleinstmöglichen Varianz zu finden, d.h.\ der Varianz aus der Informationsungleichung.

Zuerst überprüfen wir die (\textsc{cr}) Bedingungen indem wir $p_\lambda(x)$ als
exponentielle Familie
\begin{equation}
    p_\lambda(x) = \exp\left( a\log \lambda -\Gamma(a) +(a-1)\log x -\lambda x  \right) 1_{\R_{>0}}(x)    
\end{equation}
darstellen. $c(\lambda)=-\lambda$ ist differenzierbar mit $c'(\lambda)=-1\neq 0$ und 
$\Theta=\R_{>0}$ ist offen. Damit sind \textsc{(cr)} Bedingungen erfüllt.

Nachdem $q(\lambda)=\frac{1}{\lambda}$ gerade $E_\lambda X_i=\frac{a}{\lambda}$ ist, 
wählen wir den Schätzer 
\begin{equation}
    T(X)=\frac{1}{a}\bar X.
\end{equation}
$T(X)$ ist erwartungstreu und hat
die Varianz
\begin{eqnarray}
    \Var_\lambda T(X) &=&  \Var_\lambda \frac{1}{na} \sum_{i}^{} X_i \\ 
    &=& \frac{1}{a^2 n^2} \Var_\lambda \sum_{i}^{} X_i \\
    &=&  \frac{1}{a^2 n^2} \frac{an}{\lambda^2} = \frac{1}{an \lambda^2}. 
\end{eqnarray}
Letzes gilt wegen $\sum_{i}^{} X_i \sim \textrm{Gamma}(na,\lambda)$. Nun berechnen wir die
Fischer-Information für $X=X_i$.
\begin{eqnarray}
    I_1(\lambda) &=&  \Var_\lambda \frac{\partial}{\partial \lambda} \log p(X,\lambda) \\
    &=& \Var_\lambda \frac{\partial}{\partial \lambda} \log \frac{\lambda^a}{\Gamma(a)} X^{a-1}e^{-\lambda X} 1_{\R_{>0}}(X) \\
    &=& \Var_\lambda \frac{\partial}{\partial \lambda} \left( a \log \lambda - \lambda X \right) \\
    &=& \Var_\lambda \left( \frac{a}{\lambda} - X \right) = \frac{a}{\lambda^2}
\end{eqnarray}
Mit 
\begin{equation}
    \Psi(\lambda) = q(\lambda)=E_\lambda T(X) =\frac{1}{\lambda} \\
\end{equation}
\begin{equation}
    \left( \Psi'(\lambda) \right)^2 = \frac{1}{\lambda^4}
\end{equation}
ergibt sich 
\begin{equation}
    \frac{  \left( \Psi'(\lambda) \right)^2 }{n I_1(\lambda)} 
    = \frac{1}{\lambda^4} \frac{\lambda^2}{an} = \frac{1}{ an \lambda^2}
\end{equation}
und
\begin{equation}
    \Var_{\lambda} T(X) = \frac{  \left( \Psi'(\lambda) \right)^2 }{I(\lambda)}.
\end{equation}
Damit ist $T(X)$ ein \textsc{umvue}-Schätzer für $q(\lambda)=\frac{1}{\lambda}$.







\paragraph{Cram\'er-Rao-Schranke und die Gleichverteilung.}   % Shao E 3.22
Seien $X_1,\ldots,X_n$ i.i.d.\ und $X_i \sim \mathcal U(0,\theta)$ mit unbekanntem 
$\theta > 0$. Es bezeichne $I(\theta)$ die Fischer-Information. Weisen Sie nach, dass
$T(X)=\frac{n+1}{n}X_{(n)}$ ein erwartungstreuer Schätzer für $\theta$ ist und
\begin{equation}
	\Var_\theta \left( T(X) \right) < \frac{1}{I(\theta)}, \quad \textrm{für alle } \theta>0. 
\end{equation}
Klären Sie, wieso dies nicht im Widerspruch zu der Cram\'er-Rao-Schranke
\begin{equation}
	\Var_\theta\left( T(X) \right) \geq \frac{1}{I(\theta)}
\end{equation}
steht.

\paragraph*{Lösung. } Wir zeigen zuerst, dass $T(X)$ ein erwartungstreuer Schätzer
für $\theta$ ist. Dazu berechnen wir die Verteilung von $X_{(n)}$.
\begin{eqnarray}
    P\left( X_{(n)} \leq t \right) &=& P\left( X_1\leq t,\ldots, X_n\leq t \right) \\
    &=& \left( P(X_1 \leq t) \right)^n
\end{eqnarray}
Daraus ergibt sich
\begin{eqnarray}
    E_\theta X_{(n)} &=& \int_{0}^{\theta} x dP_{ X_{(n)} }(x) \\
    &=& \int_{0}^{\theta} x \left( \frac{x^n}{\theta^n} \right)' dx \\
    &=& \int_{0}^{\theta} n \frac{x^n}{\theta^n} dx \\
    &=& \frac{n}{(n+1)\theta^n} \int_{0}^{\theta} (n+1)x^n dx \\
    &=& \frac{n}{n+1} \frac{1}{\theta^n} \theta^{n+1} = \frac{n}{n+1} \theta.
\end{eqnarray}
Daher ist $T(X) = \frac{n+1}{n} X_{(n)}$ erwartungstreu für $\theta$.

Die Fischer-Information für $X=(X_1,\ldots,X_n)\in [0,\theta]^n$ mit der Dichte
$p_\theta(X) = \theta^{-n} 1_{[0,\theta]^n}(X)$ ist 
\begin{eqnarray}
    I(\theta) &=& E_\theta\left( \frac{\partial}{\partial \theta} \log \frac{1}{\theta^n} \right)^2 \\
    &=& E_\theta \left( \frac{\partial}{\partial \theta} -n \log \theta \right)^2 \\
    &=& E_\theta \frac{n^2}{\theta^2} = \left( \frac{n}{\theta} \right)^2.
\end{eqnarray}

Berechnen wir nun
\begin{eqnarray}
    \Var T(X) &=& \Var \frac{n+1}{n} X_{(n)} \\
    &=& E\left( \frac{n+1}{n} X_{(n)} \right)^2 - \left( E \frac{n+1}{n} X_{(n)} \right)^2 \\
    &=& \left( \frac{n+1}{n} \right)^2 E X_{(n)}^2 - \theta^2 \\
    E X_{(n)}^2 &=& \int_{0}^{\theta} x^2 d P_{X_{(n)}}(x) \\
    &=& \frac{n}{\theta^n} \int_{0}^{\theta} x^{n+1} = \frac{\theta^2 n}{n+2} \\
    \Var T(X) &=& \left( \frac{n+1}{n} \right)^2 \frac{\theta^2 n }{n+2} - \theta^2 = \frac{\theta^2}{n(n+2)}.
\end{eqnarray}
Damit ist 
\begin{equation}
    \Var_\theta T(X) = \frac{\theta^2}{n(n+2)} < \frac{\theta^2}{n^2} = \frac{1}{I(\theta)}.
\end{equation}

Erklärung dafür ist die folgende. Der
Träger der Dichte von $X_i$ ist vom Parameter $\theta$ abhängig und daher sind
die \textsc{(cr)} Bedingungen verletzt.  Deswegen muss die
Informationsungleichung für dieses Modell nicht gelten. 







\paragraph{Cram\'er-Rao-Schranke ist nicht scharf.    } 
Es ist durchaus möglich, dass ein \textsc{umvue}-Schätzer eine größere Varianz als die untere
Schranke
\begin{equation}
	\Var_\theta \left( T(X) \right) \geq \frac{1}{I(\theta)}
\end{equation}
hat. Betrachtet wenden dazu $X_1,\ldots,X_n$ i.i.d.\ mit $X_i \sim \textrm{Poiss}(\theta)$ für 
unbekanntes $\theta>0$. Zeigen Sie, dass
\begin{equation}
	T(X) = \left( 1- \frac{1}{n} \right)^{ \sum_{i=1}^{n} X_i}
\end{equation}
ein \textsc{umvue}-Schätzer für $g(\theta)=e^{-\theta}$ ist. Zeigen Sie weiterhin, dass die
Varianz von $T(X)$ die Schranke in der Informationsungleichung
\begin{equation}
	\Var_\theta\left( T(X) \right) \geq \frac{\Phi^{'}(\theta)^2}{I(\theta)}
\end{equation}
für kein $\theta$ annimmt.

\paragraph*{Lösung. } Wir stellen zuerst sicher, dass $T(X)$ ein \textsc{umvue}-Schätzer 
für $g(\theta)=e^{-\theta}$. Nachdem die Summe von poissonverteilten Zufallsvariablen
wieder poissonverteilt ist, und zwar 
$\sum_{i}^{} X_i \sim \textrm{Poiss}(\sum_{i}^{} \lambda_i)$ für 
$X_i \sim \textrm{Poiss}(\lambda_i)$, erhalten wir
\begin{eqnarray}
    E_\theta T(X) &=& \sum_{k\geq 0} \left( 1-\frac{1}{n} \right)^k P\left( \sum_{i}^{} X_i =k \right)\\
    &=& \sum_{k\geq 0}^{} \left( 1-\frac{1}{n} \right)^k e^{-n\theta} \frac{(n\theta)^k}{k!} \\
    &=& e^{-n\theta} \sum_{k\geq 0} \frac{ (n-1)^k\theta^k}{k!} \\
    &=& e^{-n \theta} e^{(n-1)\theta} = e^{-\theta}.
\end{eqnarray}
$T(X)$ ist also erwartungstreu für $g(\theta)$.

Aus der Vorlesung wissen wir, dass $\sum_{i}^{} X_i$ eine vollständige und suffiziente
Statistik ist, wenn die Zufallsvariablen $X_i$ poissonverteilt sind. $T(X)$ ist 
eine Funktion von $\sum_{i}^{} X_i$ und damit ein \textsc{umvue}-Schätzer
für $e^{-\theta}$.

Wir berechnen nun die Varianz von $T(X)$.
\begin{eqnarray}
    \Var_\theta T(X)&=& \Var_\theta \left( 1-\frac{1}{n} \right)^{\sum_{i} X_i} \\ 
    &=& E_\theta \left( \left( 1-\frac{1}{n} \right)^{\sum_{i} X_i} \right)^2 - \left( E_\theta \left( 1-\frac{1}{n} \right)^{\sum_{i} X_i} \right)^2 \\
    E_\theta \left( 1-\frac{1}{n} \right)^{\sum_{i} X_i} &=& e^{-\theta} \\
    E_\theta \left( \left( 1-\frac{1}{n} \right)^{\sum_{i} X_i} \right)^2 
    &=& \sum_{k\geq 0}^{} \left( 1-\frac{1}{n} \right)^{2k} P\left( \sum_{i}^{} X_i =k \right) \\
    &=& \sum_{k\geq 0}^{} \left( 1-\frac{1}{n} \right)^{2k} e^{-n\theta} \frac{(n\theta)^k}{k!} \\
    &=& e^{-n \theta} \sum_{k\geq 0} \frac{ \left( \frac{ (n-1)^2 \theta }{n} \right)^k   }{k!} \\
    &=& \exp\left( -n\theta \right) \exp\left( \frac{(n-1)^2\theta}{n} \right)
    = \exp\left( \theta\left( -2 + \frac{1}{n} \right) \right) \\
    \Var_\theta T(X) &=& \exp\left( \frac{\theta(1-2n)}{n} \right) - \exp\left( -2\theta \right). 
\end{eqnarray}

Nun zur Informationsungleichung. Wir haben $\Psi(\theta)=g(\theta)=E T(X)$ und 
$\Psi'(\theta)=-e^{-\theta}$ und daher 
$\left( \Psi'(\theta) \right)^2 = e^{-2\theta}$. Die Fischer-Information 
$I(\theta)=\frac{n}{\theta}$ wurde bereits in der Vorlesung berechnet. Damit erhalten wir 
\begin{equation}
    \frac{\left( \Psi'(\theta) \right)^2}{I(\theta)} = \frac{\theta e^{-2 \theta} }{n}.
\end{equation}

Wir zeigen nun
\begin{equation}
    \exp\left( \frac{\theta(1-2n)}{n} \right) - \exp\left( -2\theta \right) = \Var_\theta T(X)
    > \frac{\left( \Psi'(\theta) \right)^2}{I(\theta)} = \frac{\theta e^{-2 \theta} }{n}
\end{equation}
indem wir äquivalente Ungleichungen betrachten.
\begin{eqnarray}
    \exp\left( \frac{\theta(1-2n)}{n} \right) - \exp\left( -2\theta \right) 
    &>& \frac{\theta e^{-2 \theta} }{n} \\
    \exp\left( \frac{\theta}{n} -2\theta \right) &>& e^{-2\theta} \left( 1+\frac{\theta}{n} \right) \\
    \theta &>& \log \left( 1+\frac{\theta}{n} \right)^n.
\end{eqnarray}
Wir wissen aber, dass
\begin{equation}
    e^\theta = \lim_{n\to\infty} \left( 1+\frac{\theta}{n} \right)^n
\end{equation}
gilt, und dass die Approximationsfolge streng monoton wachsend ist. Damit ist die Ungleichung
für alle $\theta>0$ bewiesen. Daraus können wir ableiten, dass die Schranke in der 
Informationsungleichung nie angenommen wird.

\begin{figure}[htb]
    \begin{center}
        \includegraphics[width=\textwidth]{figures/ue-blatt-10-fvv.pdf}
    \end{center}
    \caption{Varianz von $T(X)$ und die Schranke der Informationsungleichung für $n=10$.}
    \label{fig:fvv}
\end{figure}





\section{Asymptotische Theorie}

\paragraph{Verschobene Gleichverteilung und Konsistenz.} 
Die Zufallsvariablen $X_1,\ldots,X_n$ seien i.i.d.\ mit $X_i\sim U(\theta,\theta+1)$.
Der Parameter $\theta$ sei unbekannt. Für die Schätzer $T_1(X) = \bar X -\frac{1}{2}$
und $T_2(X)=X_{(1)} - \frac{1}{n+1}$ haben wir bereits
\begin{eqnarray}
    \Var_\theta T_1(X) &=& \frac{1}{12 n} \\
    \Var_\theta T_2(X) &=& \frac{n}{\left( n+1 \right)^2\left( n+2 \right)}
\end{eqnarray}
gezeigt. Überprüfen Sie nun die beiden Schätzer auf schwache Konsistenz.

\paragraph*{Lösung.} Wir wissen bereits, dass die Schätzer $T_1$ und $T_2$ erwartungstreu sind 
sowie $E X_i = \theta + \frac{1}{2}$ gilt. Es ist leicht zu sehen, dass beide dieser Schätzer
die Bedingungen aus der \ref{Konsistenz-Hinreichende-Bedingungen}\ Aufgabe erfüllen. 
Es gilt für $i=1,2$
\begin{equation}
    E T_i^2 = \Var T_i + (E T_i)^2 <\infty
\end{equation}
sowie $E T_n \to \theta$ wegen erwartungstreue und $\Var T_i \to 0$. Somit sind beide dieser
Schätzer schwach konsistent.

Die Aussage kann man auch direkt  beweisen. Im Fall $T_1$ bietet sich die Anwendung
des schwachen Gesetzes der großen Zahl an. Die schwache Konvertenz von $T_2$ gegen $\theta$ 
folgt mit Hilfe der Markov-Ungleichung.





\paragraph{Konsistenz. Hinreichende Bedingungen. }\label{Konsistenz-Hinreichende-Bedingungen}
Seien $X_1,\ldots,X_n$ i.i.d.\ mit Verteilung $\mathbb P_\theta$ und $\theta\in\Theta\subset\R$.
Für jedes $n\in\mathbb N$ sei $T_n = T(X_1,\ldots,X_n)$ ein Schätzer für $\theta$ mit 
folgenden Eigenschaften:
\begin{enumerate}
    \item $E_\theta(T^2_n)< \infty$ für alle $\theta\in\Theta$ und alle $n\in\mathbb N$.
    \item $\lim_{n\to\infty}E_\theta(T_n) = \theta$ für alle $\theta\in\Theta$.
    \item $\lim_{n\to\infty} \Var_\theta T_n = 0$ für alle $\theta\in\Theta$.
\end{enumerate}
Dann ist der Schätzer $T_n$ schwach konsistent, d.h.\ $T_n \to^{\mathbb P_\theta} \theta$
für $n\to\infty$.

\paragraph*{Lösung.} Der Schätzer $T_n$ ist schwach konsistent wenn
\begin{equation}
    P\left( |T_n -\theta|\geq \varepsilon \right) \to 0
\end{equation}
wenn $n\to\infty$ für alle $\varepsilon>0$ und $\theta\in\Theta$. Wir schätzen die obige
Wahrscheinlichkeit mit Hilfe der Markov-Ungleichung
\begin{equation}
    P\left( |X|\geq \varepsilon \right) \leq \frac{E f(|X|)}{f(\varepsilon)}
\end{equation}
ab und erhalten für feste $n,\theta$ und $\varepsilon$
\begin{eqnarray}
    P\left( |T_n -\theta|\geq \varepsilon \right) &\leq& \frac{E \left( T_n -\theta \right)^2}{\varepsilon^2} \\
    &=& \frac{E \left( T_n^2 -2\theta T_n + \theta^2 \right) }{\varepsilon^2} \\
    &=& \varepsilon^{-2}\left( E T_n^2 - 2\theta E T_n + \theta^2 \right) \\
    &=& \varepsilon^{-2}\left( \Var T_n + \left( E T_n \right)^2 - 2\theta E T_n + \theta^2 \right).
\end{eqnarray}
Laut Voraussetzung ist $T_n$ quadratisch integrierbar, daher macht die obige Berechnung Sinn.
Für $n\to\infty$ erhalten wir
\begin{equation}
    \varepsilon^{-2}\left( \Var T_n + \left( E T_n \right)^2 - 2\theta E T_n + \theta^2 \right) \to 0
\end{equation}
und daraus
\begin{equation}
    P\left( |T_n -\theta|\geq \varepsilon \right) \to 0.
\end{equation}
Der Schätzer $T_n$ ist also schwach Konsistent.







\paragraph{Asymptotische Verteilung von MLE.   }   Seien $X_1,\ldots,X_n$ i.i.d.\ mit 
der Dichte $p(x,\theta)= \theta x^{\theta-1}1_{(0,1)}(x)$ für $\theta>0$. Finden Sie 
einen Maximum-Likelihood-Schätzer für $\theta$ und seine asymptotische Verteilung.

\paragraph*{Lösung.} Die Likelihoodfunktion der Stichprobe $x_1,\ldots,x_n$ ist
\begin{eqnarray}
    L(x,\theta) &=& \prod_i p(x_i, \theta) = \theta^n \prod_i x_i^{\theta-1}.
\end{eqnarray}
Daraus erhalten wir
\begin{eqnarray}
    l(x,\theta) &=& \log L(x,\theta) \\
    &=& n \log \theta + (\theta-1) \sum_{i}^{} \log x_i \\
    \frac{\partial}{\partial\theta} l(x,\theta) &=&  \frac{n}{\theta} + \sum_{i}^{} \log x_i \\
    \frac{\partial^2}{\partial \theta^2} l(x,\theta) &=& -\frac{n}{\theta^2} < 0 \quad \forall\theta>0.
\end{eqnarray}
Aus der Gleichung
\begin{equation}
    \frac{\partial}{\partial\theta} l(x,\theta) = 0
\end{equation}
erhalten wir den Maximum-Likelihood-Schätzer
\begin{equation}
    \hat \theta(X_1,\ldots,X_n) = - \frac{n}{\sum_{i}^{} \log X_i}.
\end{equation}

Wir berechnen die Fischer-Information pro Beobachtung. 
\begin{eqnarray}
    I_1(\theta) &=& E_\theta \left( \frac{\partial}{\partial \theta} \log p_\theta(X) \right)^2 \\
    &=& E_\theta \left( \frac{\partial}{\partial\theta} \log \theta + (\theta-1) \log X \right)^2 \\
    &=& E_\theta \left( \frac{1}{\theta} + \log X \right)^2 \\
    &=& E_\theta \left( \frac{1}{\theta^2} + \frac{2}{\theta} \log X + (\log X)^2 \right)
\end{eqnarray}
Es gilt nun
\begin{eqnarray}
    E_\theta \log X &=&  \int_{0}^{1} \log x \theta x^{\theta-1} dx = \int_{0}^{1} \log x (x^{\theta})' dx \\
    &=& x^\theta \log x \Big|_0^1 - \int_{0}^{1} \frac{x^{\theta-1}}{\theta} dx = - \frac{1}{\theta} \\
    E_\theta (\log X)^2 &=&  \int_{0}^{1} (\log x)^2 \theta x^{\theta-1} dx \\ 
    &=& x^\theta (\log x)^2 \Big|_0^1 - \int_{0}^{1} 2(\log x) x^{\theta-1} dx = \frac{2}{\theta^2} \\
\end{eqnarray}
Die Fischer-Information ist nun
\begin{eqnarray}
    I_1(\theta) &=& \frac{1}{\theta^2}
\end{eqnarray}
und die asymptotische Verteilung von $\hat \theta$ erfüllt
\begin{equation}
    \sqrt{n}\left( \hat\theta_n(X) - \theta_0 \right) \xrightarrow{\mathcal{L}} \mathcal N(0, \theta^2)
\end{equation}
falls die Bedingungen \textsc{(ar)} erfüllt sind. Das ist in der Tat der Fall denn der Parameterraum
$\Theta=\R_{>0}$ offen ist. Weiters gilt
\begin{eqnarray}
    \int_{0}^{1} \frac{\partial^2}{\partial\theta^2} p(x,\theta) dx &=& 
    \int_{0}^{1} \frac{\partial^2}{\partial\theta^2} \theta x^{\theta-1} dx \\
    &=&  \int_{0}^{1} \theta\,x^{\theta-1}\,\log ^2x+2\,x^{\theta-1}\,\log x dx\\
    &=& \left[   x^{\theta} \log^2x  \right]_0^1 = 0
\end{eqnarray}
und
\begin{eqnarray}
    A(\theta,x) &=& \frac{\partial^2}{\partial \theta^2} \log p(x,\theta) \\
    &=& \frac{\partial^2}{\partial \theta^2} \log \theta + (\theta-1) \log x \\
    &=& -\frac{1}{\theta ^2}. 
\end{eqnarray}
$A(\theta,x)$ ist von $x$ unabhängig und daher gleichmässig beschränkt in einer
hinreichend kleinen Umgebung von $\theta_0$ die $0$ nicht enthält. $K$ kann daher konstant 
gewählt werden. Die Fisher-Information ist positiv und aus $p(x,\theta)=p(x,\theta_0)$ f.s.\ in
$x\in\R$ folgt $\theta=\theta_0$ denn $p(x,\theta)= \theta x^{\theta-1}$ ein Polynom in $x$ ist. 
Damit sind die \textsc{(ar)} Regularitätsbedingungen erfüllt und obige Aussage über die
asymptotische Verteilung des \textsc{mle} gerechtfertigt.








\paragraph{Asymptotische Effizienz. } Seien $X_1,\ldots,X_n$ i.i.d.\ mit $E X_1=\mu\neq 0$,
$\Var X_1 = 1$ und $E X_1^4 < \infty$. Der Erwartungswert $\mu$ sei unbekannt. 
Ferner seien 
\begin{eqnarray}
    T_1(X) &=& \frac{1}{n} \sum_{i=1}^{n} \left( X_i^2 -1 \right) \\
    T_2(X) &=& {\bar X}^2 - \frac{1}{n}
\end{eqnarray}
zwei Schätzer für $\mu^2$. Zeigen Sie, dass $T_1$ und $T_2$ asymptotisch normalverteilt sind
und berechnen Sie deren asymptotische Erwartung und Varianz. Berechnen Sie 
die asymptotische Effizienz von $T_1$ zu $T_2$. Zeigen Sie, dass die asymptotische
Effizienz von $T_1$ zu $T_2$ nicht größer ist als $1$, falls Verteilung von $X_1 -\mu$
um $0$ symmetrisch ist.

\paragraph*{Lösung. } Betrachten wir zunächst den Schätzer $T_1$. Wir erhalten
\begin{eqnarray}
    E X_i^2 &=&  \Var X_i + \left( E X_i \right)^2 = 1 + \mu^2  \\
    \sqrt{n} \left( T_1 - \mu^2 \right) &=& 
    \sqrt{n}\left( \frac{1}{n} \sum_{i}^{} X_i^2 -1 -\mu^2 \right) \to_d \mathcal N(0, \Var X_i^2)
\end{eqnarray}
indem wir zentralen Grenzverteilungssatz auf Zufallsvariablen $X_i^2$ anwenden. $\Var X_i^2$ existiert
nach Voraussetzung.

Wir bestimmen die asymptotische Verteilung von $T_2$. Zentraler Grenzverteilungssatz liefert uns
\begin{equation}
    \sqrt{n}\left( \bar X - \mu \right) \to_d \mathcal N(0,1).
\end{equation}
Mit Hilfe der Delta-Methode und des Satzes von
\href{http://books.google.de/books?id=6anRT6Cz5GAC&lpg=PA101&dq=Slutsky's%20theorem&hl=de&pg=PA101#v=onepage&q&f=false}{Slutsky}
erhalten wir
\begin{equation}
    \sqrt{n}\left( T_2 - \mu^2 \right) =
    \sqrt{n}\left( {\bar X}^2 - \mu^2 \right) - \frac{1}{\sqrt{n}} \to_d 
    2\mu \mathcal N(0,1) = \mathcal N(0, 4\mu^2).
\end{equation}
%Im Falle $\mu=0$ ergibt sich aus $\sqrt{n}\left( \bar X \right) \to_d \mathcal N(0,1)$ 
%\begin{eqnarray}
%    n\left( T_2 - \mu^2 \right) &=&  n\left( {\bar X}^2 - \frac{1}{n} \right) \\
%    &=& n {\bar X}^2 -1 \to_d \mathcal N(0,1)^2 -1 = \chi_1^2 -1.
%\end{eqnarray}
%Sei $V$ eine $\chi_1^2$-verteilte Zufallsvariable. Es gilt
%\begin{eqnarray}
%    E \left( V-1 \right) &=& 0, \\
%    \Var \left( V-1 \right) &=& 2.
%\end{eqnarray}
Die asymptotische Effizienz von $T_1$ zu $T_2$ ist gegeben durch
\begin{equation}
    e(\theta, T_1, T_2) = \frac{4\mu^2}{\Var X_1^2}
%        \left\{
%            \begin{array}{ll}
%                \frac{4\mu^2}{\Var X_1^2}  & \mbox{if } \mu\neq 0 \\
%                \frac{2}{n \Var X_1^2} & \mbox{if } \mu=0
%            \end{array}
%        \right.
\end{equation}

Ist die Verteilung $X_1 - \mu$ symmetrisch um $0$, so ist die Schiefe der Verteilung
gleich $0$, d.h.\ $E \left( X_1 -\mu \right)^3=0$ und wir können mit Hilfe der
Jensenschen Ungleichung auf
\begin{equation}
    \Var X_1^2 \geq 4 \mu^2
\end{equation}
und $e\geq 1$ schliessen.






\section{Konfidenzintervalle}

\paragraph{Konfidenzintervalle. Normalverteilung. } Seien $X_1,\ldots,X_n$ i.i.d.\ mit 
$X_i\sim \mathcal N(\mu,\sigma^2)$, wobei sowohl $\mu$ als auch $\sigma$ unbekannt seien.
Zeigen Sie, dass
\begin{equation}
    \left[ \frac{n-1}{\chi^2_{n-1,1-\alpha/2}}s^2(X), \frac{n-1}{\chi^2_{n-1,\alpha/2}}s^2(X) \right]
    \label{}
\end{equation}
win $\left( 1-\alpha \right)$-Konfidenzintervall für $\sigma^2$ ist, wobei
$s^2(X)$ die Stichprobenvarianz und $\chi^2_{n,a}$ das $a$-Quantil der $\chi^2_n$-Verteilung bezeichnen.

\paragraph*{Lösung.} Die Verteilung der Stichprobenvarianz ist bereits bekannt.
\begin{equation}
    \frac{(n-1) s^2(X)}{\sigma^2} = \sum_{i}^{} \left( \frac{ X_i -\bar X}{\sigma} \right)^2
    \sim \chi^2_{n-1}
\end{equation}
Das Konfidenzintervall erhalten wir aus
\begin{eqnarray}
    P \left( \frac{n-1}{\chi^2_{n-1,1-\frac{\alpha}{2}} } s^2(X) 
    \leq \sigma^2 \leq 
    \frac{n-1}{ \chi^2_{n-1,\frac{\alpha}{2}} } s^2(X)
    \right) &=& \\
    P \left( \frac{n-1}{\sigma^2} s^{2}(X) \leq \chi^2_{n-1,1-\frac{\alpha}{2}} \wedge
    \chi^2_{n-1,\frac{\alpha}{2}} \leq \frac{n-1}{\sigma^2} s^2(X)
    \right) &=& 1-\alpha.
\end{eqnarray}








\paragraph{Konfidenzintervalle. Exponentialverteilung.  }
Seien $X_1,\ldots,X_n$ i.i.d.\ exponentialverteilt mit der Dichte
\begin{equation}
    p_\theta(x) = \theta e^{- \theta x} 1_{ \R_{>0} }(x).
\end{equation}
Zeigen Sie, dass 
\begin{equation}
    \left[  \frac{- \log \left( 1 - \frac{\alpha}{2} \right)}{n X_{(1)}}, 
    \frac{- \log \left( \frac{\alpha}{2} \right)}{n X_{(1)}} \right]
\end{equation}
ein $(1-\alpha)$-Konfidenzintervall für $\theta$ ist.

\paragraph*{Lösung.} Zu zeigen ist folgende Gleichung, die das 
Konfidenzintervall charakterisiert:
\begin{equation}
    P\left(   \frac{- \log \left( 1 - \frac{\alpha}{2} \right)}{n X_{(1)}} \leq \theta \leq 
    \frac{- \log \left( \frac{\alpha}{2} \right)}{n X_{(1)}} \right) = 1-\alpha.
\end{equation}
Betrachten wir die Wahrscheinlichkeiten
\begin{eqnarray}
    P \left( n\theta X_{(1)}\geq x \right) &=&  \prod_i P \left( n\theta X_i \geq x \right) \\
    &=& \prod_i P\left( X_i \geq \frac{x}{n\theta} \right) \\
    &=& \prod_i e^{-\frac{x}{n}} = e^{-x}.
\end{eqnarray}
Die Gleichung
\begin{equation}
    P \left( c_1(\alpha) \leq n\theta X_{(1)} \leq c_2(\alpha) \right) = 
    e^{-c_1(\alpha)} - e^{-c_2(\alpha)} = 1-\alpha
\end{equation}
ist unter anderem für $c_1 = -\log \left( 1-\frac{\alpha}{2} \right)$ und
$c_2 = -\log \left( \frac{\alpha}{2} \right)$ erfüllt.





\paragraph{Konfidenzintervalle. Varianzenvergleich bei Normalverteilung. } 
Seien $X_1,\ldots,X_n,Y_1,\ldots,Y_n$ unabhängig und normalverteilt, mit 
$X_i \sim \mathcal N(0, \sigma^2_X)$, $Y_i \sim \mathcal N(0, \sigma^2_Y)$
und unbekannten Parametern $\sigma^2_X>0$ und $\sigma^2_Y>0$. 
Zeigen Sie, dass für $S^2_X = \sum_{i=1}^{n} X_i^2$ und $S^2_Y = \sum_{i=1}^{n} Y_i^2$
\begin{equation}
    \left[ F^{-1}_{n,n} \left( \frac{\alpha}{2} \right) \frac{S_Y^2}{S_X^2},
    F^{-1}_{n,n} \left( 1 - \frac{\alpha}{2}  \right) \frac{S_Y^2}{S_X^2}     \right]
\end{equation}
ein $(1-\alpha)$-Konfidenzintervall für den Quotienten $\frac{\sigma^2_Y}{\sigma^2_X}$ ist.
$F_{n,n}$ bezeichnet dabei die Verteilungsfunktion der $F_{n,n}$-Verteilung.

\paragraph*{Lösung.} Zu zeigen ist die Gleichung
\begin{equation}
    P \left[ F^{-1}_{n,n} \left( \frac{\alpha}{2} \right) \frac{S_Y^2}{S_X^2} \leq 
    \frac{\sigma^2_Y}{\sigma^2_X} \leq
    F^{-1}_{n,n} \left( 1 - \frac{\alpha}{2}  \right) \frac{S_Y^2}{S_X^2}     \right]
    = 1-\alpha.
\end{equation}
Es gilt aber 
\begin{equation}
    \frac{\sigma^{-2}_X S^2_X  }{ \sigma^{-2}_Y S^2_Y} = 
    \frac{\sum_{i}^{} \left( \frac{X_i}{\sigma_X} \right)^2  }{ \sum_{i}^{} \left( \frac{Y_i}{\sigma_Y} \right)^2 } \sim F_{n,n}.
\end{equation}


\section{Testtheorie}

\paragraph{Tests. Mittelwertvergleich unter Normalverteilung. } 
Seien die Zufallsvariablen $X_1,\ldots,X_n$, $Y_1,\ldots,Y_n$ unabhängig und normalverteit, 
mit $X_i\sim\mathcal N(\mu_X, \sigma^2_X)$ und $Y_i\sim\mathcal N(\mu_Y, \sigma^2_Y)$.
Die Mittelwerte $\mu_X$ und $\mu_Y$ sind dabei unbekannt, die Varianzen dagegen
$\sigma_X^2>0$, $\sigma^2_Y>0$ bekannt und fix.
\begin{enumerate}
    \item Zeigen Sie, dass 
        \begin{equation}
            \bar X - \bar Y \pm \sqrt{ \frac{\sigma^2_X + \sigma^2_Y}{n}  }z_{1-\frac{\alpha}{2}}
        \end{equation}
        ein $(1-\alpha)$-Konfidenzintervall für die Differenz der Mittelwerte $\mu_X - \mu_Y$ ist.
    \item Konstruieren Sie einen Test zu dem Signifikanzniveau von $5\%$ für die Hypothese
        $H_0: \mu_X = \mu_Y$ gegen die Alternative $H_1: \mu_X \neq \mu_Y$.
    \item Drücken Sie die Gütefunktion zu dem Test aus Teil (2) in Abhängigkeit von
        $\Delta = \mu_X -\mu_Y$ aus und skizzieren Sie die Gütefunktion. 
\end{enumerate}

\paragraph*{Lösung.} Wir betrachten
\begin{equation}
    P\left( \bar X - \bar Y - \sqrt{ \frac{\sigma^2_X + \sigma^2_Y}{n}  }z_{1-\frac{\alpha}{2}} \leq \mu_X - \mu_Y \leq \bar X - \bar Y + \sqrt{ \frac{\sigma^2_X + \sigma^2_Y}{n}  }z_{1-\frac{\alpha}{2}}   \right) = 1-\alpha.
\end{equation}
Und daraus
\begin{eqnarray}
    P \left(  - \sqrt{ \frac{\sigma^2_X + \sigma^2_Y}{n}  }z_{1-\frac{\alpha}{2}} \leq
    \bar Y -\mu_Y -\left( \bar X - \mu_X \right) \leq 
    \sqrt{ \frac{\sigma^2_X + \sigma^2_Y}{n}  }z_{1-\frac{\alpha}{2}}
    \right) &=& \\ 
    P \left(  - z_{1-\frac{\alpha}{2}} \leq
    \frac{ \bar Y -\mu_Y -\left( \bar X - \mu_X \right)  }{  \sqrt{ \frac{\sigma^2_X + \sigma^2_Y}{n}  } } 
    \leq  z_{1-\frac{\alpha}{2}} 
    \right).
\end{eqnarray}
Mit
\begin{equation}
    \bar Y - \mu_Y - \left( \bar X - \mu_X \right) \sim 
    \mathcal N\left(0, \frac{\sigma^2_X}{n} + \frac{\sigma^2_Y}{n}   \right)
\end{equation}
erhalten wir die erste Gleichung.  
Wir setzen
\begin{eqnarray}
    \theta_1 &=& \bar X - \bar Y - \sqrt{ \frac{\sigma^2_X + \sigma^2_Y}{n}  }z_{1-\frac{\alpha}{2}} \\
    \theta_2 &=& \bar X - \bar Y + \sqrt{ \frac{\sigma^2_X + \sigma^2_Y}{n}  }z_{1-\frac{\alpha}{2}} 
\end{eqnarray}
und konstruieren den Test
\begin{equation}
    \delta\left( X,0 \right) = 1_{ \left\{ 0 \nin \left[ \theta_1,\theta_2 \right] \right\} }.
\end{equation}
Mit $\alpha=0.05$ ist $\delta(X,0)$ der gesuchte Test mit Signifikanzniveau $5\%$.

Berechnen wir die Gütefunktion des Tests $\delta\left( X,0 \right)$. Mit $\theta=\left( \mu_X,\mu_Y \right)$
und $\sigma= \sqrt{\frac{\sigma_X^2 + \sigma_Y^2}{n}}$ erhalten wir
\begin{eqnarray}
    G_\delta(\theta) &=&  E_\theta \left( \delta \left( X,0) \right)  \right) \\
    &=& P_\theta \left( \delta(X,0) = 1 \right) \\
    &=& P_\theta \left( 0 \leq \bar X - \bar Y - \sigma z_{1-\frac{\alpha}{2}} \vee
        \bar X - \bar Y + \sigma z_{1-\frac{\alpha}{2}} \leq 0\right) \\
    &=& P_\theta \left( \frac{\bar X - \bar Y}{\sigma} \geq z_{1-\frac{\alpha}{2}} \vee
    \frac{\bar X - \bar Y}{\sigma} \leq z_{\frac{\alpha}{2}}\right).
\end{eqnarray}
Sei nun $Z\sim\mathcal N(0,1)$ und $\Delta=\mu_X - \mu_Y$. 
Da $\frac{\bar X -\bar Y}{\sigma}\sim \mathcal N(\mu_X-\mu_Y,1)$ können wir die Gütefunktion des 
Test folgendermassen notieren
\begin{eqnarray}
    G_\delta (\Delta) &=& P \left( Z \leq z_{\frac{\alpha}{2}} -\Delta \vee
    Z \geq z_{1-\frac{\alpha}{2}} -\Delta \right)  \\
    &=& \Phi\left( z_{\frac{\alpha}{2}} - \Delta\right) 
    + 1 - \Phi\left( z_{1-\frac{\alpha}{2}} -\Delta \right). 
\end{eqnarray}


\begin{figure}[htb]
    \begin{center}
        \includegraphics[width=\textwidth]{figures/ue-blatt-12-power.pdf}
    \end{center}
    \caption{Die Gütefunktion des Tests $\delta(X,0)$ mit Signifikanzniveau $\alpha=0.05$.}
    \label{fig:power}
\end{figure}


\begin{lstlisting}[language=R,caption={R Code zum Plotten der Gütefunktion}]
require(ggplot2)
alpha=.05
X=seq(-6,6,0.05)
g <- function(delta) {
    return( pnorm(qnorm(alpha/2) - delta ) +1 -pnorm(qnorm(1-alpha/2)-delta))  
}
qplot(X,g(X), geom="line", xlab="Delta", ylab="Guete")
\end{lstlisting}







\paragraph{Test. Poisson-Verteilung. }
Seien $X_1,\ldots,X_n$ i.i.d.\ Poisson-verteilt mit unbekanntem Parameter $\lambda>0$.
\begin{enumerate}
    \item Verwenden Sie die natürliche suffiziente Statistik, um einen Test mit Signifikanzniveau
         $\alpha$ für die Hypothese $H_0: \lambda\leq\lambda_0$ gegen die Alternative
         $H_1: \lambda>\lambda_0$ zu finden. Konstruieren Sie dazu zunächst einen Test für die
         Hypothese $\lambda=\lambda_0$ und zeigen Sie, dass die Gütefunktion streng monoton 
         wachsend in $\lambda$ ist. Benutzen Sie den zentralen Grenzwertsatz, um eine 
         Approximation für den kritischen Wert zu finden.
     \item Seien $\alpha=0.05$, $n=200$, $\sum_{i=1}^{200} X_i= 2085$ und $\lambda_0=10$. Klären
         Sie, ob die Hypothese $H_0: \lambda\leq\lambda_0$ verworfen wird und bestimmen Sie den 
         $p$-Wert.
\end{enumerate}

\paragraph*{Lösung.}
Wir konstruieren zunächst einen Test für 
\begin{equation}
    H_0 : \lambda=\lambda_0 \quad \textrm{gegen} \quad H_1 : \lambda>\lambda_0
\end{equation}
und werden anschließend zeigen, dass $H_0: \lambda=\lambda_0$ durch
$H_0: \lambda<\lambda_0$ ersetzt werden kann.

Die Wahrscheinlichkeitsfunktion der Poisson-Verteilung ist
\begin{eqnarray}
    p_\lambda(x) &=& P\left( X_i=x \right) = e^{-\lambda} \frac{\lambda^x}{x!} \\
    &=& \exp \left( x \log \lambda -\lambda - \log x! \right),
\end{eqnarray}
woraus wir die natürliche suffiziente Statistik $T(X)= \sum_{i}^{} X_i$ ablesen können.
Betrachten wir also den Test
\begin{equation}
    \delta_j (X) = 1_{ \left\{ T(X) >j \right\}  } \left( X \right)
\end{equation}
für $j>0$. Die Güte dieses Tests ist 
\begin{eqnarray}
    G_{\delta_j} (\lambda) &=&  E_\lambda \delta(X) = P_\lambda \left( \delta(X)=1 \right) \\
    &=& P_\lambda \left( T(X) > j \right) =  1-P_\lambda \left( T(X) \leq j \right) \\
    &=& 1 - \sum_{i=0}^{j} e^{-\lambda n} \frac{\left( \lambda n \right)^i}{i!}, \\
\end{eqnarray}
da $T(X)\sim$ Poiss($n\lambda$). Wir legen $j=j(\alpha)$ so fest, dass der Test $\delta_j$ 
Signifikanzniveau $\alpha$ hat, d.h.\ 
\begin{equation}
    G_\delta (\lambda_0) = 1 - \sum_{i=0}^{j} e^{-\lambda n} \frac{\left( \lambda n \right)^i}{i!} \leq \alpha.
\end{equation}
Die Gütefunktion ist streng monoton wachsend, denn die Ableitung nach $\lambda$ 
\begin{eqnarray}
    \frac{\partial}{\partial \lambda} G_\delta (\lambda) &=& 
    n e^{-\lambda n} \sum_{i=0}^{j} \frac{\left( \lambda n \right)^i}{i!} 
    - n e^{ -\lambda n } \sum_{i=1}^{j} \frac{ i \left( \lambda n  \right)^{i-1} }{i!} \\
    &=& e^{-\lambda n }n \left[ \sum_{i=0}^{j} \frac{\left( \lambda n \right)^i}{i!}   
    - \sum_{i=0}^{j-1} \frac{ \left( \lambda n  \right)^{i} }{i!} \right] > 0 
\end{eqnarray}
ist positiv, solange $\lambda>0$ gilt. Die Hypothese $H_0 : \lambda=\lambda_0$ kann also 
durch $H_0 : \lambda\leq \lambda_0$ ersetzt werden, ohne dass Anpassung des Signifikanzniveaus 
notwendig ist.

Berechnen wir nun eine Approximation für $j=j(\alpha)$ vom zentralen Grenzwertsatz ausgehend.
\begin{eqnarray}
    P_{\lambda_0} \left( T(X) > j \right) &=& P_{\lambda_0} \left( \sum_{i}^{} X_i > j \right) \\
    &=& P_{\lambda_0} \left( \sum_{i}^{} X_i -\lambda_0 > j - \lambda_0 \right) \\
    &=& P_{\lambda_0} \left( \frac{1}{\sqrt{n}} \sum_{i}^{} \frac{ X_i - \lambda_0}{ \sqrt{\lambda_0}} > \frac{j-n\lambda_0}{\sqrt{\lambda_0}}\frac{1}{\sqrt{n}} \right) \\
    &\simeq& 1-\Phi \left( \frac{j-n\lambda_0}{\sqrt{\lambda_0}}\frac{1}{\sqrt{n}}  \right). 
\end{eqnarray}
Aus
\begin{equation}
    1-\Phi \left( \frac{j-n\lambda_0}{\sqrt{\lambda_0 n}} \right) = \alpha
\end{equation}
erhalten wir
\begin{eqnarray}
    \frac{j-n\lambda_0}{\sqrt{\lambda_0 n}} &=&  z_{1-\alpha} \\
    j &=&  \sqrt{\lambda_0 n} z_{1-\alpha} + n\lambda_0.
\end{eqnarray}
Für $\alpha=0.05$, $n=200$, $\sum_{i}^{}X_i=2085$ und $\lambda_0=10$ ist $j\simeq 2073.56$,
und somit wird in diesem Fall die Nullhypothese nicht verworfen.

% j <- function(alpha, n, sumXi, lambda) { return(  sqrt(lambda*n)*qnorm(1-alpha) + n*lambda    )  } 
% j(0.05,200,2085,10) 








\paragraph{Test. Exponentialverteilung. } Seien $X_1,\ldots,X_n$ i.i.d.\ und exponentialverteilt
zum Parameter $\theta$. Der Mittelwert werde mit $\mu=\theta^{-1}$ bezeichnet.
Man interessiert sich für den Test $H_0 : \mu \leq \mu_0$ gegen die Alternative $H_1 : \mu > \mu_0$.
\begin{enumerate}
    \item Sei $c_{1-\alpha}$ das $(1-\alpha)$-Quantil der $\chi^2_{2n}$-Verteilung. Zeigen Sie, dass
        ein Test mit Verwerfungsbereich
        \begin{equation}
            \left\{ \bar X \geq \frac{\mu_0 c_{1-\alpha}}{2n} \right\}
        \end{equation}
        ein Test mit Signifikanzniveau $\alpha$ ist.
    \item Bestimmen Sie die Güte des Tests aus Punkt 1 an der Stelle $\mu$.
    \item Zeigen Sie, dass 
        \begin{equation}
            \Phi\left( \frac{\mu_0 z_\alpha}{\mu} + \sqrt{n} \frac{\mu-\mu_0}{\mu} \right) 
        \end{equation}
        eine Approximation der Güte des Tests aus Punkt 1 an der Stelle $\mu$ ist, wobei $\Phi$ die
        Verteilungsfunktion und $z_\alpha$ das $\alpha$-Quantil der Standardnormalverteilung bezeichnen.
    \item Gegeben sei folgende Stichprobe:
        \begin{equation}
            3,150,40,34,32,37,34,2,31,6,5,14,150,27,4,6,27,10,30,27.
        \end{equation}
        Berechnen Sie den $p$-Wert zum Test aus Punkt 1 und interpretieren Sie diesen für gegebenes $\mu_0=25$.
\end{enumerate}

\paragraph*{Lösung. } Die Verteilungen in dieser Aufgabe gehören zur Gamma-Familie, denn
$\chi^2_{2n}=$ Gamma($n,\frac{1}{2}$) und Exp$(\theta) =$ Gamma($1,\theta$). Die Dichte von $\chi^2_n$ ist ja 
\begin{equation}
    p_{\chi^2_n} (x) = 
    \frac{1}{2^{\frac{n}{2}} \Gamma\left( \frac{n}{2} \right)} x^{ \frac{n}{2} -1}e^{-\frac{x}{2}} 1_{\R_{>0}}(x).
\end{equation}
Das Signifikanzniveau des Tests ist also
\begin{eqnarray}
    P_{\mu_0} \left( \bar X \geq \frac{ \mu_0 c_{1-\alpha} }{2n} \right)  
    &=& P_{\mu_0} \left( \frac{2}{\mu_0} \sum_{i}^{} X_i  > c_{1-\alpha}   \right) \\
    &=& P_{\mu_0} \left( 2 \theta_0 \sum_{i}^{} X_i > c_{1-\alpha} \right) = \alpha
\end{eqnarray}
denn 
\begin{eqnarray}
    X_i & \sim & \textrm{Exp}(\theta) \\
    \sum_{i}^{} X_i & \sim & \textrm{Gamma}(n,\theta) \\
    2\theta \sum_{i}^{} X_i & \sim & \textrm{Gamma}(n,\frac{1}{2}).
\end{eqnarray}

Die Güte des Tests ist
\begin{eqnarray}
    G_\delta (\theta) &=&  E_\theta \delta(X) \\
    &=& P_\theta \left( \bar X  \geq \frac{\mu_0 c_{1-\alpha}}{2n} \right) \\
    &=& P_\theta \left( 2 \sum_{i}^{} X_i \geq \frac{c_{1-\alpha}}{\theta_0} \right).
\end{eqnarray}
Nachdem $2 \sum_{i} X_i \sim $ Gamma($n, \frac{\theta}{2}$) gilt, können wir 
\begin{equation}
    Z = 2 \theta \sum_{i}^{} X_i \sim \textrm{Gamma}(n,\frac{1}{2})
\end{equation}
setzen und erhalten
\begin{equation}
    G_\delta(\theta) = P \left( Z \geq \frac{\theta}{\theta_0} c_{1-\alpha} \right).
\end{equation}

Um die Güte des Tests zu approximieren, entwickeln wir zunächst eine approximative 
Version des Tests selbst. Das hat den Vorteil, dass hier das Signifikanzniveau 
genauer kontrolliert werden kann. Wir berechnen zunächst mit Hilfe der Gleichung
\begin{eqnarray}
    P_{\mu_0} \left( \sum_{i}^{} X_i \geq c \right) \leq \alpha
\end{eqnarray}
den approximativen Verwerfungsbereich $\left[ c,+\infty \right]$.
\begin{eqnarray}
    P_{\mu_0} \left( \frac{1}{\sqrt{n}} \sum_{i}^{} \frac{X_i - \mu_0}{\mu_0} \geq \frac{1}{\sqrt{n}} 
    \frac{c -n\mu_0}{ \mu_0}\right) &\leq& \alpha \\
    \Phi \left( \frac{1}{\sqrt{n} } \frac{n \mu_0 - c}{\mu_0}  \right) &=&  \alpha \\
    n \mu_0 - \mu_0 \sqrt{n} z_\alpha &=&  c
\end{eqnarray}
Daraus errechnet sich wie folgt die approximative Güte des Tests.
\begin{eqnarray}
    P_\mu \left( \frac{1}{\sqrt{n}} \sum_{i}^{} \frac{X_i - \mu}{\mu} \geq \frac{1}{\sqrt{n}} \frac{c-n \mu}{ \mu}   \right) &=&  \Phi \left( \frac{1}{\sqrt{n}} \frac{n \mu - c}{ \mu} \right) \\
    &=& \Phi \left( \frac{1}{\sqrt{n}} \frac{n \mu - \left( n \mu_0 -\mu_0 \sqrt{n} z_\alpha \right)}{\mu} \right) \\
    &=& \Phi \left( \sqrt{n} \frac{\mu - \mu_0}{\mu} + z_\alpha \frac{\mu_0}{\mu} \right).
\end{eqnarray}

Für eine feste Beobachtung $X$ ist der $p$-Wert des Tests das kleinste Signifikanzniveau, 
bei welchem der Test die Nullhypothese $H_0$ verwirft. Für die Berechnung des $p$-Werts
benutzen wir die approximative Version des Tests. Dieser Test verwirft $H_0$, wenn
$\sum_{i}^{} X_i \geq c = n \mu_0 - \mu_0 \sqrt{n} z_\alpha$. Das ist äquivalent zu
\begin{eqnarray}
    \sum_{i}^{} X_i &\geq & n \mu_0 - \mu_0 \sqrt{n} z_\alpha \\ 
    \Phi \left(  - \frac{ \sum_{i}^{} X_i -n \mu_0   }{\sqrt{n} \mu_0} \right) &\leq& \alpha \\
\end{eqnarray}
Das kleinste $\alpha$, das diese Ungleichung erfüllt, ist der $p$-Wert
\begin{equation}
    \Phi \left(  - \frac{ \sum_{i}^{} X_i -n \mu_0   }{\sqrt{n} \mu_0} \right).
\end{equation}
Für die oben angegebene Stichprobe und $\mu_0=25$ ist der $p$-Wert gleich $0.0653$, was als
Evidenz gegen die Nullhypothese interpretiert werden kann.

% X=c(3,150,40,34,32,37,34,2,31,6,5,14,150,27,4,6,27,10,30,27)
% n=length(X); mu_0=25
% pnorm((n*mu_0 - sum(X))/(sqrt(n)*mu_0) )







\paragraph{UMP-Test. Rayleigh-Verteilung. } Seien $X_1,\ldots,X_n$ i.i.d.\ und Rayleigh-verteilt
zum unbekannten Parameter $\theta>0$, d.h.\ $X_1$ hat die Dichte 
\begin{equation}
    p_\theta(x) = \frac{x}{\theta^2} e^{- \frac{x^2}{2\theta^2}} 1_{\R_{>0}}(x).
\end{equation}
\begin{enumerate}
    \item Finden Sie eine optimale Teststatistik $T_n$ für
        \begin{equation}
            H_0 : \theta\leq 1 \quad \textrm{gegen} \quad H_1 : \theta >1.
        \end{equation}
    \item Konstruieren Sie unter Benutzung von $T_n$ einen \textsc{ump}-Test mit Signifikanzniveau 
        $\alpha$, wobei der kritische Wert $c$ approximativ mit Hilfe des zentralen Grenzwertsatzes
        bestimmt werden soll.
\end{enumerate}

\paragraph*{Lösung. } Der Likelihoodquotient für das Testproblem $H_0: \theta=1$ gegen $H_1 : \theta=\nu$
\begin{eqnarray}
    L(x,1,\nu) &=& 
    \frac{\prod_i \frac{x_i}{\nu^2} \exp \left( -\frac{x_i^2}{2 \nu^2} \right)}{ \prod_i x_i \exp \left( - \frac{x_i^2}{2} \right)  } \\
    &=& \frac{1}{ \nu^{2n}} \exp\left( \left( \frac{1}{2} -\frac{1}{2 \nu^2} \right) \sum_{i}^{} x_i^2 \right)\\
\end{eqnarray}
ist eine optimale Teststatistik. $T(X)=f(L(x,1,\nu))= \sum_{i}^{} X_i^2$ 
ist auch eine optimale Teststatistik, denn
\begin{equation}
    f(\eta) = \frac{ \log \left( \nu^{2n} \eta \right) }{\frac{1}{2} - \frac{1}{2\nu^2}}
\end{equation}
ist monoton wachsend. Der Test
\begin{equation}
    \delta_k(X) = 1_{  \left\{ \sum_{i}^{}X_i^2 \geq k \right\} }
\end{equation}
ist ein \textsc{np}-Test für $H_0: \theta=1$ gegen $H_1 : \theta= \nu$ für alle $\nu>1$.
Daher ist $\delta_k$ auch ein \textsc{ump}-Test für das Testproblem $H_0: \theta=1$ gegen $H_1 : \theta>1$.
Um dieses Resultat auf das Testproblem $H_0: \theta \leq 1$ gegen $H_1 : \theta>1$ zu verallgemeinern, zeigen
wir, dass $G_{\delta_k}(1)\geq G_{\delta_k} (\theta)$ für alle $\theta \in (0,1)$ gilt und somit wird das
Signifikanzniveau auf der nun erweiterten Menge $\Theta_0 = \left\{ \theta\leq 1 \right\}$ gehalten.

Für ein positives $k$ ist 
\begin{eqnarray}
    P \left( X_i^2 \geq k \right) &=& P \left( X \geq \sqrt{k}  \right) \\
    &=& \int_{\sqrt{k}}^{\infty} \frac{x}{\theta^2} e^{ - \frac{x^2}{ 2 \theta^2}} d x \\
    &=& \int_{\frac{k}{\theta^2}}^{\infty} e^{-z} dz
\end{eqnarray}
monoton fallend, also gilt $P_\theta \left( X_i^2 \geq k \right) \leq \alpha$ für $\theta\in (0,1)$,
wenn $P_1 \left( X_i^2 \geq k \right)\leq \alpha$ gilt.

Konstruieren wir nun mit Hilfe des zentralen Grenzwertsatzes einen zum Signifikanzniveau 
$\alpha$ gehörigen Verwerfungsbereich $\left\{ \theta>k \right\}$. Zunächst berechnen wir
\begin{eqnarray}
    E X^{2n} &=&  \int_{0}^{\infty} x^{2n} \frac{x}{\theta^2} \exp \left( - \frac{x^2}{2\theta^2} \right) dx \\
    &=&  \int_{0}^{\infty} \left( 2 \theta^2 z \right)^n e^{-z} dz \\
    &=& 2^n \theta^{2n} \int_{0}^{\infty} z^{(n+1)-1}e^{-z} dz  \\
    &=& 2^n \theta^{2n} \Gamma(n+1) = 2^n \theta^{2n} n!
\end{eqnarray}
und daraus $E X^2 = 2 \theta^2$, $E X^4 = 8 \theta^4$, $\Var X^2 = 4 \theta^4$ sowie
\begin{eqnarray}
    P \left( \sum_{i} X_i^2 \geq k \right)&=&  
    P \left( \sum_{i}^{} \frac{X_i^2 - \mu}{\sqrt{n} \sigma} \geq \frac{k -n \mu}{\sqrt{n} \sigma} \right) \\
    &=& P \left( \sum_{i}^{} \frac{X_i^2 - 2}{2\sqrt{n}} \geq \frac{k-2n}{2\sqrt{n}} \right) \\
    &\simeq & \Phi \left( \frac{2n -k}{2 \sqrt{n}} \right). 
\end{eqnarray}
Schließlich erhalten wir aus $P \left( T(X) \geq k \right) \leq \alpha$ die Approximation
\begin{eqnarray}
    \Phi\left( \frac{2n - k}{ 2\sqrt{n}}  \right) &\leq & \alpha \\
    k &=& 2n - 2 \sqrt{n} z_\alpha.
\end{eqnarray}





\paragraph{Optimaler Test. Pareto-Verteilung. }  Eine Zufallsvariable heißt Pareto-Verteilt
zu den Parametern $k>0,a>0$, falls sie die Dichte 
\begin{equation}
    p(x) = a k^a x^{-a-1} 1_{\R_{>k}}(x)
\end{equation}
besitzt. $X_1,\ldots,X_n$ seien i.i.d.\ Pareto$(k,a)$-verteilt, wobei $k=1$ und $a$ unbekannt sei.
Zeigen Sie, dass $T(X_1,\ldots,X_n)= \sum_{i=1}^{n} \log\left( X_i \right)$ eine optimale 
Teststatistik für
\begin{equation}
    H_0 : \frac{1}{a} \leq \frac{1}{a_0} \quad \textrm{gegen} \quad H_1 : \frac{1}{a} > \frac{1}{a_0}
\end{equation}
ist.

\paragraph*{Lösung. } Substituieren wir zunächst $\tilde a = \frac{1}{a}$. Das Testproblem lässt sich
nun schreiben als $H_0: \tilde a \leq \tilde a_0$ gegen $H_1: \tilde a > \tilde a_0$. Die gemeinsame
Dichte der Zufallsvariablen $X_1,\ldots,X_n$ kann in exponentieller Form notiert werden
\begin{eqnarray}
    p_{\tilde a} &=& \prod_i \frac{1}{\tilde a} x_i^{ -\frac{1}{\tilde a} -1 } 1_{\R_{>1}}(x_i) \\
    &=& \exp \left( \left( -\frac{1}{\tilde a} -1 \right) \sum_{i}^{}\log x_i
    -n \log \tilde a \right) 1_{\R_{>1}^n}(x_1,\ldots,x_n), \\
\end{eqnarray}
woraus wir $c(\tilde a)=-\frac{1}{\tilde a} -1$ und $T(X)= \sum_{i}^{} \log X_i$ ablesen können.
Die Funktion $c(\tilde a)$ ist streng monoton wachsend, denn $c'(\tilde a ) >0$ für alle $\tilde a>0$. 
Die Statistik $T(X)$ ist daher optimal für das Testproblem. 







\paragraph{KQS ist auch MLS im Normalverteilungsfall. }
Zeigen Sie, dass der Kleinste-Quadrate-Schätzer $\hat \beta$ auch Maximum-Likelihood-Schätzer 
im allgemeinen linearen Modell ist, falls $\varepsilon \sim \mathcal N(0, \sigma^2 I_n)$.

\paragraph*{Lösung. } Betrachten wir ein allgemeines lineares Model
\begin{equation}
    Y_i =  X_i \beta+ \varepsilon_i, \quad i=1,\ldots,n.
\end{equation}
Dabei ist $\beta$ der $p$-dimensionale Vektor der Modelparameter, den wir schätzen wollen,
$X_i$ sind Zeilen einer bekannten Matrix $X\in\R^{n\times p}$ und 
$\left( \varepsilon_1, \ldots, \varepsilon_n \right)$ sind i.i.d. mit 
$\varepsilon_i \in\mathcal N(0,\sigma^2)$, $\sigma>0$. Die Beobachtung
$Y=\left( Y_1,\ldots,Y_n \right)$ ist demnach multivariat normalverteilt, d.h.\ 
$Y\sim\mathcal N(X\beta,\sigma^2 I_n)$. $Y_i$ sind i.i.d.\ mit
$Y_i\sim\mathcal N(X_i \beta, \sigma^2 )$ und die Likelihoodfunktion der Beobachtung 
$y=\left( y_1,\ldots, y_n \right)$ ist 
\begin{eqnarray}
    L(Y, \beta) &=&  \prod_i \frac{1}{ \sqrt{2\pi} \sigma} 
    \exp\left( - \frac{1}{2} \left( \frac{y_i - X_i \beta }{\sigma}  \right)^2 \right) \\
    &=& \left( \frac{1}{\sqrt{2\pi}\sigma} \right)^n 
    \exp \left( -\frac{1}{2 \sigma^2} \sum_{i}^{} \left( y_i - X_i\beta \right)^2 \right).
\end{eqnarray}
Bis auf die positive multiplikative Konstante $\left( \sqrt{2 \pi}\sigma \right)^{-n}$ ist 
die Log-Likelihood-Funktion gleich
\begin{eqnarray}
    l(Y, \beta) &=& -\frac{1}{2 \sigma^2}  \sum_{i}^{} \left( y_i - X_i \beta \right)^2 
\end{eqnarray}
Die Maximierung von $l(Y, \beta)$ ist gleichbedeutend mit der Minimierung von 
\begin{equation}
    \sum_{i}^{} \left( y_i - X_i \beta \right)^2 = \| y - X \beta \|^2, 
\end{equation}
was genau die Eigenschaft des Kleinste-Quadrate-Schätzers ist. 

Beachten Sie, dass wir in den obigen Überlegungen nicht
$\textrm{Rang}\left( X \right)=p$ annehmen mussten.





\section{Lineare Regression}

\paragraph{Einfache lineare Regression. } Seien folgende Beobachtungen von $(x_i)$ und $(y_i)$ 
gegeben.

\begin{lstlisting}
    X   26  23  27  28  24  25
    Y   179 150 160 175 155 150
\end{lstlisting}
\begin{enumerate}
    \item Berechnen Sie die Kleinste-Quadrate-Schätzer für $\beta_0$ und $\beta_1$
        der einfachen linearen Regression $Y_i = \beta_0 + \beta_1 x_i + \varepsilon_i$.
    \item Veranschaulichen Sie die Daten und die Regressionsgerade graphisch.
\end{enumerate}

\paragraph*{Lösung.  } Das betrachtete lineare Model hat die Form
\begin{equation}
    Y_i = X_i \beta + \varepsilon_i, \quad i=1,\ldots, 6
\end{equation}
mit der Designmatrix
$$ X^\top = \left(\begin{matrix} 1&1&1&1&1&1\cr 26&23&27&28&24&25\cr \end{matrix} \right) $$
und der Zielvariable
\begin{equation}
    Y^\top = \left( \begin{matrix} 179&150&160&175&155&150\cr   \end{matrix}  \right).
\end{equation}
Die Designmatrix hat also den vollen $\textrm{Rang}(X)=2$ und der Kleinste-Quadrate-Schätzer
ist 
\begin{equation}
    \hat \beta = \left( X^\top X \right)^{-1} X^\top Y.
\end{equation}
Die Berechnung von $\hat \beta$ ist also eine einfache Übung in linearer Algebra. 
\begin{eqnarray}
    X^\top X &=&  
    \left( {
    \begin{array}{cc}
        6 & 153 \\ 153 & 3919
    \end{array} }
    \right) \\
    \left(  X^\top X\right)^{-1} &=& \frac{1}{105} 
    \left( {
    \begin{array}{cc} 
        3919&-153 \\ -153&6
    \end{array} }
    \right) \\
    \left(  X^\top X\right)^{-1} X^\top Y &=& \left(
    \begin{array}{c}
        \frac{1343}{35} \\ \frac{169}{35}
    \end{array}
    \right)
\end{eqnarray}
Wir erhalten also 
\begin{eqnarray}
    \beta_0 &=&  38.37142857142857 \\
    \beta_1 &=&  4.828571428571428
\end{eqnarray}

Die Daten und die Regressionsgerade können mit Hilfe des folgenden R-Codes veranschaulicht werden. 
\begin{lstlisting}
X=c( 26, 23, 27, 28, 24, 25 )
Y=c( 179, 150, 160, 175, 155, 150  )
b_0 = 38.37142857142857
b_1 = 4.828571428571428

t= seq(23,28,0.01)

pdf("ue-blatt-14-reg.pdf")
plot(X,Y); lines(t, b_0 + b_1 * t, lw=2)
dev.off()
\end{lstlisting}

\begin{figure}[htb]
    \begin{center}
        \includegraphics[width=\textwidth]{figures/ue-blatt-14-reg.pdf}
    \end{center}
    \caption{Regression von $X$ gegen $Y$}
    \label{fig:reg}
\end{figure}







\paragraph{Quadratisches Modell. }  Es sei das Modell
\begin{equation}
    Y_i = \beta_0 + \beta_1 x_i + \beta_2 x_{i}^2 + \varepsilon_i
\end{equation}
für $i=1,\ldots,n$ gegeben. Die zufälligen Fehler $\varepsilon_1,\ldots,\varepsilon_n$ 
seien i.i.d.\ und $\varepsilon_i\sim \mathcal N(0,\sigma^2)$ mit $\sigma>0$.
Geben Sie die Kleinste-Quadrate-Schätzer für $\beta_i$, $i\in \left\{ 0,1,2 \right\}$ explizit an.

\paragraph*{Lösung. } Das Modell lässt sich in der Form
\begin{equation}
    Y = X \beta + \varepsilon
\end{equation}
schreiben. Wir nehmen an, dass die Beobachtungen $\left( x_1,\ldots, x_n \right)$ 
voneinander verschieden sind und erhalten die Designmatrix
\begin{equation}
    X^\top = \left(
    \begin{array}{cccc}
        1       & 1     & \ldots & 1 \\
        x_1     & x_2   & \ldots & x_n \\
        x_1^2   & x_2^2 & \ldots & x_n^2
    \end{array}
    \right)
\end{equation} 
mit $\textrm{Rang}(X) = 3$. Daraus ergibt sich
\begin{equation}
    X^\top X = \left(
    \begin{array}{ccc}
        \sum_{i}^{} x_i^0 & \sum_{i}^{} x_i^1 & \sum_{i}^{} x_i^2 \\
        \sum_{i}^{} x_i^1 & \sum_{i}^{} x_i^2 & \sum_{i}^{} x_i^3 \\
        \sum_{i}^{} x_i^2 & \sum_{i}^{} x_i^3 & \sum_{i}^{} x_i^4 \\
    \end{array}
    \right)
\end{equation}
und mit der Bezeichnung $T_k = \sum_{i}^{} x_i^k$ 
\begin{eqnarray}
    \Delta &=&  \det \left( X^\top X \right) \\
    &=& { T_0}\,\left({ T_2}\,{ T_4}-{ T_3}^2\right)-{ T_1}\,\left({ T_1}\,{ T_4}-{ T_2}\,{ T_3}\right)+{ T_2}\,\left({ T_1}\,{ T_3}-{ T_2}^2\right).
\end{eqnarray}
Die Regel von Cramer liefert die Inverse der Matrix $X^\top X$, nämlich
\begin{eqnarray}
    \left( X^\top X \right)^{-1} &=& \Delta^{-1} \left(
    \begin{array}{ccc}
{ T_2}\,{ T_4}-{ T_3}^2&{ T_2}\,{ T_3}-
 { T_1}\,{ T_4}&{ T_1}\,{ T_3}-{ T_2}^2\cr { T_2}\,
 { T_3}-{ T_1}\,{ T_4}&{ T_0}\,{ T_4}-{ T_2}^2&
 { T_1}\,{ T_2}-{ T_0}\,{ T_3}\cr { T_1}\,{ T_3}-
 { T_2}^2&{ T_1}\,{ T_2}-{ T_0}\,{ T_3}&{ T_0}\,
 { T_2}-{ T_1}^2
    \end{array}
    \right). 
\end{eqnarray}
Sei $Z = X^\top Y$, d.h.\
\begin{equation}
    Z = \left(
    \begin{array}{c}
        Z_1 \\ Z_2 \\ Z_3
    \end{array}
    \right) = \left(
    \begin{array}{c}
        \sum_{i}^{} Y_i \\ \sum_{i}^{} Y_i x_i \\ \sum_{i} Y_i x_i^2
    \end{array}
    \right).
\end{equation}
Der Kleinste-Quadrate-Schätzer 
$\hat \beta = \left( X^\top X \right)^{-1} X^\top Y = \left( X^\top X \right)^{-1} Z$ ist dann explizit
gegeben durch
\begin{equation}
    \beta_0 = 
\frac{\left({ T_1}\,{ T_3}-{ T_2}^2\right)\,{ Z_2}+
 \left({ T_2}\,{ T_3}-{ T_1}\,{ T_4}\right)\,{ Z_1}+
 \left({ T_2}\,{ T_4}-{ T_3}^2\right)\,{ Z_0}}{
 \left({ T_0}\,{ T_2}-{ T_1}^2\right)\,{ T_4}-{ T_0}\,
 { T_3}^2+2\,{ T_1}\,{ T_2}\,{ T_3}-{ T_2}^3},
\end{equation}
\begin{equation}
    \beta_1 = 
-  \frac{\left({ T_0}\,{ T_3}-{ T_1}\,{ T_2}\right)\,
 { Z_2}+\left({ T_2}^2-{ T_0}\,{ T_4}\right)\,{ Z_1}+
 \left({ T_1}\,{ T_4}-{ T_2}\,{ T_3}\right)\,{ Z_0}
 }{\left({ T_0}\,{ T_2}-{ T_1}^2\right)\,{ T_4}-
 { T_0}\,{ T_3}^2+2\,{ T_1}\,{ T_2}\,{ T_3}-{ T_2}^
 3},
\end{equation}
und
\begin{equation}
    \beta_2 = 
 \frac{\left({ T_0}\,{ T_2}-{ T_1}^2\right)\,{ Z_2}+
 \left({ T_1}\,{ T_2}-{ T_0}\,{ T_3}\right)\,{ Z_1}+
 \left({ T_1}\,{ T_3}-{ T_2}^2\right)\,{ Z_0}}{
 \left({ T_0}\,{ T_2}-{ T_1}^2\right)\,{ T_4}-{ T_0}\,
 { T_3}^2+2\,{ T_1}\,{ T_2}\,{ T_3}-{ T_2}^3}.
\end{equation}

